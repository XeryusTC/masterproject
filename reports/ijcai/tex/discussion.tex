\section{Discussion}\label{sec:discussion}
\begin{table*}[h]
    \centering
    \caption{Comparison of several cooperative pathfinding algorithms.}
    \label{tbl:planning-overview}
    \begin{tabular}{ll|l|l|l|l|l}
        & & Category & Complete & Priority & Comm. & Eval. \\
        \hline
        OD+ID & \cite{standley2010,standley2011} & Centralized & Yes & No &
        All & No \\
        ICTS & \cite{sharon2013} & Centralized & Yes & No & All & No\\
        IADPP & \cite{cap2012} & Decoupled & No & Yes & All & No \\
        WHCA* & \cite{silver2005} & Decoupled & No & Yes & Window & No \\
        DMRCP & \cite{wei2016} & Decentralized & No & No & 2 nodes & No
        \\
        DiMPP & \cite{chouhan2017} & Decentralized & Yes & Yes & Ring & No
        \\ \hline
        PCA* & & Decentralized & No & Yes & All & No \\
        \multicolumn{2}{l|}{DPCA* / DPCA*+} & Decentralized & No & Yes & All & 
        Yes \\
    \end{tabular}
\end{table*}

A comparison of the features of various cooperative pathfinding algorithms from 
the literature and those presented here is given in 
\autoref{tbl:planning-overview}. Algorithms are \emph{categorized} by how  they 
calculate a set of paths. They can be centralized, decoupled or decentralized. 
Whether an algorithm is guaranteed to find a solution if one exists is 
indicated by the \emph{complete} column. Some algorithms impose a hierarchy on 
the agents, this is indicated by the \emph{priority} column. The 
\emph{communication} column indicates restrictions on which agents are allowed 
to communicate with each other. Finally the \emph{evaluation} column means that 
agents can vote on a plan.


% TODO: scalability
% TODO: more compartmentalized than decoupled methods
% TODO: discuss WHCA* vs Continual Planning vs WDPCA*
% TODO: talk about agents changing destinations and how WDCPA* helps.
% TODO: extracting reasons for why solution was settled on
% TODO: going back to previous conflicts

The performance graph in \autoref{fig:perfgraph} shows that PCA* took the most 
time and solved the fewest problem instances. PCA* requires more time that 
OD+ID for most instances. This is because the algorithm 
evaluates all permutations of partial priority orders and selects one that has
minimal cost, there is no consideration for the side-effects of the partial 
priority order. This led to the development of DPCA* and DPCA*+. These two 
algorithms are the fastest among those that were evaluated. This contrast 
between PCA* and DPCA* / DPCA*+ shows the effectiveness of deliberation 
dialogues in this setting.

There is little difference between DPCA* and DPCA*+ in \autoref{fig:perfgraph}. 
From \autoref{fig:solved} we can see that DPCA* is able to solve more problem 
instances while \autoref{fig:lengths} shows that it finds slightly longer paths 
than DPCA*+. We looked at the average number of agents that participate in 
dialogues as shown in \autoref{fig:conflict-sizes}. From that figure we see the 
vast majority of dialogues involve two agents, only a few involve three or four 
agents. Dialogues with more than three agents require extra complexity in the 
arguments and evaluations that agents can make. This extra complexity uses more 
computational resources while it is rarely needed. The benefit that the added 
complexity of DPCA*+ has is that it finds solutions with shorter paths.

During a DPCA* DPCA*+ dialogue agents can put forward arguments for or against 
partial 
priority orders. They also evaluate and vote on each proposal based on several 
criteria. This gives the agents some power over which solution 
is picked for a problem instance. The arguments and evaluations can be used by 
an outside observer to explain why a group of agents have picked a particular 
solution. Conventional algorithms find an abstract solution based on minimal 
cost. 

The four stages of DPCA* and DPCA*+ deliberation dialogues are based on the 
four stages of \textsc{TeamLog}~\cite{dunin-keplicz2011}. Some of the stages of 
McBurney \emph{et al.}~\shortcite{mcburney2007} are not necessary to find 
priority orders for the agents, most of its verbosity is encoded in 
\textsc{TeamLog}. By basing our algorithms on the decoupled method we separated 
finding paths from 
the argumentative process. The ordering proposals influence which possible 
paths are valid but they do not directly alter a path. It is possible to use 
\textsf{DeLP-MAPOP} to integrate planning and the deliberation process more 
closely. In this case the agents can discuss individual actions in a plan and 
propose alternate courses of action. This would result in an algorithm that 
would be similar to a distributed version of OD+ID. It would also mean that the 
paths are more tightly coupled than they are in our proposal. As a result 
agents would be less flexible to change their plan to resolve other conflicts.

Partial global planning ensures that all of our proposed algorithms 
incrementally build towards a global well coordinated plan from the optimal 
plans of individual agents. Because we incrementally construct a global plan we 
can start deliberation dialogues for each local conflict. Here partial global 
planning bridges the gap between conventional decoupled approaches and 
computational argumentation. The norm in cooperative pathfinding is to 
calculate a global priority ordering. Partial global planning allows us to 
create partial priority ordering that imply a global priority ordering.

The family of algorithms we presented have been applied to an abstract setting 
that only allows for basic arguments and evaluations to made during a 
dialogue. Potential applications like traffic management are more complex and 
can allow domain specific arguments. For example, in an air traffic control 
system the agents could make an argument for a high priority if their fuel 
levels are low.
In such a case it is also possible to determine why a certain solution was 
arrived at. Currently that is mainly possible by inspecting the evaluations 
made during dialogues. When the arguments made during a dialogue play a larger 
role in finding the eventual solution then these arguments can be used to 
extract reasons for why the solution is the most appropriate.

\section{Conclusion}
We have combined ideas from cooperative pathfinding, computational 
argumentation and multi-agent planning to propose three algorithms that can 
find conflict free paths for groups of mobile agents. Agents engage in 
dialogues to resolve conflicts in their optimal plans. The local views of 
agents are incrementally combined to form a global well-coordinated plan.
In contrast to other cooperative pathfinding algorithms it is possible for 
agents to motivate why they arrived at a particular solution.
Our algorithms are faster and can solve instances with more agents than a 
centralized state of the art algorithm and a recent decentralized algorithm.
We also found that agents resolving conflicts in pairs is slightly faster than 
solving them in larger groups.
