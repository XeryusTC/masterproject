\section{Experimental evaluation}\label{sec:results}
% TODO: investigate run time vs number of conflicts found

The three proposed algorithms are tested and compared against the complete
algorithm OD+ID.

\subsection{Experimental setup}
To compare the algorithms they are all given a large set of cooperative
pathfinding problems that have be to solve. Each problem consists of a $16
\times 16$ 8-connected grid. Each grid location has a $20\%$ chance of being an
impassable obstacle. Agents cannot enter these grid locations, but the
obstacles do not block agents from moving along diagonals as specified in
\autoref{sec:problem}. Agents are randomly placed in the grid such that no two
agents have the same starting position, each agent is also given a randomly
chosen destination location, again these are chosen in such a way that no two
agents have the same destination. There are $10000$ problems in the set, each
of which has between 2 and \agentsupb agents in it. The number of agents in the
problems form a uniform distribution.

% TODO: add reference to performance graph origin
Each algorithm solves each instance in the set of problems, the time it takes
to do this is recorded. There is a time limit of 2000 ms to solve a single
instance. When the run times are sorted in ascending order they can be plotted
in a \emph{performance graph} as in \autoref{fig:perfgraph}. The $x$-axis shown
the number of the sorted instance while the $y$-axis shows the time it takes an
algorithm to solve that instance. Because the instances are sorted it is not
necessarily the case that the $n$th instance for one algorithm is the same as
for a different algorithm. The graph gives several pieces of information. The
first is a comparison of run times for the different algorithms, a lower line
means that an algorithm was able to solve problem instances quicker. At the
same time the graph also shows how many instances an algorithm can solve within
the 2000 ms time limit. Instances that were not solved are not included in the
graph, so where the graph crosses the top indicates how many of the problems
were solved by the algorithm.

%The time it takes each algorithm to solve a problem is recorded

%Each algorithm solves every problem in the set of problems, . The time it
%takes to do this is recorded. When