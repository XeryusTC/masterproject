\section{Introduction}\label{sec:intro}
Path planning for a single agent is considered a solved problem
\cite{sharon2013}. When multiple agents have to find their way through a
shared
space (\autoref{fig:world}), then a level of complexity is added to the problem 
of path planning for single agents. The agents have to find paths
around obstacles while they also need to ensure that they do not collide with
each other. Even when agents can prevent collisions then it is still possible
that congestions or even deadlocks may occur. Some form of coordination between 
agents is needed to avoid such undesirable situations. This problem has been 
referred to as cooperative pathfinding, or multi-agent pathfinding. It is 
encountered in robotics, aviation, road traffic management, crowd simulations, 
and video games \cite{standley2011}.

\begin{figure}[t]
    \centering
    \def\svgscale{.5}
    \input{images/world.pdf_tex}
    \caption{An environment shared by multiple agents. Obstacles are black,
        agents are circles inscribed with the agent's number ($a_i$). The 
        destination for agent $a_i$ is given by $g_i$.}
    \label{fig:world}
\end{figure}

\begin{figure}[t]
    \centering
    \begin{subfigure}[b]{.18\textwidth}
        \centering
        \def\svgscale{.5}
        \input{images/conflict1.pdf_tex}
        \caption{Moving to the same position.}
        \label{fig:conflict-position}
    \end{subfigure}
    ~
    \begin{subfigure}[b]{.13\textwidth}
        \centering
        \def\svgscale{.5}
        \input{images/conflict2.pdf_tex}
        \caption{Moving along the same edge.}
        \label{fig:conflict-same}
    \end{subfigure}
    ~
    \begin{subfigure}[b]{.13\textwidth}
        \centering
        \def\svgscale{.5}
        \input{images/conflict3.pdf_tex}
        \caption{Moving on crossing edges.}
        \label{fig:conflict-crossing}
    \end{subfigure}
    \caption{Examples of conflicting actions. Agents are circles inscribed with
        $a_i$. Their movements are indicated by the arrows starting in the cell
        they occupy, the action ends in the cell that the arrow points to.}
    \label{fig:conflicts}
\end{figure}

The problem of cooperative pathfinding can be defined as follows. A shared
space is divided into discrete cells such that it forms an 8-connected grid.
Some of the cells in this grid are static obstacles while the other cells are
open. A set of $k$ agents $\{a_1, \ldots, a_k\}$ occupy cells within the grid,
the agents have respective goal positions $g_1, \ldots, g_k$. A set of paths
need to be found, one for each agent, such that each agent gets to its goal
position without colliding with any of the other agents. A path consists of a
series of actions. An action can either be to move to one of the eight
neighbouring cells or \emph{wait} at the current location. Each time step an
agent must do exactly one of these actions. Each action has unit cost, with the
exception of waiting in the goal position which has zero cost. The cost 
function is then
\[
\text{\textsc{cost}(P,Q)} =
\begin{cases}
0 & \text{if } P = Q = G \\
1 & \text{otherwise}
\end{cases}
\]
where $P$ is the node where the agent's location node, $Q$ the node where the
agent moves to, and $G$ the agent's goal node. A single agent's path has a
cost that is the sum of the costs of all its actions. The cost of a solution is
defined as the sum of the costs of the paths of the agents. An optimal solution 
has minimal cost.

The paths of two agents are not in conflict iff at no time step the agents
occupy the same cell, the agents move along the same edge (swap positions), or
the agents move along crossing edges. Obstacle cells can be considered as
stationary agents. Examples of each of these conflicts are given in
\autoref{fig:conflicts}. A single action can result in an agent having multiple
conflicts at the same time. If \autoref{fig:conflict-position} had an agent
$a_3$ in the top right cell moving to the bottom middle cell then $a_2$ would
have a conflict with both $a_1$ and $a_3$ at the same time. Agents are allowed
to move along a diagonal even when the two cells on the opposing diagonal are
blocked, i.e. $a_5$ in \autoref{fig:world} can move to its destination in a
single time step. An agent $a_i$ can move to a cell occupied by agent $a_j$
given that $a_j$ will move to a different cell at the same time. Agents $a_1$,
$a_2$, $a_3$ and $a_4$ in \autoref{fig:world} can reach their respective
destinations in a single time step by ``rotating'' clockwise. Agents $a_7$ and
$a_8$ cannot move to their destinations in a single time step because that
would mean that they move along crossing edges at the same time.

The rest of this article is structured as follows.
We briefly discuss previous research in cooperative pathfinding, computational 
argumentation and multi-agent coordination in \autoref{sec:related}. Our famoly 
of algorithms combining ideas from these three fields is proposed in 
\autoref{sec:method}. This is followed by an evaluation and comparison to other 
algorithms in \autoref{sec:results}. The results are discussed in 
\autoref{sec:discussion}.
