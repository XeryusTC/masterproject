\section{Related Work}\label{sec:related}

\subsection{Argumentation}
\subsubsection{Dialogues}
Multiple agents can have an argument through a dialogue. Walton and
Krabbe \cite{walton1995} proposed a typology of main dialogues that humans
partake in. They distinguish six main types of dialogues, it should be noted
that the list of dialogue types is not exhaustive. In \emph{information
    seeking} dialogues some of the participating agents aim to gather
    information
from another agent that knows the anser. In \emph{inquiry} dialogues a group of
agents collectively seeks an answer to a question to which non of the
participating agents knows the answer on its own. \emph{Deliberation} dialogues
are about what course of action to take in a given situation. A
\emph{persuasion} dialogue occurs when an agent tries to convince on or
multiple other agents of its position. It is successful when the other agent(s)
adopt its position. Participants of \emph{negotiation} dialogues try to find a
division of a scarce resource that all agents can be satisfied with. Finally
\emph{eristic} dialogues are a verbal substitute for fighting. Note that most
actual dialogues combine these dialogue types.

Dialogues are often analysed in a game-theoretic sense, where the utterances
that agents can make are analogous to the moves in a game. Which utterances are
appropriate at each moment is then defined by the rules of the game. Most of
the research into dialogues follows this approach
\cite{prakken2006,prakken2009}. Most dialogue systems have a two language
set-up. The first is the topic language which is about what agents are
discussing and is typically a formal logic, it defines the context of the
dialogue. The second language is the communication language which specifies
which utterances can be made, what effects they have and the rules of outcome.
This latter language is at the core of dialogue games. Most dialogue systems
have the following syntax in common \cite{prakken2006,prakken2009,mcburney2009}.
\begin{description}
    \item[Commencement rules] Rules that concern when and how a dialogue can
    start and what its context is.
    \item[Locution rules] Which utterances are permitted are known as the
    locution rules. They may also define when an utterance is obligatory.
    Common locutions include asserting propositions, questioning or contesting
    assertions and justifying previous assertions after they have been
    questioned.
    \item[Commitments] Some locutions incur commitments on an agent which are
    subsequently put into the agent's commitment store. A dialogue system may
    limit which utterances an agent can make based on what is in its commitment
    store.
    \item[Speaker order] Most dialogue systems specify an order in which agents
    can speak, this can range from agents alternating turns to each agent being
    allowed to make an utterance at any time.
    \item[Outcome rules] These determine what the outcome of the dialogue is.
    Some systems define an outcome while the dialogue may continue and lead to
    a different outcome at a later point.
\end{description}

One model of deliberation dialogues is presented in \cite{mcburney2007}. It
consists of eight stages, starting with an \textbf{Open} stage and ending with
a \textbf{Close} stage. The other stages can occur multiple times during a
dialogue in any valid order. During the dialogue agents will collect the
preferences, goals and other constraints that need to be considered. Agents
will then propose common plans of action, when multiple plans have been
proposed agents can specify which they prefer. At some points an agent can
recommend a plan after which all agents will vote for that plan. The dialogue
requires unanimity before the recommended plan is adopted, but it allows for a
voting mechanism to pick the most preferred plan among many. There are variants
of this model that require fewer stages while still allowing for the same
expressiveness \cite{dunin-keplicz2011}.
