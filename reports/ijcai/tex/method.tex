\section{Family of algorithms}\label{sec:method}
We now discuss the algorithms proposed in this paper connecting to work in 
computational argumentation and multi-agent coordination. Each algorithm is 
based on a variant of A*~\cite{hart1968} called Cooperative 
A*~\cite{silver2005}, it disallows expanding nodes if it 
would cause a conflict with the path of a higher priority agent.
%Decoupled algorithms are able to solve cooperative pathfinding problems while
%requiring only minimal computational resources. The agents calculate their path
%individually so this is inherently distributed. Calculating a hierarchy needs
%to be done centrally so all dependencies between agents can be taken into
%account \cite{latombe1991,bennewitz2002}. This means that there is a single
%centralized bottleneck in an otherwise distributed system. In this article we
%aim to overcome this bottleneck and make the calculation of the priority
%ordering distributed as well. Below are the details of two different versions
%of this algorithm. The second version version of the algorithm has some
%improvements over the previous version. A summary of their properties is given
%in \autoref{tbl:proposed}. Each algorithm is build upon Cooperative A*, a
%variation on A* \cite{hart1968} which allows teams of agents to cooperate.
%This variant does not allow agents to expand nodes if it would mean that the
%action will conflict with the path of an agent with higher priority.

%\begin{table}
%    \centering
%    \caption{Comparison of proposed cooperative pathfinding algorithms. The
%        columns have the same meaning as those in
%        \autoref{tbl:planning-overview}. Both algorithms are decentralized and
%        not complete. The valuations column indicates whether agents vote on
%        their preferred solution.}
%    \label{tbl:proposed}
%    \begin{tabular}{l|l|l|l}
%        & Comm. & Online & Valuations \\ \hline
%        PCA*   & All & No & No \\
%        DPCA*  & All & No & Yes \\
%    \end{tabular}
%\end{table}

\subsection{Partial Cooperative A* (PCA*)}

\begin{algorithm}[t]
	\caption{Partial Cooperative A*}
	\label{alg:pca}
	\begin{algorithmic}
		\State $path \gets \Call{FindPath}{\emptyset, \emptyset}$
		\State $permanent \gets \emptyset$
		\While{\Call{HasConflict}{}}
			\State $c \gets $ \Call{EarliestConflict}{}
			\State $orders \gets$ \Call{PriorityOrders}{$c$}
			\State $cost \gets \emptyset$
			\ForAll{$order \in orders$}
				\State $path \gets \Call{FindPath}{permanent, order}$
				\State $cost(order) \gets$ \Call{Cost}{$path$}
			\EndFor
			\State $permanent \gets permanent \cup \{ \argmin{cost} \}$
			\State $path \gets \Call{FindPath}{permanent, \emptyset}$
		\EndWhile
	\end{algorithmic}
\end{algorithm}

PCA* follows the approach of decoupled methods combined with partial 
global planning, it is outlined in \autoref{alg:pca}. Initially the agents 
calculate their optimal paths and share them with each other. This allows
each agent to determine where they have conflicts. Next the
agents will resolve each conflict in order of their occurrence and update their 
plan accordingly. Agents always start resolving the conflict that occurs the 
earliest in their path. This is because solving a conflict may have the 
side-effect of solving conflicts that occur at a later time. To resolve a 
conflict all possible partial priority order permutations for the agents 
involved in the conflict are evaluated. The partial priority order that has the 
lowest increase in solution cost (sum of path lengths) is permanently adapted 
by the agents. The plans of the involved agents are updated accordingly. This 
process is repeated until all conflicts have been resolved. Resolving the 
conflict locally leads to the incremental construction of a global plan. If 
there is no partial priority order that resolves the conflict then PCA* has 
failed to find a solution.

Conventional decoupled algorithms calculate a permutation of the full priority 
ordering $a_1 > a_2 > \ldots > a_k$ that is assigned to all agents. The partial 
priority orderings used by PCA* only include agents that are involved in a 
single conflict. Because of this an agent only has to keep track of which 
agents it has to yield to, it does not need to concern itself with the 
relative order of other agents. A global priority order is implied by the 
combination of the local views of the agents. Some agents may not be involved 
in any conflict and are not part of any partial priority order. 
They will also not be present in the implied global priority order. Similarly 
it is possible that disjoint subsets of partial priority orders exist when the 
agents in one subset never had to resolve a conflict with the agents in the 
other subset(s). There has never been communication past sharing the initial 
optimal paths, the result of which is that there is implied Independence 
Detection.

\subsection{Dialogue-based Partial Cooperative A* (DPCA*)}

\begin{algorithm}[t]
	\caption{Dialogue during DPCA* / DPCA*+}
	\label{alg:dpca}
	\begin{algorithmic}
		\State $path \gets \Call{FindPath}{(\emptyset, \emptyset)}$
		\State $permanent \gets \emptyset$
		\While{\Call{HasConflict}{}}
			\State $c \gets \Call{EarliestConflict}{}$
		\EndWhile
	\end{algorithmic}
\end{algorithm}

\begin{table}
    \centering
    \caption{Stages of a conflict resolution dialogue.}
    \label{tbl:stages}
    \begin{tabularx}{\columnwidth}{l|X|l}
        Stage & Purpose & Next stage \\ \hline
        Opening & Exchange information & Proposal \\
        Proposal & Make (incomplete) priority proposals & Evaluation \\
        Evaluation & Vote on suitability of proposals & Proposal, Closing \\
        Closing & Permanently adapt best proposal & \\
    \end{tabularx}
\end{table}

Going through all possible permutations and evaluating them on a single
criterion is not the most optimal method of finding a priority ordering
\cite{bennewitz2002}. A path can be considered an argument for the usage of a 
resource, namely being in a series of position at certain times, then a 
conflict means that two or more of these arguments are attacking each other. A 
priority order $a_1 > a_2$ is then an argument which supports the path argument 
of $a_1$ by attacking that of $a_2$. To find the most appropriate priority 
order the agents can engage in a conflict resolving deliberation dialogue. The 
algorithms this results in are called DPCA* and DPCA*+. Each deliberation 
dialogue consists of four stages 
as summarized in \autoref{tbl:stages}, these stages are based on those of 
\textsc{TeamLog}. Each conflict is resolved in a 
dedicated dialogue which starts in the opening stage. Only agents that are 
involved in the conflict participate in the dialogue. During the opening stage 
agents can put the dialogue on hold if they are involved an earlier occurring 
conflict. If this is not the case for any of the agents then the dialogue moves 
to the proposal stage.
During the proposal stage agents can make partial priority order proposals. In 
our implementation agents always propose that they get priority over the other 
agents. It is valid for a proposal to assign agents an equal rank if that is 
the lowest rank, so the 
proposal $a_1 > a_2 = a_3$ is valid. Each proposal supports the path of the 
highest ranked agent by attacking the path arguments of lower ranked agents. 
The proposals made also attack each other.

After all agents have made a 
proposal or have declined to make one then the dialogue moves to the evaluation 
stage. Here the agents vote on the priority order proposals to indicate their 
preference for a solution. These votes can be based on several weighted
criteria, in our case these are the increase in path length caused by 
adopting the proposal and the difference in the number of conflicts that the
agent is involved in. To determine these factors each 
agent temporarily adapts the proposed priority order. The agent with the 
highest priority uses its original path while the agents of lower priority 
calculate new paths that satisfy the constraints set by the proposal and 
previous permanently adapted proposals. After all 
agents have updated their path they evaluate it and cast a vote on the 
proposal to indicate their preference. If an agent is not able to find a path 
then it can also make an argument that directly attacks the proposal, in this 
case the proposal is not a valid solution. This process is repeated for all 
proposals made during the previous proposal stage. The proposal that received 
the lowest sum of votes is 
the most preferred by the group of agents, this proposal defeats all other 
proposals. During the evaluation agents can also claim that they want to refine 
a priority order that assigns some agents an equal rank. This happens when the 
equally ranked agent still have a conflict when adapting the proposal. They 
also weigh this in their vote, effectively penalizing proposals that do not 
solve the conflict for all involved agents. If an 
agent makes this claim then additional proposal round follows in which agents 
can refine any of the priority proposals that assign equal rank to two or more 
agents. This in turn is followed by another evaluation stage.

If none of the agents have requested an additional proposal round then the 
dialogue can move to the closing stage. Agents now permanently adapt the 
priority order which was the most preferred. The agents update their plan to 
satisfy the constraints set by the accepted partial priority order and 
previously adapted priority orders. These new 
plans are communicated to all other agents so that they can find any new 
occurring conflicts, and cancel dialogues for conflicts that have been 
indirectly solved. The entire process of resolving dialogues 
continues until there are no more conflicts.

Conflicts that involve more than two agents can be solved in two different
ways. The most simple is to let agents solve conflicts in pairs, this is known
as Dialogue-based Partial Cooperative A* (DPCA*). A conflict which involves
three agents requires three dialogues, however it is possible that the 
completion of two of these dialogues makes the third obsolete. Instead of three 
separate dialogues it is also possible to have a single dialogue with all 
agents involved in the conflict participating, this is known as
Dialogue-based Partial Cooperative A* Plus (DPCA*+). The dialogue process 
described above is that of DPCA*+. DPCA* consists of one proposal and 
evaluation stage because only two priority orders are valid when there 
are just two agents participating in a dialogue.
This makes DPCA*+ more complex than DPCA*, but it also requires fewer dialogues 
to find a solution and therefore it may be faster than DPCA*.

Each time that agents evaluate a proposal they have to compute paths that
satisfy the constraints imposed by the priority ordering, often recomputing
the same paths. To reduce the amount of duplicate computation the
agents store the paths that they have calculated. Before calculating a path
the agents consult their path store for a path path which satisfies the
constraints imposed by the temporary and permanent priority orderings. In the 
case
that there is such a path and it does not cause any conflicts with higher
priority agents then it is used. If this is not the case then the agent
calculates a new path which does satisfy the constraints. An agent can use a 
cached path even if it causes
conflicts with lower priority agents or agents with which it has not
established a priority ordering with.

%\begin{figure*}
%    \centering
%    \begin{subfigure}[t]{.3\textwidth}
%        \centering
%        \def\svgscale{.6}
%        \input{images/dialogue-example-initial.pdf_tex}
%        \caption{Initial configuration.}
%        \label{fig:example-initial}
%    \end{subfigure}
%    ~
%    \begin{subfigure}[t]{.3\textwidth}
%        \centering
%        \def\svgscale{.6}
%        \input{images/dialogue-example-prop1.pdf_tex}
%        \caption{Configuration after proposal $a_2 > a_1$.}
%        \label{fig:example-prop1}
%    \end{subfigure}
%    ~
%    \begin{subfigure}[t]{.3\textwidth}
%        \centering
%        \def\svgscale{.6}
%        \input{images/dialogue-example-solution.pdf_tex}
%        \caption{Configuration after proposal $a_1 > a_2$.}
%        \label{fig:example-solution}
%    \end{subfigure}
%    \caption{Three stages of resolving a conflict. Agents are circles inscribed
%    by $a_i$, their respective goals are $g_i$. Paths that are followed are
%    indicated by the arrows. The red dot represent where two paths have
%    conflicting moves.}
%    \label{fig:example}
%\end{figure*}
%
%\paragraph{Example of conflict resolution} Consider the cooperative pathfinding
%problem in \autoref{fig:example-initial}. It shows a $4 \times 4$ grid with
%three agents in it in their starting positions. The initial optimal paths they
%found during the first step of the algorithm are shown as arrows.
%Agent $a_1$ has a path that consists of three \emph{south} moves, $p_{a_{1},1}
%= \{\text{\emph{south, south, south}}\}$, while both $a_2$ and $a_3$ have paths
%that consist of three consecutive \emph{east} moves, $p_{a_2,1} = p_{a_3,1} =
%\{\text{\emph{east, east, east}}\}$. None of the agents has a \emph{wait}
%action in their path. After finding the initial paths all three agents share
%their paths with each other, agents $a_1$ and $a_2$ discover that they collide
%after their first action. This means that they will have to resolve this
%conflict to prevent the collision.
%
%Before discussing the details of how the agents resolve the conflict in this
%situation some definitions are needed. A proposal where agent $a_i$ has
%priority over a agent $a_j$ is represented as $a_i > a_j$. Agents can cast
%votes on proposals, $\vote_{a_n}(a_i > a_j)$ is the vote of $a_n$ on the
%proposal $a_i > a_j$. To be able to cast votes agents need to evaluate a
%proposal. An evaluation  is based on two effects which can be weighted
%differently. The first effect is the difference in path length before and after
%a proposal has been adopted. The second effect is the change in the number of
%conflicts that an agent is involved in, this effect is weighted three times
%heavier than the former effect. Qualitatively this means that $\vote_{a_n}(a_i
%> a_j) = 1 \cdot \Delta l + 3 \cdot \Delta c$ where $\Delta l$ is the
%difference in path lengths and $\Delta c$ is the difference in conflicts.
%Agents only communicate their final vote and not how they arrived at it.
%
%Continuing with the example, after the agents have determined that they have a
%conflict they start a new dialogue. The dialogue starts in its initial stage,
%the opening. All messages are broadcast to all agents in the conflict dialogue,
%in this case $a_1$ and $a_2$:
%\\ \-\qquad $a_1$: no earlier conflicts
%\\ \-\qquad $a_2$: no earlier conflicts
%
%Because both agents don't have any conflicts that occur at an earlier time
%(this is the first time step) the dialogue can move on to the proposal stage.
%Both agents make a proposal in which they go first:
%\\ \-\qquad $a_2$: propose $a_2 > a_1$
%\\ \-\qquad $a_1$: propose $a_1 > a_2$
%
%Both agents have made a proposal to resolve the conflict. The dialogue can
%move to the evaluation stage. Agent $a_2$'s proposal will be evaluated first
%because $a_2$ submitted the proposal before $a_1$. In this proposal agent $a_2$
%has the highest priority so it doesn't need to update its plan and its path
%remains $p_{a_2,1}$. It updates $a_1$ of the fact that its path hasn't changed
%and still consists of three consecutive \emph{east} actions:
%\\ \-\qquad $a_2$: new path $p_{a_2,1} = \{\text{\emph{east, east, east}}\}$
%
%Agent $a_1$ does have to yield to $a_2$ so it will have to consider possible
%conflicts that arise with $p_{a_2,1}$ and plan around them. It finds a new path
%$p_{a_1,2} = \{\text{\emph{south east, south, south west}}\}$ which is shown in
%\autoref{fig:example-prop1}. After finding the new path $a_1$ will evaluate its
%quality so that it can send its vote to $a_2$. Before adopting the proposal the
%agent had one conflict, this has not changed after making a new plan so $\Delta
%c = 0$. There is no difference in the length of the paths before and after
%temporarily adopting the constraints of the proposal, so $\Delta l = $. This
%means that agent $a_1$ can send the vote to $a_2$. It will also send its new
%path along with the proposal
%\\ \-\qquad $a_1$: new path $p_{a_1,2} = \{\text{\emph{south east, south,
%south west}}\}$
%\\ \-\qquad $a_1$: $\vote_{a_1}(a_2 > a_1) = 0$
%
%Now that $a_2$ knows the new path of agent $a_1$ after adopting to $a_2 > a_1$
%it can also vote on the proposal. This happens in a similar vain as $a_1$'s
%evaluation, there is however one fewer conflict for $a_1$ as its path doesn't
%conflict with that of $a_2$ any more. Achieving this was the goal of the
%dialogue. The path remained the same so $\Delta l = 0$ and $\Delta c = -1$, the
%vote of $a_2$ is then $\vote_{a_2} = 1 \cdot \Delta l + 3 \cdot \Delta c = -3$.
%This vote is then cast:
%\\ \-\qquad $a_2$: $\vote_{a_2}(a_2 > a_1) = -3$
%
%Now that this proposal has been evaluated by both agents they can find the sum
%score of the proposal which is $\eval(a_2 > a_1) = \eval(a_1, a_2 > a_1) +
%\eval(a_2, a_2 > a_1) =
%-3$. Next the agents can evaluate $a_1 > a_2$. When agent $a_1$ goes to
%evaluate this conflict if finds that it can use the path $p_{a_1,1}$ which is
%stored in its cache. It can use this because it has a priority scheme similar
%to the one being evaluated; $a_1$ does not have to yield to any agent. It
%notifies $a_2$ of this:
%\\ \-\quad $a_1$: new path $p_{a_1,1} = \{\text{\emph{south, south, south}}\}$
%
%Next $a_2$ can evaluate the proposal. First it needs to plan a new path that
%does not conflict with the path that $a_1$ has just send. It finds the path
%$p_{a_2,2} = \{\text{\emph{north easth, east, south east}}\}$. The new
%situation is shown in \autoref{fig:example-solution}. It can
%immediately evaluate it, the lengths of $p_{a_2,1}$ and $p_{a_2,2}$ are equal
%and $a_1$ has one fewer conflicts so $\Delta c = -1$.
%\\ \-\quad $a_2$: new path $p_{a_2,2} = \{\text{\emph{north easth, east, south
%east}}\}$
%\\ \-\quad $a_2$: $\eval(a_2, a_1 > a_2) = 1 \cdot (|p_{a_2,2}| - |p_{a_2,1}|)
%+ 3 \cdot \Delta c = -3$
%
%Agent $a_1$ can also send its evaluation to $a_2$:
%\\ \-\quad $a_1$: $\eval(a_1, a_1 > a_2) = 1 \cdot (|p_{a_1,1}| - |p_{a_1,1}|)
%+ 3 \cdot \Delta c = -3$
%
%Now both agents have evaluated the proposal $a_1 > a_2$ they can find the sum
%of the evaluations which is $\eval(a_1 > a_2) = \eval(a_1, a_1 > a_2) +
%\eval(a_2, a_1 > a_2) =
%-6$. All proposals have been evaluated so the agents can notify each other if
%they want to make more proposals:
%\\ \- \quad $a_1$: no more proposals
%\\ \- \quad $a_2$: no more proposals
%
%Neither agent wants to make more priority ordering proposals, this is because
%there are no more possible priority orderings to make with these two agents.
%The dialogue can then move to the closing stage. In the closing stage agents
%will pick the best proposal, in this case that is the proposal with the lowest
%sum score which is the proposal $a_1 > a_2$. To adapt this proposal agents will
%need to use the respective paths that they used during the evaluation of the
%proposal. So $a_1$'s path will be $p_{a_1,1}$ while $a_2$'s path will be
%$p_{a_2,2}$. They also need to keep track of which agents have a higher
%priority than themselves. For agent $a_1$ nothing changes, while $a_2$ must now
%store that $a_1$ has a higher priority than it. All conflicts in
%\autoref{fig:example} have now been solved and agents are free to execute their
%paths.
%
%To evaluate a proposal an agent needs to weigh different factors of the
%quality. In this example a weight of 1 for the path length and a weight of 3
%for the number of solved/introduced conflicts was used. These weights are used
%for demonstration only and an implementation should have these weights set
%empirically. The weights can be any real number. When dialogues between groups
%of three or more agents are possible then a third penalty weight can be added.
%This penalty weight is included in the evaluation if there are two or more
%agents that still have a conflict with each other after adapting a priority
%proposal. It is a penalty for when a proposal only partially solves a conflict.
