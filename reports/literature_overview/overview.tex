\documentclass[a4paper]{article}

\usepackage[utf8]{inputenc}
\usepackage[T1]{fontenc}

\title{MSc Project Literature Overview}
\author{Xeryus Stokkel}
\date{}

\begin{document}

\maketitle

\section{Complete Algorithms for Cooperative Pathfinding 
Problems\cite{standley2010}}
Standley proposes two methods of tackling the state space explosion of 
cooperative pathfinding. The first is known as Operator Decomposition (OD) 
which decreases the complexity of the problem by introducing sub-time-step 
planning to A*. Planning for multiple agents at the same time is usually done 
by taking the Cartesian product of their state-spaces and searching that, this 
has the downside that the branching factor grows exponentially in the number of 
agents. Standley instead has one planning sub-step per agent, this reduces the 
growth of the branching from exponential to linear, but also increases the 
depth of the solution in the search linearly in the number of agents. This is 
however less of a problem.

OD is realised by letting the current state be described as location-action 
pairs for all agents instead of just their locations. Only one agent is given 
an action during each planning steps, leading to incomplete states (not all 
agents have an action). This has the benefit of decreasing the branching factor 
as only the possible actions of one agent have to be considered. It also has 
the benefit of decreasing the required computation for when the heuristic 
distance to the goal would increase. When the heuristic distance increases as 
the result of an agent's action then the new (partial) state will be lower on 
the priority queue and thus the algorithm is less likely to spend computational 
power searching in the wrong direction.

The second part of the paper relates to Independence Detection (ID). This 
decreases the required computation by allowing independent subgroups to plan 
without regard for the other agents, this also has the benefit of decreasing 
the state space. This is achieved by initially assigning all agents to their 
own group. Each group plans independently of the others, but if two groups have 
a conflict then one group has to make a new plan in which it avoids the 
conflict. If it is not possible to find a new plan then the groups are merged 
and a solution like OD is used to plan for all agents in the group 
collectively. OD and ID can be used independent from each other.


\section{Formalizing Value-Guided Argumentation for Ethical Systems 
Design\cite{verheij2016}}
In this paper Verheij starts to tackle the problem of ethical systems design. 
He does this by introducing a formal method for evaluating an argument based on 
the values that it promotes. At the same time he also connects it to the logic 
and structure of the argument.

\section{Cooperative pathfinding\cite{silver2005}}
Before OD+ID the state of the art was WHCA* as introduced by Silver. First he 
introduces Cooperative A* where agents plan independent from each other, but 
they do note their actions in a reservation table to let other agents know 
about their actions. Next he introduces Hierarchical Cooperative A* (HCA*). 
Here an abstract domain is added where agents can plan without regard for the 
reservation table. HCA* also makes use of Reverse Resumable A* (RRA*) to 
calculate the optimal path through the abstract domain, but when the agent runs 
into other agents then RRA* can be used to quickly find an alternative path. 

Finally Windowed Hierarchical Cooperative A* (WHCA*) is introduced to prevent 
several of the problems introduced by the algorithms above. Using WHCA* agents 
plan up to a certain depth (the window) using cooperative search, while the 
rest of the search is done in the abstract domain where agents don't consider 
each other. While agents execute their plans the position of the window will 
periodically be updated so that agents can respond to the non-cooperative part 
of the plans of the other agents.

\section{Argumentation in Artificial Intelligence \cite{bench-capon2007}}
This is a dense review paper. Sections of interest include 3.2, 3.3.

\nocite{*}
\bibliographystyle{plain}
\bibliography{biblio}

\end{document}