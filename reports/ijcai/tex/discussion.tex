\section{Discussion}\label{sec:discussion}
\begin{table*}[h]
    \centering
    \caption{Comparison of several cooperative pathfinding algorithms.}
    \label{tbl:planning-overview}
    \begin{tabular}{ll|l|l|l|l|l|l}
        & & Category & Complete & Priority & Comm. & Online & Eval. \\
        \hline
        OD+ID & \cite{standley2010,standley2011} & Centralized & Yes & No &
        All &
        No & No \\
        ICTS & \cite{sharon2013} & Centralized & Yes & No & All & No & No\\
        IADPP & \cite{cap2012} & Decoupled & No & Yes & All & No & No \\
        WHCA* & \cite{silver2005} & Decoupled & No & Yes & Window & Yes & No \\
        DMRCP & \cite{wei2016} & Decentralized & No & No & 2 nodes & Yes & No
        \\
        DiMPP & \cite{chouhan2017} & Decentralized & Yes & Yes & Ring & No & 
        No
        \\ \hline
        PCA* & & Decentralized & No & Yes & All & No & No \\
        DPCA* & & Decentralized & No & Yes & All & No & Yes
    \end{tabular}
\end{table*}


% TODO: scalability
% TODO: more compartmentalized than decoupled methods
% TODO: discuss WHCA* vs Continual Planning vs WDPCA*
% TODO: talk about agents changing destinations and how WDCPA* helps.
% TODO: extracting reasons for why solution was settled on
% TODO: going back to previous conflicts
% TODO: WDPCA* sacrifices global plan for speed.
% TODO: $w=8$ is early faster because of less replanning

% DPCA*+ and PCA are simular, later is simpler
%With DPCA* agents resolve conflicts in pairs only while DPCA*+ can have larger
%groups of agents resolve a conflict. This only happens when multiple agents try
%to make conflicting moves at the same time. Allowing larger groups in a
%dialogue means that it gets more complex, this is reflected in the fact that
%PPCPF+ is slower overall and on individual problem instances. Larger groups of
%agents solving conflicts is not necessarily faster than pairs of agents solving
%conflicts. Several small pairwise dialogues can be very effective because the
%pair $a_1$ and $a_2$ resolving their conflict means that $a_2$ gets rerouted,
%this means that $a_2$'s conflict with $a_3$ is also resolved. This means that
%$a_1$ and $a_3$ only need to come to a solution. In the case that all three
%agents would find a solution in a single dialogue they will have to evaluate
%multiple proposals. The added complexity of three agents trying to find a
%solution to the conflict does not outweigh the speed of starting several
%dialogues which come to a conclusion quickly.

Of all the evaluated algorithms PCA* was the slowest and was able to solve the 
least amount of problem instances. This is not surprising because the algorithm 
will evaluate different permutations of partial priority orderings. This is 
computationally inefficient. Because PCA* requires more time than OD+ID for 
most instances it seems to be a poor approach to the cooperative pathfinding 
problem. On the other hand  DPCA* and DPCA*+ are able to solve problem 
instances the fastest of all tested algorithms. They have also been able to 
solve more problem instances than any of the other algorithms. This contrast 
between PCA* on one hand and DPCA* and DPCA*+ on the other shows the 
effectiveness of deliberation dialogues in this setting.

There is little difference between DPCA* and DPCA*+ in \autoref{fig:perfgraph}. 
From \autoref{fig:solved} we can see that DPCA* is able to solve more problem 
instances of the two while \autoref{fig:lengths} shows that it finds slightly 
longer paths than DPCA*+. All of this suggest that the algorithms perform on an 
equal level. To investigate this we looked at the average number of agents that 
participate in dialogues as shown in \autoref{fig:conflict-sizes}. From that 
figure we see that just two agents participate in the vast majority of 
dialogues. There are only a few dialogues that involve three or four agents.
Dialogues with more than three agents require some extra complexity in the 
arguments and evaluations that agents can make. This extra complexity uses more 
computational resources while it is rarely needed. The only benefit that the 
added complexity of DPCA*+ has is that it finds solutions with shorter paths.

A comparison of the features of various cooperative pathfinding methods from 
the literature and those presented here is given in 
\autoref{tbl:planning-overview}. Algorithms can categorized by how  they 
calculate a set of paths. They can be centralized, decoupled or decentralized. 
Whether an algorithm is guaranteed to find a solution if one exists is 
indicated by the complete column. Some algorithms impose a hierarchy on the 
agents, this is indicated by the priority column. The communication column 
indicates restrictions on which agents are allowed to communicate with each 
other. An algorithm is online when it interleaves calculating a plan with the 
execution of it. Finally the evaluation column means that agents can vote on a 
plan.