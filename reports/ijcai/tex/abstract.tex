\begin{abstract}
When multiple mobile agents have to move to their destination in a shared
workspace they run the risk of causing congestions and deadlocks or even
colliding with each other. To avoid this they need to coordinate at some level,
cooperative pathfinding is the field that studies algorithms that find
solutions to this problem. In this thesis an algorithm called DPCA* is proposed
that takes ideas developed in the study of argumentation, mainly from
deliberation dialogues. In these dialogues agents propose a course of action to
solve some common problem, it is used here to allow agents to find the best
common solution to a conflict in movements. DPCA* also builds on partial global
planning, a distributed method to create a single plan for a group of agents
where no agent ever has a complete overview of the situation. Partial global
planning is used to develop the algorithm such that it does not rely on a
central point of calculation at any point while calculating a solution. We show
that DPCA* is able to solve cooperative pathfinding problems faster than a
complete and optimal algorithm, but that the quality of the solutions found is
lower. We also show that an online version of the algorithm that uses a window
to limit the range in which agents coordinate is able to solve problem
instances faster, the price for this is that the quality of the solution is
often even lower.
\end{abstract}