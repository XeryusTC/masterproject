\begin{abstract}
When multiple mobile agents have reach their destination in a shared
workspace they run the risk of causing congestions and deadlocks or even
collisions. To avoid this they need to coordinate at some level.
Cooperative pathfinding is the field that studies algorithms that find
solutions to this problem. In this article a family of decentralized algorithms
is proposed that are inspired by ideas studied in argumentation and
multi-agent  coordination. From the study of computational argumentation we use
argumentative deliberation dialogues, in which agents can discuss and resolve
conflicting courses of actions. From the study of multi-agent coordination we
use partial global planning, a distributed method to create a
single plan for a group of agents where no agent has a complete overview of the
situation. This means that the agents are able to find a solution without the
need for a central processor. We show that our family of algorithms is able to
solve cooperative pathfinding problems faster than a complete and optimal
algorithm. The quality of the solutions found is generally lower. We also show
that agents resolving conflicts in pairs instead of in larger groups results in
a more simple algorithm with a small improvement in speed performance.
\end{abstract}
