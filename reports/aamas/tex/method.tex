\section{Family of algorithms}\label{sec:method}
We now discuss the algorithms proposed in this paper. Each algorithm is 
based on a variant of A*~\cite{hart1968} called Cooperative 
A*~\cite{silver2005}, which disallows expanding graph nodes if that 
causes a conflict with the path of a higher priority agent.
%Decoupled algorithms are able to solve cooperative pathfinding problems while
%requiring only minimal computational resources. The agents calculate their path
%individually so this is inherently distributed. Calculating a hierarchy needs
%to be done centrally so all dependencies between agents can be taken into
%account \cite{latombe1991,bennewitz2002}. This means that there is a single
%centralized bottleneck in an otherwise distributed system. In this article we
%aim to overcome this bottleneck and make the calculation of the priority
%ordering distributed as well. Below are the details of two different versions
%of this algorithm. The second version version of the algorithm has some
%improvements over the previous version. A summary of their properties is given
%in \autoref{tbl:proposed}. Each algorithm is build upon Cooperative A*, a
%variation on A* \cite{hart1968} which allows teams of agents to cooperate.
%This variant does not allow agents to expand nodes if it would mean that the
%action will conflict with the path of an agent with higher priority.

%\begin{table}
%    \centering
%    \caption{Comparison of proposed cooperative pathfinding algorithms. The
%        columns have the same meaning as those in
%        \autoref{tbl:planning-overview}. Both algorithms are decentralized and
%        not complete. The valuations column indicates whether agents vote on
%        their preferred solution.}
%    \label{tbl:proposed}
%    \begin{tabular}{l|l|l|l}
%        & Comm. & Online & Valuations \\ \hline
%        PCA*   & All & No & No \\
%        DPCA*  & All & No & Yes \\
%    \end{tabular}
%\end{table}

\subsection{Partial Cooperative A* (PCA*)}

\begin{algorithm}[t]
	\caption{Partial Cooperative A*}
	\label{alg:pca}
	\begin{algorithmic}[1]
		\State $permanent \gets \emptyset$
		\State $path \gets \Call{FindPath}{permanent}$
		\While{\Call{HasConflict}~}
			\State $conflict \gets \Call{EarliestConflict}$
			\State $orderings \gets \Call{PriorityOrders}{conflict}$ 
			\State $cost \gets \emptyset$
			\ForAll{$ordering \in orderings$}
				\State $path \gets \Call{FindPath}{permanent \cup ordering}$
				\State $cost[ordering] \gets \Call{Cost}{path}$
			\EndFor
			\State $permanent \gets permanent \cup \{ \argmin{cost[~]} \}$
			\State $path \gets \Call{FindPath}{permanent}$
		\EndWhile
	\end{algorithmic}
\end{algorithm}

PCA* follows the approach of decoupled methods combined with partial 
global planning and is outlined in \autoref{alg:pca}. Initially the agents 
calculate their optimal paths and share them with each other (\textsc{FindPath}). This allows
each agent to determine whether they have conflicts (\textsc{HasConflict}). Next the
agents will resolve each conflict in ordering of their occurrence and update 
their 
plan accordingly. Conflict resolution starts with a conflict earliest in some path (\textsc{EarliestConflict}). This is because solving a conflict may have the 
side-effect of solving conflicts that occur at a later time. To resolve a 
conflict all possible partial priority ordering permutations for the agents 
involved in the conflict are evaluated (\textsc{PriorityOrderings}). The partial priority ordering that has 
the 
lowest increase in solution cost (\textsc{Cost}, defined as the sum of path lengths) is permanently adopted 
by the agents. The plans of the involved agents are updated accordingly. This 
process is repeated until all conflicts have been resolved. Resolving the 
conflict locally leads to the incremental construction of a global plan. If 
there is no partial priority ordering that resolves the conflict then PCA* has 
failed to find a solution.

Conventional decoupled algorithms calculate a permutation of the full priority 
ordering $a_1 > a_2 > \ldots > a_k$ that is assigned to all agents.
The partial priority orderings used by PCA* only include agents that are 
involved in a single conflict.
An agent only has to keep track of which 
agents it has to yield to, it does not need to concern itself with the 
relative ordering of other agents. 
The local views of the agents imply a global priority ordering.
%A global priority ordering is implied by the combination of the local views of 
%the agents.
Some agents may not be involved 
in any conflict and are not part of any partial priority ordering, 
they will also not be present in the implied global priority ordering. 
Similarly 
it is possible that disjoint subsets of partial priority orderings exist when 
the 
agents in one subset never had to resolve a conflict with the agents in the 
other subset(s). Those disjoint subsets do not communicate past 
sharing their initial optimal paths.

\subsection{Dialogue-based Partial Cooperative A* (DPCA*)}

\begin{algorithm}[t]
    \caption{Dialogue-based Partial Cooperative A* (DPCA*)}
    \label{alg:dpca}
    \begin{algorithmic}[1]
        \Require $topic$: conflict that is to be solved by the dialogue
        
        \State $Permanent \gets \emptyset$
        \State $Path \gets FindPath(Permanent)$
        \State $CommunicatePath(Path)$
        
        \State $Conflicts \gets \Call{FindConflicts}$
        \ForAll{$conflict \in Conflicts$}
            \Comment{Stage 1: Opening~~~~}
%        \If{$topic \neq \Call{EarliestConflict}$}
%        \State $\Call{PutDialogueOnHold}$
%        \EndIf
            \If{$conflict \neq \Call{EarliestConflict}$}
                \State $\Call{PutDialogueOnHold}{conflict}$
                \State continue
             \EndIf
        
            \Comment{Stage 2: proposal~~~}
            \Repeat
                \State $\Call{Propose}$
                
            \Comment{Stage 3: evaluation}
            
            \ForAll{$proposal \in unevaluatedProposals$}
            \State $path \gets \Call{FindPath}{permanent \cup proposal}$
            \State $vote, expand \gets \Call{Evaluate}{path}$
            \State $\Call{CastVote}{vote}$
            \EndFor
            \Until{$\lnot expand$}
        
            \Comment{Stage 4: closing~~~~}
            
            \State $permanent \gets permanent \cup {\argmin \sum votes}$
            \State $Path \gets \Call{FindPath}{Permanent}$
            \State $\Call{CommunicatePath}{Path}$
            \State $Conflicts \gets \Call{FindConflicts}$
        \EndFor
    \end{algorithmic}
\end{algorithm}

%\begin{table}
%    \centering
%    \caption{Stages of a conflict resolution dialogue.}
%    \label{tbl:stages}
%    \begin{tabularx}{\columnwidth}{l|X|l}
%        Stage & Purpose & Next stage \\ \hline
%        Opening & Exchange information & Proposal \\
%        Proposal & Make (incomplete) priority proposals & Evaluation \\
%        Evaluation & Vote on suitability of proposals & Proposal, Closing \\
%        Closing & Permanently adapt best proposal & \\
%    \end{tabularx}
%\end{table}

Evaluating all possible permutations and on a single
criterion is not an optimal method of finding a priority ordering. Intuitively, 
a path may be 
considered an argument for the usage of resources, namely being in a series of 
positions at certain times, then a conflict means that two or more of these path 
arguments are attacking each other. A 
priority ordering $a_i > a_j$ would then be an argument which supports the path 
argument 
of $a_i$ by attacking that of $a_j$. To find the most appropriate priority 
ordering the agents can engage in a conflict resolving dialogue. 
%This 
%dialogue replaces line~5 to line~12 of \autoref{alg:pca}.
%The algorithms this results in are called DPCA* and DPCA*+.
Each deliberation 
dialogue consists of four stages that we have based on those of 
\textsc{TeamLog}~\cite{dunin-keplicz2011}. An outline is given in \autoref{alg:dpca}. The dialogue 
starts with the opening stage, during which agents can notify each other that 
they have a conflict at an earlier time. In this case the dialogue is put on 
hold, otherwise there is a propose stage. In \autoref{alg:dpca} 
this stage is encapsulated by \textsc{Propose}. Agents will always propose that 
they receive the highest priority. It is valid for a proposal to assign agents 
an equal rank if that is the lowest rank, so the proposal $a_i > a_j = a_k$ is 
valid. In subsequent rounds agents can expand a proposal by assigning 
themselves a rank higher than the lowest rank. Agents will decline to make a 
proposal if they cannot expand on any proposal by assigning themselves a rank. 
Agents will not make duplicate proposals.

The dialogue enters the evaluation stage after all agents have made a 
proposal or declined to make one (line~12 to line~17 in 
\autoref{alg:dpca}).
During this stage agents vote on each proposal to indicate their preference.
%During this stage agents will indicate their preference for each proposal by 
%casting votes on them.
A vote is based on the effects of a 
proposal on the paths of the agents. To determine this the agents 
temporarily adapt a proposal and calculate a new path that adheres to the 
constraints set by the proposal.
The vote of an individual agent is calculated using
$ v = w_1 \cdot \Delta l + w_2 \cdot \Delta c + w_3 \cdot e $
where $\Delta l$ is the difference in path length, $\Delta c$ the change in 
number of conflicts the agent is involved in before and after adopting the 
proposal, $e$ represent the \emph{expand} 
flag, and $w_1, w_2, w_3$ are weights for each of these factors. The 
\emph{expand} flag is a penalty that is applied when the agent still has a 
conflict with another agent that is involved in 
the dialogue. An agent sets this flag if it wants to request an additional 
proposal and evaluation stage, during the proposal stage agents can expand on 
partial ordering proposals which assigned agents to an equal rank. An agent can 
directly attack a proposal if it was not able to find a path after adopting it. 
When this happens it is not possible to accept the proposal as a valid solution.

If none of the agents have requested an additional proposal round then the 
dialogue can move to the closing stage. 
By summing the votes for each proposal agents form a consensus on what the best 
solution is. The 
proposal with the lowest sum of votes is accepted and permanently adopted.
%Agents now permanently adapt the 
%priority ordering with the lowest sum of votes.
The agents update their plan to 
satisfy the constraints set by the accepted partial priority ordering and 
previously adapted priority orderings. These new 
plans are communicated to all other agents so that they can find any new 
occurring conflicts, and cancel dialogues for conflicts that have been 
indirectly solved. Dialogues terminate because a finite amount of priority 
order permutations are evaluated. The entire process of 
resolving dialogues 
continues until there are no more conflicts.

Conflicts that involve more than two agents can be solved using two different
strategies. The simplest strategy requires agents to solve conflicts in pairs, implemented 
as Dialogue-based Partial Cooperative A* (DPCA*). A conflict that involves
three agents is solved in three dialogues, though it can happen that the 
completion of two dialogues makes the third obsolete. 
Instead of three 
separate dialogues it is also possible to hold a single dialogue with all 
agents involved in the conflict participating. This is the second conflict resolution strategy implemented as Dialogue-based Partial Cooperative A* Plus (DPCA*+). The dialogue process 
described above is that of DPCA*+. DPCA* consists of one proposal and 
evaluation stage because only two priority orderings ($a_i > a_j$ and $a_j > 
a_i$) 
are valid when there 
are just two agents participating in a dialogue.
This makes DPCA*+ more complex than DPCA*, but it also requires fewer dialogues 
to find a solution and therefore it may be faster than DPCA*.

Each time that agents evaluate a proposal they have to compute paths that
satisfy the constraints imposed by the priority ordering, often recomputing
the same paths. To reduce the amount of duplicate computation the
agents store the paths that they have calculated. Before calculating a path
the agents consult their path store for a path that satisfies the
constraints imposed by the temporary and permanent priority orderings. In the 
case
that there is such a path and it does not cause any conflicts with higher
priority agents then it is used. If this is not the case then the agent
calculates a new path which does satisfy the constraints. An agent can use a 
cached path even if it causes
conflicts with lower priority agents or agents with which it has not
established a priority ordering.

%\begin{figure*}
%    \centering
%    \begin{subfigure}[t]{.3\textwidth}
%        \centering
%        \def\svgscale{.6}
%        \input{images/dialogue-example-initial.pdf_tex}
%        \caption{Initial configuration.}
%        \label{fig:example-initial}
%    \end{subfigure}
%    ~
%    \begin{subfigure}[t]{.3\textwidth}
%        \centering
%        \def\svgscale{.6}
%        \input{images/dialogue-example-prop1.pdf_tex}
%        \caption{Configuration after proposal $a_2 > a_1$.}
%        \label{fig:example-prop1}
%    \end{subfigure}
%    ~
%    \begin{subfigure}[t]{.3\textwidth}
%        \centering
%        \def\svgscale{.6}
%        \input{images/dialogue-example-solution.pdf_tex}
%        \caption{Configuration after proposal $a_1 > a_2$.}
%        \label{fig:example-solution}
%    \end{subfigure}
%    \caption{Three stages of resolving a conflict. Agents are circles inscribed
%    by $a_i$, their respective goals are $g_i$. Paths that are followed are
%    indicated by the arrows. The red dot represent where two paths have
%    conflicting moves.}
%    \label{fig:example}
%\end{figure*}
%
%\paragraph{Example of conflict resolution} Consider the cooperative pathfinding
%problem in \autoref{fig:example-initial}. It shows a $4 \times 4$ grid with
%three agents in it in their starting positions. The initial optimal paths they
%found during the first step of the algorithm are shown as arrows.
%Agent $a_1$ has a path that consists of three \emph{south} moves, $p_{a_{1},1}
%= \{\text{\emph{south, south, south}}\}$, while both $a_2$ and $a_3$ have paths
%that consist of three consecutive \emph{east} moves, $p_{a_2,1} = p_{a_3,1} =
%\{\text{\emph{east, east, east}}\}$. None of the agents has a \emph{wait}
%action in their path. After finding the initial paths all three agents share
%their paths with each other, agents $a_1$ and $a_2$ discover that they collide
%after their first action. This means that they will have to resolve this
%conflict to prevent the collision.
%
%Before discussing the details of how the agents resolve the conflict in this
%situation some definitions are needed. A proposal where agent $a_i$ has
%priority over a agent $a_j$ is represented as $a_i > a_j$. Agents can cast
%votes on proposals, $\vote_{a_n}(a_i > a_j)$ is the vote of $a_n$ on the
%proposal $a_i > a_j$. To be able to cast votes agents need to evaluate a
%proposal. An evaluation  is based on two effects which can be weighted
%differently. The first effect is the difference in path length before and after
%a proposal has been adopted. The second effect is the change in the number of
%conflicts that an agent is involved in, this effect is weighted three times
%heavier than the former effect. Qualitatively this means that $\vote_{a_n}(a_i
%> a_j) = 1 \cdot \Delta l + 3 \cdot \Delta c$ where $\Delta l$ is the
%difference in path lengths and $\Delta c$ is the difference in conflicts.
%Agents only communicate their final vote and not how they arrived at it.
%
%Continuing with the example, after the agents have determined that they have a
%conflict they start a new dialogue. The dialogue starts in its initial stage,
%the opening. All messages are broadcast to all agents in the conflict dialogue,
%in this case $a_1$ and $a_2$:
%\\ \-\qquad $a_1$: no earlier conflicts
%\\ \-\qquad $a_2$: no earlier conflicts
%
%Because both agents don't have any conflicts that occur at an earlier time
%(this is the first time step) the dialogue can move on to the proposal stage.
%Both agents make a proposal in which they go first:
%\\ \-\qquad $a_2$: propose $a_2 > a_1$
%\\ \-\qquad $a_1$: propose $a_1 > a_2$
%
%Both agents have made a proposal to resolve the conflict. The dialogue can
%move to the evaluation stage. Agent $a_2$'s proposal will be evaluated first
%because $a_2$ submitted the proposal before $a_1$. In this proposal agent $a_2$
%has the highest priority so it doesn't need to update its plan and its path
%remains $p_{a_2,1}$. It updates $a_1$ of the fact that its path hasn't changed
%and still consists of three consecutive \emph{east} actions:
%\\ \-\qquad $a_2$: new path $p_{a_2,1} = \{\text{\emph{east, east, east}}\}$
%
%Agent $a_1$ does have to yield to $a_2$ so it will have to consider possible
%conflicts that arise with $p_{a_2,1}$ and plan around them. It finds a new path
%$p_{a_1,2} = \{\text{\emph{south east, south, south west}}\}$ which is shown in
%\autoref{fig:example-prop1}. After finding the new path $a_1$ will evaluate its
%quality so that it can send its vote to $a_2$. Before adopting the proposal the
%agent had one conflict, this has not changed after making a new plan so $\Delta
%c = 0$. There is no difference in the length of the paths before and after
%temporarily adopting the constraints of the proposal, so $\Delta l = $. This
%means that agent $a_1$ can send the vote to $a_2$. It will also send its new
%path along with the proposal
%\\ \-\qquad $a_1$: new path $p_{a_1,2} = \{\text{\emph{south east, south,
%south west}}\}$
%\\ \-\qquad $a_1$: $\vote_{a_1}(a_2 > a_1) = 0$
%
%Now that $a_2$ knows the new path of agent $a_1$ after adopting to $a_2 > a_1$
%it can also vote on the proposal. This happens in a similar vain as $a_1$'s
%evaluation, there is however one fewer conflict for $a_1$ as its path doesn't
%conflict with that of $a_2$ any more. Achieving this was the goal of the
%dialogue. The path remained the same so $\Delta l = 0$ and $\Delta c = -1$, the
%vote of $a_2$ is then $\vote_{a_2} = 1 \cdot \Delta l + 3 \cdot \Delta c = -3$.
%This vote is then cast:
%\\ \-\qquad $a_2$: $\vote_{a_2}(a_2 > a_1) = -3$
%
%Now that this proposal has been evaluated by both agents they can find the sum
%score of the proposal which is $\eval(a_2 > a_1) = \eval(a_1, a_2 > a_1) +
%\eval(a_2, a_2 > a_1) =
%-3$. Next the agents can evaluate $a_1 > a_2$. When agent $a_1$ goes to
%evaluate this conflict if finds that it can use the path $p_{a_1,1}$ which is
%stored in its cache. It can use this because it has a priority scheme similar
%to the one being evaluated; $a_1$ does not have to yield to any agent. It
%notifies $a_2$ of this:
%\\ \-\quad $a_1$: new path $p_{a_1,1} = \{\text{\emph{south, south, south}}\}$
%
%Next $a_2$ can evaluate the proposal. First it needs to plan a new path that
%does not conflict with the path that $a_1$ has just send. It finds the path
%$p_{a_2,2} = \{\text{\emph{north easth, east, south east}}\}$. The new
%situation is shown in \autoref{fig:example-solution}. It can
%immediately evaluate it, the lengths of $p_{a_2,1}$ and $p_{a_2,2}$ are equal
%and $a_1$ has one fewer conflicts so $\Delta c = -1$.
%\\ \-\quad $a_2$: new path $p_{a_2,2} = \{\text{\emph{north easth, east, south
%east}}\}$
%\\ \-\quad $a_2$: $\eval(a_2, a_1 > a_2) = 1 \cdot (|p_{a_2,2}| - |p_{a_2,1}|)
%+ 3 \cdot \Delta c = -3$
%
%Agent $a_1$ can also send its evaluation to $a_2$:
%\\ \-\quad $a_1$: $\eval(a_1, a_1 > a_2) = 1 \cdot (|p_{a_1,1}| - |p_{a_1,1}|)
%+ 3 \cdot \Delta c = -3$
%
%Now both agents have evaluated the proposal $a_1 > a_2$ they can find the sum
%of the evaluations which is $\eval(a_1 > a_2) = \eval(a_1, a_1 > a_2) +
%\eval(a_2, a_1 > a_2) =
%-6$. All proposals have been evaluated so the agents can notify each other if
%they want to make more proposals:
%\\ \- \quad $a_1$: no more proposals
%\\ \- \quad $a_2$: no more proposals
%
%Neither agent wants to make more priority ordering proposals, this is because
%there are no more possible priority orderings to make with these two agents.
%The dialogue can then move to the closing stage. In the closing stage agents
%will pick the best proposal, in this case that is the proposal with the lowest
%sum score which is the proposal $a_1 > a_2$. To adapt this proposal agents will
%need to use the respective paths that they used during the evaluation of the
%proposal. So $a_1$'s path will be $p_{a_1,1}$ while $a_2$'s path will be
%$p_{a_2,2}$. They also need to keep track of which agents have a higher
%priority than themselves. For agent $a_1$ nothing changes, while $a_2$ must now
%store that $a_1$ has a higher priority than it. All conflicts in
%\autoref{fig:example} have now been solved and agents are free to execute their
%paths.
%
%To evaluate a proposal an agent needs to weigh different factors of the
%quality. In this example a weight of 1 for the path length and a weight of 3
%for the number of solved/introduced conflicts was used. These weights are used
%for demonstration only and an implementation should have these weights set
%empirically. The weights can be any real number. When dialogues between groups
%of three or more agents are possible then a third penalty weight can be added.
%This penalty weight is included in the evaluation if there are two or more
%agents that still have a conflict with each other after adapting a priority
%proposal. It is a penalty for when a proposal only partially solves a conflict.
