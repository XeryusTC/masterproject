\section{Related Work}\label{sec:related}
The following sections discuss previous work into cooperative pathfinding,
argumentation and coordination. The cooperative pathfinding problem requires
that agents are able to coordinate their movement. Several different approaches
that achieve this will be discussed below. Computational argumentation has been
used in various domains. One of this is the construction of a plan for an agent,
this application of argumentation is known as \emph{practical reasoning}. It 
can also be used to make plans in a multi-agent system which allows the agents 
in such a system to coordinate. Work in multi-agent coordination is also 
discussed to help bridge the gap between argumentation and cooperative 
pathfinding.

\begin{table*}[h]
    \centering
    \caption{Comparison of several cooperative pathfinding algorithms.}
    \label{tbl:planning-overview}
    \begin{tabular}{l|l|l|l|l|l}
        & Category & Complete & Priority & Comm. & Online \\
        \hline
        OD+ID \citep{standley2010,standley2011} & Centralized & Yes & No & All &
        No \\
        ICTS \citep{sharon2013} & Centralized & Yes & No & All & No \\
        ADPP \citep{cap2012} & Decoupled & No & Yes & All & No \\
        WHCA* \citep{silver2005} & Decoupled & No & Yes & Window
        & Yes \\
		DMRCP \citep{wei2016} & Decentralized & No & No & 2 nodes & Yes \\
        DiMPP \citep{chouhan2017} & Decentralized & Yes & Yes & Ring & No \\
        %Proposed & Decentralized & No & Partial & All & Before
    \end{tabular}
\end{table*}

\subsection{Cooperative pathfinding}
In the grid world of \autoref{fig:world} each agent can take one of $b+1$
actions, where $b$ is the current number of neighbouring cells without
obstacles. There is also a \textit{wait} action where an agent does not move. 
In our formulation the neighbouring cells include the cells that can be reached
by moving diagonally. The naive approach to finding conflict free paths takes
the Cartesian product the state spaces of all $k$ agents and searches the new
combined state space with an algorithm like A*. This is also known as the
Standard Algorithm \citep{standley2010}. This results in a branching factor of
$(b+1)^k$. This exponential growth in branching factor grows when the number of 
agents increases quickly becomes intractable even with efficient search 
algorithms like A* \citep{sharon2013}.

An overview of several cooperative pathfinding algorithms is shown in 
\autoref{tbl:planning-overview}. A short overview of each algorithm follows 
below. The completeness of an algorithm indicates whether it is able to find a 
solution if one exists. The priority indicates whether an algorithm enforces a 
hierarchy on the agents. Some algorithms limit the communication range of 
agents, this is shown in the Comm. column. An online algorithm interleaves 
planning and execution.

There are a few common strategies that are used to tackle the intractability of 
the cooperative pathfinding problem. Centralised methods use one single 
processor to calculate the paths for all agents. They are often complete 
meaning that they 
will find a solution to the problem if one exists. This also means that they 
are slow. An alternative strategy is to decouple the agents from each other. 
Each agent plans its own path after a hierarchy is enforced on the agents. 
Agents with a lower priority need to give way to agents with a higher priority. 
Decoupled methods sacrifice completeness for speed. They often calculate the 
priorities at a central processor. There are also decentralized methods that 
will only solve conflicts when they occur during execution. These methods often 
lack a global overview so they cannot avoid deadlocks and congestions.

One centralized method called Operator Decomposition (OD) 
\citep{standley2010,standley2011} the problem more tractable by assigning 
actions to agents individually. The branching factor of the problem then 
becomes $(b+1)$ instead of $(b+1)^k$ but the depth of the solution grows with a 
linear factor $k$. OD is a complete and optimal algorithm, meaning that it 
will always find a solution if one exists and it will find the best solution. 
OD can be sped up by splitting the problem into independent sub-problems, this 
is called Independence Detection (ID) \citep{standley2010}. The run time of 
OD+ID is dominated by the largest subgroup. The algorithm is still complete and 
optimal. Several variants on ID+OD have been proposed, leading to the Optimal 
Anytime algorithm \citep{standley2011} which will quickly find a solution and 
can then spend more time on improving the solution.

Another centralized method is called the Increasing Cost Tree Search (ICTS)
which is a two-fold search method \citep{sharon2013}. It searches combinations 
of possible paths of agents to find a solution. This is done in such a way that 
it will always finds the solution with the lowest cost. It is possible to use 
ID with ICTS as well. The ICTS is a complete algorithm like OD+ID. It is faster 
than OD+ID in situations when the number of agents relative to the number of 
nodes is high.

The above algorithms both fall into the centralized category of algorithms.
These methods can become very slow because of the state-space explosion.
Decoupled methods reduce the required calculation time by considering each
agent separately. They generally use the same three step approach
\begin{enumerate}
    \item Find optimal paths for each agent independently of each other.
    \item Assign all agents a unique priority.
    \item Make new plans for all the agents, but consider agents with a high 
    priority as a moving obstacle. Agents with a lower priority can safely be 
    ignored.
\end{enumerate}
Finding the optimal priority order is a combinatorial problem 
\citep{bennewitz2002}. A dependence graph can be used to find possible priority 
schemes. The total length of the final solution depends highly on the priority 
ordering employed. Some of the possible priority schemes may not even lead to a 
solution. This category of algorithms is not complete because it may be the 
case that none of the possible priority schemes lead to a solution while a 
solution to the problem does exist.

Determining the prioritization scheme is often centralized since a
single processor needs to determine all dependencies \citep{bennewitz2002}. One
method called Asynchronous Decentralized Prioritized Planning (ADPP)
\citep{cap2012} exploits the inherent parallelism of a multi-robot team during
the planning stages. Agents make their own plan and communicate it with lower 
priority agents such that they update their plans to avoid conflicts. There is 
no need for agents to wait for each other to finish their plans. Agents are 
only required to update their plan if they conflict with that of a higher 
priority agent.

Windowed Hierarchical Cooperative A* (WHCA*) is a decoupled algorithm that has
been very successful in the video-game industry \citep{silver2005}. 
It uses a reservation table in which agents can denote where they plan to be at 
each time step. The agents use this reservation table for a window of $w$ time 
steps, after this limit they ignore reservations of other agents. After 
executing $w/2$ time steps they will make a new plan and coordinate with other 
agents in their current window. The priority between agents varies with the 
window. Other decoupled methods suffer from agents becoming uncooperative when 
they reach their goal. The use of a window by WHCA* ensures that agents will 
continue to move out of the way even if they have reached their goal.

One model of completely decentralized cooperative pathfinding called DMRCP has
been proposed by \citep{wei2016}. Agents move towards their destination and only
communicate with other agents that are at most two grid cells away. Agents use 
various strategies to deal with different conflict situations. It requires 
slightly less computation time than OD+ID and on average the agents need fewer 
movement steps to reach their goal positions. Because of the limited 
communication range there is an increased risk of deadlocks which requires 
additional strategies to resolve them. Although completeness is not discussed 
the algorithm is based on decoupled methods which are generally not complete.

Another method that doesn't use a central processor is Distributed Multi-agent
Path Planning (DiMPP) \citep{chouhan2017}. This is a distributed algorithm of 
which the authors claim that is complete. Instead of evaluating $k!$ possible 
priority schemes the algorithm only searches $k$ schemes. Each one has a 
different agent with the highest priority, but the agents always occur in the 
same order. All agents start planning with the priority order in which they 
have the highest priority and pass it on to the next agent. If an agent cannot 
find a solution then it will drop the plan and not pass it on. Once all agents 
have a path in a plan it is executed. Proof for the completeness of DiMPP is 
given in \cite[subsection 5.1]{chouhan2017}.

\subsection{Argumentation}
Cooperative pathfinding can be seen as an instance of a resource sharing
problem. From this perspective a conflict occurs when two agents try to access
the same resource at the same time. One way of dealing with this resource
sharing dispute is by constructing an argument with the goal of determining
which agent gets to access the resource at what time. Argumentation has long
been studied by philosophers, and in recent decades it has also been
extensively researched in the field of Artificial Intelligence as well. One 
pillar is non-monotonic logic. A logic is non-monotonic when a conclusion that 
follows based on the premises does not necessarily hold any more when 
additional premises are added \citep{vaneemeren2014,modgil2013}.

\citep{pollock1995} distinguishes two different types of defeating arguments. 
Other researchers have formulated additional forms of defeaters, but they can 
be distilled into three main forms \citep{vaneemeren2014}:
\begin{description}
	\item[Undermining defeaters] attack the premises or assumptions of an
	argument.
	\item[Undercutting defeaters] attack the connection between a set of
	reasons and the conclusion in an argument.
	\item[Rebutting defeaters] raise an argument in favour of an opposite
	conclusion, thereby attacking an argument.
\end{description}

OSCAR is an agent that argues with itself to create a plan was introduced by 
\cite{pollock1995}. It uses defeasible reasoning and epistemic to create 
courses of actions. It uses intentions to achieve its goals. The plans that it 
makes can be hierarchical; planning is deferred until the time when a more 
specific course of actions is needed or when more information is available to 
the agent. The agent may also discard its current plan in case that determines 
that it is not feasible any more.

\subsubsection{Dialogues}
Multiple agents can have an argument through a dialogue. Walton and
Krabbe \citep{walton1995} proposed a typology of the main dialogues that humans
partake in. They distinguish six main types of dialogues. Dialogues are often 
analysed in a game-theoretic sense. The utterances that agents can make are 
analogous to the moves in a game. Which utterances are appropriate at each 
moment is then defined by the rules of the game. There are also rules about how 
to start and end a dialogue, and when agents can speak 
\citep{prakken2006,prakken2009,mcburney2009}. 

One model of deliberation dialogues is presented in \citep{mcburney2007}, it is
also known as the MHP model. It
consists of eight stages. It starts with an \textbf{Open} stage and ends with
a \textbf{Close} stage. The other stages can occur multiple times during a
dialogue as long as they occur following the rules of the dialogue game.
During the dialogue agents will collect the preferences, goals and other
constraints that need to be considered. Agents will then propose common courses
of action. The dialogue requires unanimity before the plan is adopted but it 
allows for any voting mechanism to pick the most preferred plan among many. By 
gathering the requirements of all agents during the dialogue their local views 
combine into a single global view that can be used to create a plan. One 
variant called \textsc{TeamLog} \citep{dunin-keplicz2011} requires fewer 
stages. 
Besides the opening and closing stages there are only a proposal and evaluation 
stage. During these stages agents can still put arguments for or against 
proposals forward. The \textsc{TeamLog} model has the same expressiveness as 
the MHP model.

There are also some problems with the MHP model when modelling
deliberation dialogues. The model does not have an easy method of integrating
additional information into the deliberative process. It also doesn't have a
method of dealing with failures to find a course of action. The dialogue can 
only end when the agents have reached an agreement on a course of action. If 
they cannot find a satisfactory solution then the dialogue cannot end. These 
two shortcomings are raised and addressed by \citep{walton2014}. The problem of 
integrating additional information into the dialogue is addressed by adding a 
knowledge base that is specific to the dialogue. The extended model lists ten
criteria for when a dialogue can be closed.

\subsection{Multi-agent coordination}
Argumentation can be used to allow coordination between agents by letting them
deliberate in a dialogue. Cooperative pathfinding is a particular instance of a
coordination problem. \cite{pollock1995} has constructed an agent which is able 
to discard and create new plans. This can also be done in a multi-agent system 
through Partial Global Planning (PGP) 
\citep[pp.~202--204]{durfee1991,decker1992,woodridge2009}. The agents 
coordinate their actions without any one of them formulating a global plan. 
Instead agents only coordinate with each other when they are required to do so. 
Agents adapt their plans to each other in these situations. Agents are free to 
adapt their plan without having to re-coordinate with each other. This means 
that agents make small incremental changes to their own plan which leads to a 
set of globally well coordinated plans between all agents. During this process 
each agent only knows the parts of the global plan that it is affected by. 
Generalized PGP \citep{decker1992} extends this with real-time planning, 
negotiation, and coordination relationships between goals. This allows the 
framework to be used in settings other than the multi-sensor network that PGP 
was originally developed for.