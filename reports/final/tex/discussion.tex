\section{Discussion}\label{sec:discussion}
This section will first discuss the experimental results. The implications of 
our algorithm and its relation to the literature presented in 
\autoref{sec:related} will be discussed at the end of this section.
% TODO: scalability
% TODO: more compartmentalized than decoupled methods
% TODO: discuss WHCA* vs Continual Planning vs WDPCA*
% TODO: talk about agents changing destinations and how WDCPA* helps.
% TODO: extracting reasons for why solution was settled on
% TODO: going back to previous conflicts
% TODO: WDPCA* sacrifices global plan for speed.
% TODO: $w=8$ is early faster because of less replanning

% DPCA*+ and PCA are simular, later is simpler
%With DPCA* agents resolve conflicts in pairs only while DPCA*+ can have larger
%groups of agents resolve a conflict. This only happens when multiple agents try
%to make conflicting moves at the same time. Allowing larger groups in a
%dialogue means that it gets more complex, this is reflected in the fact that
%PPCPF+ is slower overall and on individual problem instances. Larger groups of
%agents solving conflicts is not necessarily faster than pairs of agents solving
%conflicts. Several small pairwise dialogues can be very effective because the
%pair $a_1$ and $a_2$ resolving their conflict means that $a_2$ gets rerouted,
%this means that $a_2$'s conflict with $a_3$ is also resolved. This means that
%$a_1$ and $a_3$ only need to come to a solution. In the case that all three
%agents would find a solution in a single dialogue they will have to evaluate
%multiple proposals. The added complexity of three agents trying to find a
%solution to the conflict does not outweigh the speed of starting several
%dialogues which come to a conclusion quickly.

\subsection{Results}

We have seen that only PCA* takes more time to solve problem instances and
solves fewer instances than OD+ID and DiMPP. All other versions of the proposed
algorithm can solve more
instances and are generally faster than the two reference algorithms. This
comes with the caveat that DPCA* and especially WDPCA* find solutions for which 
the cost is higher than those found with the algorithms from the literature. It 
is expected that the proposed algorithms are faster than OD+ID
since the latter is a centralized algorithm while the proposed algorithms are
all decentralized. Centralized methods are generally slower than decentralized
algorithms because they consider the combined state spaces
of all agents. Decentralized methods only use the state space of a single agent
to create a plan for that agent. This results in a reduction in the complexity
of the problem \cite{bennewitz2002,sharon2013}.

% TODO: Move this to results?
Although centralized methods are slower than decentralized methods they find a 
solution in which each agent's path is optimal. They are
also complete; given enough time they will find a solution to the problem if 
one exists. The experimental setup limits the amount of time that an algorithm 
can use to find a solution which results in complete methods like OD+ID not 
being able to find a solution.
Instead the experiments are focused on finding a solution in as
little time as possible. We consider being able to find a solution in
a reasonable time to be more important than being able to find an optimal 
solution when
given hours, days, or even weeks to calculate a solution. In a real-time 
applications
like video games or groups of autonomous robots it is more important to find any
solution in reasonable time than it is to find an optimal solution
\cite{botea2013}.
Because WDPCA* is fully decentralized it is able to find a solution in a
reasonable amount of time while sacrificing the quality of the solution. This
means that WDPCA* is more suitable for real-time applications because it is
able to resolve a problem instance in a minimal amount of time. It is also able 
to
find a solution to problems with a large amount of agents because of this speed.

%window version being better
WDPCA* is able to solve more problem instances than DPCA* (which does not use a 
window), see \autoref{tbl:quality} and \autoref{fig:solved}. How many more 
instances can be solved depends on the
size of the window. As a trade-off the amount of actions required for agents to
reach their destination increases. This effect get stronger when the window
size gets smaller (see \autoref{fig:lengths}). When $w=2$ most of the instances 
were solved, however the
sum of the path lengths was also the highest. The required time of WDPCA*-2 
being so low is not surprising. With a small window the algorithm is reduced to 
be
reactive. There is barely any global planning any more because agents don't
look far ahead when trying to find their way to the destination. Agents mainly
solve conflicts that will happen during their next action or the time-step 
directly after that. Agents will also determine whether they have
conflicts after each time-step. This shows that it is a valid strategy to create
many small plans with low computational effort instead of a complete global
plan. Cooperating in too small a window does have a negative effect on the
quality of the solution.

In several figures there is a clear initial trend which trails off when the
number of agents in the problem instance becomes larger. The number of
dialogues per number of agents in \autoref{fig:dialogues} is an example of
this. There is an initial increasing trend which flattens out in the last last
quarter of each plot. The plots for WDPCA* break this trend starting
from between 25 agents and 33 agents per instance. The point where the growth
in the number of dialogues required to solve a problem decreases coincide with
the point in \autoref{fig:solved} where each algorithm has a cliff in the
fraction of instances that have been solved. This suggests that the effect of
the number of dialogues required to find a solution is caused by the algorithm
not being able to find a
solution to complex problem instances within the 2000ms time limit. Complex
problems are those that require many dialogues to find a priority scheme that
allows all agents to find a path to their destination. This in turn suggests
that WDPCA* successfully finding a solution to a problem instance depends on the
number of conflicts in a problem, not the number of agents.

There is an interaction between which algorithm is used and whether previously
found paths were stored in a cache and reused. From \autoref{fig:cache} it
becomes clear that WDPCA* with a small window also means that the effect is
small. When $w$ is small there are fewer conflicts so agents will not have
to participate in dialogues very often. This means that they will not consult
the cache as often and therefore there is less of a speed-boost. On top of that
the paths are also shorter and easier to calculate. Retrieving a short path
which consists of two or four actions from the cache is not much faster than 
calculating the path outright.

\subsection{Related work}

\begin{table*}
    \centering
    \caption{Comparison of several cooperative pathfinding algorithms 
        \cite{standley2010,standley2011,sharon2013,cap2012,silver2005,wei2016,chouhan2017}.
        DPCA* and DPCA*+ are encapsulated by the DPCA* row.}
    \label{tbl:complete-overview}
    \begin{tabular}{l|l|l|l|l|l|l}
        & Category & Complete & Priority & Comm. & Online & Dial. \\
        \hline
        OD+ID & Centralized & Yes & No &
        All & No & No \\
        ICTS & Centralized & Yes & No & All & No & No \\
        IADPP & Decoupled & No & Yes & All & No & No \\
        WHCA* & Decoupled & No & Yes & Window & Yes & No \\
        DMRCP & Decentralized & No & No & 2 nodes & Yes & No \\
        DiMPP & Decentralized & Yes & Yes & Ring & No & No \\
        \hline
        PCA* & Decentralized & No & Yes & All & No & No \\
        DPCA* & Decentralized & No & Yes & All & 
        No & Yes \\
        WDPCA* & Decentralized & No & Yes & Window & Yes & Yes
    \end{tabular}
\end{table*}

A complete comparison of the algorithms discussed and evaluated is given in 
\autoref{tbl:complete-overview}, it combines the information found in 
\autoref{tbl:planning-overview} and \autoref{tbl:proposed}. The category column 
indicates the approach the algorithm takes to solve a problem. The complete 
column indicates whether an algorithm is guaranteed to find a solution if one 
exists. The priority column indicates whether agents are assigned a priority 
ordering before calculating a solution. The communication column indicates 
which agents are allowed to communicate with each other. All means that there 
are no limits on communication, window means that agents can communicate with 
each other in a certain range, 2 nodes means that agents communicate in a 2 
graph node radius, ring means that agents communicate in a chain which forms a 
ring. The online column indicates whether planning and execution is 
interleaved. The dialogues column indicates whether agents can influence the 
solution that is found, if the value is `no' then the solution is abstract.


The online algorithms presented in \autoref{tbl:complete-overview} are faster 
than the state-of-the-art OD+ID \cite{standley2010,wei2016}. Two of these, 
WHCA* and WDPCA*, use a window to interleave planning and pathfinding. By 
limiting how far into the future agents cooperate we can speed up the time to 
find a solution. These results have also been found by research into Continual 
Planning \cite{brenner2009}. This makes online algorithms ideal candidates to 
solve the cooperative pathfinding problem. Previous research and our results 
show that this comes at a trade-off: the paths that are found are often not 
optimal and the agents may display unintelligent behaviour like loops in their 
paths. The cause of the lower quality solution is that when $w$ is small the 
algorithm becomes more reactive. Agents will move towards their goal and notice 
that they have a conflict for which the best solution is to backtrack. It may 
also occur that agents will move back to an earlier position in their path 
because they had to move out of the way of another agent. This kind of 
behaviour can be 
prevented by allowing agents to look further ahead so they can coordinate their 
actions earlier to prevent backtracking. This shows that there is a clear 
trade-off between finding a solution in a low amount of time and finding a low 
cost solution with few loops.

Conventional algorithms find an abstract solution for a pathfinding problem 
based on minimal cost. DPCA*, DPCA*+ and WDPCA* add transparency to the 
solution finding process by the dialogues in which agents can put forward 
arguments for or against partial priority orderings. Agents also evaluate and 
vote on each proposal based on several criteria. This gives agents some 
influence over which solution is picked for a problem instance. For an outside 
observer, the agents' arguments and evaluations provide an explanation why a 
group of agents have picked a particular solution. This can be used to explain 
to a user why a certain solution is more preferred than any other possible 
solution.

WDPCA* is based on the A* algorithm \cite{hart1968} but this can be changed to
any pathfinding algorithm. Dialogues result in a priority ordering for agents,
which in turn determines which agents should be considered moving obstacles by
other agents. This is independent from which path planning algorithm is used.
There are only constraints on where an agent can move to. As long as an
algorithm
is able to handle moving obstacles or can be modified to handle moving
obstacles then it can be used instead of A* in WDPCA*. This means that our
algorithm can be adapted to work with other discrete space algorithms, or even
continuous space algorithms like RRT* \cite{lavalle1998,lavalle2001,cap2013}.

% weights dependent on agents
The proposals of priority orderings are evaluated by the agents to find the
best priority ordering. To do this they weigh different effects of the proposal.
Currently these weights are static. The weights that were found using simulated
annealing are a one-size-fits-all solution. The optimal weights that are found
by simulated annealing are dependent on the lower and upper bound of the number
of agents that can be in a problem instance. This suggests that the optimal
value of the weights depends on the number of agents in the problem. Future
work may look into setting weights as a function of the number of agents in the
problem. WDPCA* could use a set of weights based on the number of agents in the
current window. Instead of basing the weights on the number of agents the
number of conflicts that an agent has can determine their value. This may make
the agents respond appropriately to the complexity of a problem instance.
Instead of using a one-size-fits-all solution the agents will adapt their
heuristics to the problem instance.

% Discuss why smaller windows have more dialogues:
%This is to be expected because agents do not start a dialogue for every
%conflict on the most optimal path, but only for those conflicts that occur
%within $w$ time steps. After they solve these conflicts they execute
%$\sfrac{w}{2}$ steps of the plan and move the centre of the window to their
%new
%position. Next they start resolving any conflicts that occur within the
%updated
%window. This means that when agents have a conflict that lies between
%$\sfrac{w}{2}$ and $w$ time steps then they are likely to have a dialogue
%about
%it several times. So when $w$ is small there are more dialogues because agents
%need to coordinate more often.

% Dialogues are pragmatic
% TODO: make this a coherent story
The model of deliberation dialogues as proposed by McBurney, Hitchcock and
Parsons (MHP model) \cite{mcburney2007} is quite complex. Walton et al.
\cite{walton2014} have expanded on this so that the human deliberation
dialogues can be modelled more accurately while addressing some of the
shortcomings of the MHP model. The resulting model is a good explanation of how 
human deliberation dialogues work. The complexity of the model is not a good 
fit for a multi-agent coordination problem like the one presented in this 
thesis. Instead the simpler \textsc{TeamLog} model \cite{dunin-keplicz2011} is 
more appropriate for this setting. The MHP model and the \textsc{TeamLog} model 
both consists of an opening part, a closing part, and an argumentation part 
where all deliberation takes place. The main difference is in the number of 
stages in the argumentation part: the MHP model consists of four stages while 
\textsc{TeamLog} consists of just two stages. These two stages are sufficient 
to be able to solve the conflicts that occur in the cooperative pathfinding 
problem. For WDPCA* we simplified the model even more by allowing agents to 
only make proposals and evaluate them. The discussion that might occur in a 
more complex model is contained within the evaluation stage. This simplicity 
means that WDPCA* is efficient at finding a solution, but it sacrifices 
richness in the dialogue.

Richness in dialogues also depends on the domain. Our problem formulation is
very abstract and generic. The only argument that agents can make against a
priority proposal is that there is no valid path to their destination.
Applications like traffic management can impose additional constraints on the
problem. In air and rail traffic management agents may have to keep to a 
schedule.
Agents can make arguments about keeping to their schedule and weigh the effects 
of possible delays. In multi robot systems the
agents may need to recharge and they should be able to make arguments about why
they need to have a high priority so that they can reach a charging station
without too much delay. Applications may have constraints like these which do
not fit well in an evaluation function, but they are well expressed in a logic.
This also gives an even greater explanatory power than that WDPCA* has with its
evaluation function and its limited argumentative capability.

%\begin{figure}[t]
%    \centering
%    \def\svgscale{.7}
%    \input{images/repair.pdf_tex}
%    \caption{An example of why it is preferred to make an entire new plan than
%    it is to alter an existing plan.}
%    \label{fig:repair}
%\end{figure}
%
%Another cause of the simplicity is that agents will find a completely new plan
%after adapting a priority ordering instead of altering their current plan. This
%is done so that agents will always use the most optimal path. If agents were to
%repair their plans locally they may end up with suboptimal paths. For example
%see \autoref{fig:repair}, agent $a_1$ has to move one step \emph{west} while
%$a_2$ has found a route which consists of two moves \emph{east} followed by one
%step \emph{south west} and \emph{south}. Given the priority scheme $a_1 > a_2$
%it is better for $a_2$ to formulate a new plan instead of solving the conflict
%by plan repair. Formulating a new plan would result in the path $\{
%\emph{south}, \emph{south east}, \emph{east}, \emph{east} \}$. Repairing the
%plan by moving out of $a_1$'s way and following the original plan may result in
%the path \{ \emph{south}, \emph{north east}, \emph{east}, \emph{south east},
%\emph{east} \}. The repaired plan is one step longer than the new plan. It also
%seems less intelligent because the agent moves in the north east direction
%while its goal is to the south and to the east of its position. Plan repair
%will never be able to find a path which is shorter than creating a new plan. It
%is also more likely that an agent exhibits less intelligent behaviour when
%using plan repair.

The argumentative methods \textsf{DeLP-POP} and 
\textsf{DeLP-MAPOP}~\cite{pardo2011,ferrando2012} are 
general approaches to multi-agent coordination that allow agents to discuss 
individual actions in a deliberation dialogue. This makes the method of finding 
a solution analogous to OD+ID. This is in contrast to our approach in which 
agents deliberate on the level of priorities and paths (sequences of actions). 
The benefit of \textsf{DeLP-MAPOP} is that it is distributed, complete and 
optimal. It can find an optimal solution without relying on a central 
processor. However the deliberation process is quite complex and agents will 
often have to keep track of the same information. It is thus not the case that 
agents will only know about what influences their paths but also what 
influences the paths of others, something which is not the case in WDPCA*.

% can be 
%regarded as an argumentative adaptation of OD+ID for general multi-agent 
%planning, whereas our algorithms are confined to the abstract setting of 
%cooperative pathfinding discussed in Section~\ref{sec:problem}. The benefit of 
%\textsf{DeLP-MAPOP} is that they allow

%Deliberation dialogue models are complex. \cite{mcburney2007} has 8 stages,
%most of which aren't used here. Instead \cite{dunin-keplicz2011} is used for
%simplicity. It is verbose enough for this purpose but \cite{mcburney2007} may
%be more appropriate for more complex applications (think planning trains),
%\cite{walton2014} may be appropriate then too. \cite{mcburney2007,walton2014}
%are both meant to model human deliberation dialogues, not necessarily meant for
%calculation/communication time optimal. Agents sharing their paths may be
%considered information sharing step of dialogue.

%The propose and evaluate stages have been compounded. Some approaches only
%allow one proposal to be made and evaluate that, while the next proposal to be
%entered has to wait for the next proposal stage. We allow agents to make
%proposals at the same time and all entered proposals will be evaluated in
%sequence during the same round. Use more complex dialogues (arguments) and
%judgement aggregation to achieve final result instead of voting.

%Problem formulation is generic enough to apply to traffic management etc

Cooperative pathfinding is a specific instance of a coordination problem. It is
possible to generalise the findings here to other resource sharing problems.
Instead of agents making moves in a grid world the agents would claim the use
of a resource for some amount of time. Two agents have conflicting claims when
they try to claim the same resource at the same or overlapping times. They can
resolve this conflict in claims by starting a deliberation dialogue and make
arguments about why an agent should be allowed to access the resource before
the other. They could also make proposals about how to resolve the conflict in
claims and agents should be able to argue for or against its adaptation. A
voting system similar to the priority scheme evaluation used by WDPCA* could
also be employed.

\section{Conclusion}
In this thesis we have combined ideas from partial global planning, continual
planning and computational argumentation and applied them to cooperative 
pathfinding.
A new novel algorithm called WDPCA* was developed starting from the decoupled
model for cooperative pathfinding. Pairs of agents solve conflicts in their
paths by finding determining a priority order. One agent will have to give
priority to the other and consider it to be a moving obstacle. Partial global
planning is used to allow agents to incrementally build their private priority
order. Doing so means that the agents will coordinate their paths so that
there are no conflicts between them. Resolving conflicts is done in a
deliberation dialogue. During such a dialogue agents will discuss possible
priority orders that lead to a solution to the conflict. If there is more than
one possibility they will vote on which priority scheme they prefer. The
resulting priority scheme is adapted by both agents that participate in a
dialogue.

Generally agents will resolve all their conflicts before they start to execute
their plan. This has several limitations: all computation is done before
execution, and finding a solution may require agents to alternate giving and
having priority. The most important limitation is that agents will stop
cooperating once they reach their goal. This may mean that they block the paths
of early agents and do not cooperate with agents that need to pass them. To
address these limitation a window is applied to the search. This limits the
range in which conflict resolution takes place. It also ensures that agents
will continue to cooperate after they have reached their destination by forcing
them to keep planning even though they have already reached their goal.

We have shown that WDPCA* is able to solve problem instances faster than the
well established algorithm OD+ID and the recent algorithm DiMPP. It is
also able to solve problem instances with a large number of agents where both
of the reference algorithms were not able to solve these instances. As a
trade-off the paths found by WDPCA* are often longer and may contain loops
where agents visit the same grid cell multiple times. Employing a large window
results in fewer of these low quality paths while it does not sacrifice much of
the speed gain that a small window has.

The agents solve conflicts by through a deliberation dialogue. This makes it
possible to retrieve arguments why agents have picked a certain priority
ordering. These arguments explain why the solution to the cooperative
pathfinding problem was picked. Because WDPCA* uses arguments it is possible to
integrate domain specific knowledge into the algorithm. The method we proposed
here is less abstract than conventional cooperative pathfinding algorithms
which only calculate a solution without giving arguments why that solution is
the most appropriate.