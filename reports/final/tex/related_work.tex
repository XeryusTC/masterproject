\section{Related Work}\label{sec:related}

\subsection{Cooperative pathfinding}
%Cooperative pathfinding is the field of study that aims to develop methods of
%finding conflict free paths for groups of agents. Given a connected grid and a
%group of $k$ agents with initial positions $s_1, \ldots, s_k$ and goal
%positions $g_1, \ldots, g_k$, a set of paths is conflict free iff all agents
%$a$ have a path from $s_a$ to $g_a$ without any two agents ever occupying
%the same position at the same time, or moving along crossing edges at the same
%time.
In the grid world each agent can take one of $b+1$ actions, where $b$ is the
current number of neighbouring cells, there is also a \textit{wait} action
where an agent does not move. The naive approach to finding conflict free paths
takes the Cartesian product of all $k$ state spaces and searches the new
combined state space with an algorithm like A*, this is also known as the
Standard Algorithm \cite{standley2010}. This results in a branching factor of
$(b+1)^k$, the branching factor grows exponentially in the number of agents and
the problem quickly becomes intractable even with efficient search algorithms
like A* \cite{sharon2013}.

There are a few common strategies that are used to tackle this problem.
%There are a few categories that methods tackling this problem fall into.
Centralised methods use one single processor to calculate the paths for all
agents. They are often complete, they will find a solution to the problem if it
exists. This also means that they are slow. An alternative strategy is to
decouple the agents from each other by letting each
agent plan its own path and enforcing a hierarchy on the agents so that agents
with a lower priority need to give way to agents with a higher priority.
Decoupled methods sacrifice completeness for speed. They often calculate the
priorities at a central processor, but can exploit the inherent parallelism in
multi-agent systems to calculate the paths. There are also reactive methods
that will only solve conflicts when they occur during plan execution.
\autoref{tbl:planning-overview} shows an overview of several algorithms and
summarises their properties, each algorithm is discussed in more detail below.

\begin{table}[t]
    \centering
    \caption{Comparison of cooperative pathfinding algorithms
    \cite{silver2005,standley2010,standley2011,sharon2013,wei2016,cap2012}.
    Communication
        indicates what limit is imposed on which agents can communicate, `all'
        means that there is no limit and agents can communicate with all others.
        Online enumerates whether all movement actions for all agents are
        determined before execution, some algorithms update the plan during
        execution.}
    \label{tbl:planning-overview}
    \begin{tabular}{l|l|l|l|l|l}
        \hline
        Method & Category & Complete & Priority & Communication & Online \\
        \hline
        WHCA* & Decoupled & No & Yes\footnotemark[1] & Window & Yes \\
        OD+ID & Centralised & Yes & No & All & No \\
        ICTS & Centralised & Yes & No & All & No \\
        DMRCP & Decentralized & No & No & 2 node radius &
        Yes \\
        ADPP & Decoupled & No & Yes & All & No \\
        %Proposed & Decentralized & No & Partial & All & Before
    \end{tabular}
\end{table}
\footnotetext[1]{Although WHCA* does give agents a priority, the assigned
    priorities can vary between windows.}

One centralized method called Operator Decomposition (OD) deals with the
intractability of
the problem is by considering the possible moves of each agent on their own
\cite{standley2010,standley2011}. Instead of taking the Cartesian product of
the agents' state spaces it assigns actions to agents individually. This leads
to two different kind of states: in standard states no agent has been assigned
an action; in intermediate states some of the agents have been assigned an
action, when all agents are assigned an action it results in a new standard
state. Because intermediate states are considered individually the algorithm is
less likely to expand those intermediate states that are sub-optimal and thus
fewer states are generated. The result of this is that the branching factor
becomes $(b+1)$, but that the depth of the solution in the search tree grows
with a linear factor $k$. This makes finding a solution with an algorithm like
A* more tractable. OD is a complete and optimal algorithm, meaning that it will
always find a solution if one exists, and it will find the best solution.

On its own OD is not always very efficient, so an additional algorithm called
Independence Detection (ID) was introduced \cite{standley2010}. Before planning
takes place all agents are placed in their own group. Each group then makes a
plan, when the paths for two groups conflict then one group is tasked with
finding a new conflict free path. If this also fails then the groups are merged
and a new plan is formed for the group using OD. This is repeated until a
set of conflict free paths for all agents has been found. Several variants on
ID+OD have been proposed, leading to the Optimal Anytime algorithm
\cite{standley2011} which will quickly find a solution and can then spend more
time on improving the solution. Because ID is an extension that can be applied
to any cooperative pathfinding algorithm OD+ID is still complete. Although
there is implicit priority in which order the agents are assigned actions, it
has no influence on the ability to find a solution or the quality of the
solution. Planning is completed before agents start executing it, and there is
no need to update the plan during execution.

Another centralized method is called the Increasing Cost Tree Search (ICTS)
which is a
two-fold search method \cite{sharon2013}. It consists of a high-level search on
an Increasing Cost Tree (ICT) which has a root node which contains the optimal
path costs for each individual agent. Each child node increases the path cost
for a different agent, so each level in the tree increases the sum of the path
costs by one. This tree is searched using breadth-first search, each node in
the tree will be searched using a low-level search. This low level search
generates all paths for each agents that is equal to the cost in the current
ICT node, it will then try to find a conflict free combination of these paths.
If such a set of paths exists then the algorithm is done, otherwise the
high-level search will continue to the next node in the ICT. Pruning can be
used to decrease the amount of duplicate nodes in the ICT, but it is possible
to use ID as well. The ICTS is a complete algorithm like OD+ID, but it is
faster in situations when the number of agents relative to the number of nodes
is high. ICTS imposes no priority on the agents at all since it only searches
for combination of paths that are a valid solution to the problem.

The above algorithms all fall into the centralized category of solutions, these
can become very slow because of the state-space explosion. Decoupled methods
reduce the required calculation time by considering each agent separately. They
generally use the same three step approach;
\begin{enumerate}
    \item Find optimal paths for each agent independently of each other.
    \item Impose a hierarchy on the agents, often this is done by assigning
    them a unique priority.
    \item Make new plans for all the agents, this time an agent has to consider
    all agents with a higher priority as a moving obstacle. Agents with a lower
    priority can safely be ignored.
\end{enumerate}
This often leads to a set of conflict free plans. Finding the optimal priority
order is a combinatorial problem \cite{bennewitz2002}. A common algorithm of
assigning priorities first calculates a dependence hierarchy based on the paths
found in the first step, then priorities can be assigned such that agents have
a priority that is higher than that of agents that may block them. Circular
dependencies may mean that multiple priority schemes have to be tried. The
quality of the final solution depends highly on the priority ordering employed,
some of the possible priority schemes may not even lead to a solution. This
means that these kind of algorithms are not complete.

Most proposals for decoupled methods don't mention whether a central processor
must make the plans for all agents, or whether the agents can do it themselves.
Determining the prioritization scheme is often centralized, since a
single processor needs to determine all dependencies \cite{bennewitz2002}. One
method called Asynchronous Decentralized prioritized Planning \cite{cap2012}
exploits the inherent parallelism of a multi-robot team during the planning
stages. The method allows
agents to make their individual plans, after an agent has found a path it will
notify all agents with a lower priority of its (new) path. These lower priority
agents will then update their plans if conflicts arise, notifying lower
priority agents. These agents will then update their plans etc. The benefit of
this method is that agents can make a new plan as soon as any one higher
priority agent has send a conflicting plan, there is no need for agents to
wait for each other to finish their plans. This means that agents can plan
simultaneously and that some agents may finish planning before higher priority
agents if their paths are conflict free.

Windowed Hierarchical Cooperative A* (WHCA*) is a decoupled algorithm that has
been very successful in the video-game industry \cite{silver2005}. It uses a
reservation table to denote where agents plan to be and thus prevent other
agents from entering the same space at the same time. It requires that agents
have been assigned a priority ordering in which they plan so that they can take
each other's reservations into account. The amount of computation required
depends on the quality of the heuristic used during A* planning. Hierarchical
A* is an abstraction of the search space used to obtain perfect distance
estimates. The reservation table and time dimension are ignored for this so
that agents can find the minimum distance to their destination. The search by
the above algorithm is windowed so that the reservation table is only used in
the window and the rest of the path is planned in the abstract space,
effectively ignoring the other agent's actions outside of the window. The
window is moved at regular intervals and the agent's plan is updated. Because
of this the agents have no fixed priority, it varies based on the current
window. Computation is spread out over the time it takes for agents to get to
their destination, so there is no need to calculate all paths before execution,
they are instead updated during execution. Another issue that is solved is that
agents would reach their goal position and stay there, potentially blocking the
paths of other agents. The window limits the size of the reservation table the
agents have to take into account, limiting the
communication between agents to the size of the window.

One model of truly decentralized cooperative pathfinding called DMRCP has been
proposed by
\cite{wei2016}. Agents move towards their destination and only communicate with
each other in a two graph node radius. They can give each other commands like
move out of the way, follow me, wait etc. Agents are altruistic which means
that they are willing to make concessions during conflicts even if that means
that they will be at a disadvantage. Agents use various strategies to deal with
different conflict situations. Because of the limited communication range and
the various strategies employed the agents often need to recalculate the
optimal path to their destination during the execution of their old plan. This
approach works well, it requires slightly less computation time than OD+ID and
on average the agents only need two thirds of the number of movement steps to
reach their goal positions. Although completeness is not discussed, the
algorithm is based in decoupled methods which are generally not complete. Some
of the conflict resolving strategies used by the agents are able to solve
situations in which decoupled methods would not find a solution.
Because agents only communicate in a limited range there is no indication
whether agents will have conflicts at a later point in time. This lack of a
global overview means that agents must include strategies to resolve deadlocks
when they occur, there is no way to prevent deadlocks from happening.

\subsection{Argumentation}
\subsubsection{Dialogues}
Multiple agents can have an argument through a dialogue. Walton and
Krabbe \cite{walton1995} proposed a typology of main dialogues that humans
partake in. They distinguish six main types of dialogues, it should be noted
that the list of dialogue types is not exhaustive. In \emph{information
    seeking} dialogues some of the participating agents aim to gather
    information
from another agent that knows the anser. In \emph{inquiry} dialogues a group of
agents collectively seeks an answer to a question to which non of the
participating agents knows the answer on its own. \emph{Deliberation} dialogues
are about what course of action to take in a given situation. A
\emph{persuasion} dialogue occurs when an agent tries to convince on or
multiple other agents of its position. It is successful when the other agent(s)
adopt its position. Participants of \emph{negotiation} dialogues try to find a
division of a scarce resource that all agents can be satisfied with. Finally
\emph{eristic} dialogues are a verbal substitute for fighting. Note that most
actual dialogues combine these dialogue types.

Dialogues are often analysed in a game-theoretic sense, where the utterances
that agents can make are analogous to the moves in a game. Which utterances are
appropriate at each moment is then defined by the rules of the game. Most of
the research into dialogues follows this approach
\cite{prakken2006,prakken2009}. Most dialogue systems have a two language
set-up. The first is the topic language which is about what agents are
discussing and is typically a formal logic, it defines the context of the
dialogue. The second language is the communication language which specifies
which utterances can be made, what effects they have and the rules of outcome.
This latter language is at the core of dialogue games. Most dialogue systems
have the following syntax in common \cite{prakken2006,prakken2009,mcburney2009}.
\begin{description}
    \item[Commencement rules] Rules that concern when and how a dialogue can
    start and what its context is.
    \item[Locution rules] Which utterances are permitted are known as the
    locution rules. They may also define when an utterance is obligatory.
    Common locutions include asserting propositions, questioning or contesting
    assertions and justifying previous assertions after they have been
    questioned.
    \item[Commitments] Some locutions incur commitments on an agent which are
    subsequently put into the agent's commitment store. A dialogue system may
    limit which utterances an agent can make based on what is in its commitment
    store.
    \item[Speaker order] Most dialogue systems specify an order in which agents
    can speak, this can range from agents alternating turns to each agent being
    allowed to make an utterance at any time.
    \item[Outcome rules] These determine what the outcome of the dialogue is.
    Some systems define an outcome while the dialogue may continue and lead to
    a different outcome at a later point.
\end{description}

One model of deliberation dialogues is presented in \cite{mcburney2007}. It
consists of eight stages, starting with an \textbf{Open} stage and ending with
a \textbf{Close} stage. The other stages can occur multiple times during a
dialogue, as long as they occur following the rules of the dialogue game.
During the dialogue agents will collect the
preferences, goals and other constraints that need to be considered. Agents
will then propose common plans of action, when multiple plans have been
proposed agents can specify which they prefer. At some points an agent can
recommend a plan after which all agents will vote for that plan. The dialogue
requires unanimity before the recommended plan is adopted, but it allows for a
voting mechanism to pick the most preferred plan among many. There are variants
of this model that require fewer stages while still allowing for the same
expressiveness \cite{dunin-keplicz2011}.
