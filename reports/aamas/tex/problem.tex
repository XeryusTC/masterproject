\section{Problem formulation}\label{sec:problem}

We define the problem of cooperative pathfinding as follows. A shared
space is divided into discrete cells such that it forms an 8-connected grid.
Some of these grid cells contain static obstacles while the other cells are
open. A set of $k$ agents $\{a_1, \ldots, a_k\}$ occupy cells within the grid,
the agents have respective goal positions $g_1, \ldots, g_k$. A set of paths
need to be found, one for each agent, such that each agent gets to its goal
position without colliding with any of the other agents. A path consists of a
series of actions. An action can either be to move to one of the eight
neighbouring cells or \emph{wait} at the current location. Each time step an 
agent must do exactly one of these actions.
%Each action has unit cost, with the exception of waiting in the goal position 
%which has zero cost. The cost function is then
A cost function assigns unit cost to all actions, except from waiting in the 
goal position which has zero cost:
\[
\text{\textsc{cost}(P,Q)} =
\begin{cases}
0 & \text{if } P = Q = G \\
1 & \text{otherwise}
\end{cases}
\]
where $P$ is the node where the agent's location node, $Q$ the node where the
agent moves to, and $G$ the agent's goal node. A single agent's path has a
cost that is the sum of the costs of all its actions. The cost of a solution is
defined as the sum of the costs of the paths of the agents. An optimal solution 
has minimal cost.

The paths of two agents are not in conflict iff at no time step the agents
occupy the same cell, the agents move along the same edge (swap positions), or
the agents move along crossing edges. Obstacle cells can be considered as
stationary agents. Examples of each of these conflicts are given in
\autoref{fig:conflicts}. A single action can result in an agent having multiple
conflicts at the same time. If \autoref{fig:conflict-position} had an agent
$a_3$ in the top right cell moving to the bottom middle cell then $a_2$ would
have a conflict with both $a_1$ and $a_3$ at the same time. Agents are allowed
to move along a diagonal even when the two cells on the opposing diagonal are
blocked, i.e. $a_5$ in \autoref{fig:world} can move to its destination in a
single time step. An agent $a_i$ can move to a cell occupied by agent $a_j$
given that $a_j$ will move to a different cell at the same time. Agents $a_1$,
$a_2$, $a_3$ and $a_4$ in \autoref{fig:world} can reach their respective
destinations in a single time step by ``rotating'' clockwise. Agents $a_7$ and
$a_8$ cannot move to their destinations in a single time step because that
would mean that they move along crossing edges at the same time.

\begin{figure}[t]
    \centering
    \begin{subfigure}[b]{.25\textwidth}
        \centering
        \def\svgscale{.55}
        \input{images/conflict1.pdf_tex}
        \caption{Moving to the same position.}
        \label{fig:conflict-position}
    \end{subfigure}
    ~
    \begin{subfigure}[b]{.25\textwidth}
        \centering
        \def\svgscale{.55}
        \input{images/conflict2.pdf_tex}
        \caption{Moving along the same edge.}
        \label{fig:conflict-same}
    \end{subfigure}
    ~
    \begin{subfigure}[b]{.25\textwidth}
        \centering
        \def\svgscale{.55}
        \input{images/conflict3.pdf_tex}
        \caption{Moving on crossing edges.}
        \label{fig:conflict-crossing}
    \end{subfigure}
    \caption{Examples of conflicting actions. Agents are circles inscribed with
        $a_i$. Their movements are indicated by the arrows starting in the cell
        they occupy, the action ends in the cell that the arrow points to.}
    \label{fig:conflicts}
\end{figure}