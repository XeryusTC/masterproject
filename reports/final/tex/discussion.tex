\section{Discussion}\label{sec:discussion}
% TODO: scalability
% TODO: more compartmentalized than decoupled methods

% DPCA*+ and PCA are simular, later is simpler
%With DPCA* agents resolve conflicts in pairs only while DPCA*+ can have larger
%groups of agents resolve a conflict. This only happens when multiple agents try
%to make conflicting moves at the same time. Allowing larger groups in a
%dialogue means that it gets more complex, this is reflected in the fact that
%PPCPF+ is slower overall and on individual problem instances. Larger groups of
%agents solving conflicts is not necessarily faster than pairs of agents solving
%conflicts. Several small pairwise dialogues can be very effective because the
%pair $a_1$ and $a_2$ resolving their conflict means that $a_2$ gets rerouted,
%this means that $a_2$'s conflict with $a_3$ is also resolved. This means that
%$a_1$ and $a_3$ only need to come to a solution. In the case that all three
%agents would find a solution in a single dialogue they will have to evaluate
%multiple proposals. The added complexity of three agents trying to find a
%solution to the conflict does not outweigh the speed of starting several
%dialogues which come to a conclusion quickly.

We have seen that only PCA* is slower and solves fewer problem instances than
OD+ID and DiMPP. All other versions of the proposed algorithm can solve more
instances and are generally faster than the two reference algorithms. This
comes with the caveat that DPCA* and especially WDPCA* find solutions in which
the paths are often longer than those found with the state-of-the-art
algorithms. It is expected that the proposed algorithms are faster than OD+ID
since the latter is a centralized algorithm while the proposed algorithms are
all decentralized. Centralized methods are generally slower than decentralized
algorithms because they consider a state space which combines the state space
of all agents. Decentralized methods only use the state space of a single agent
to create a plan for that agent.

Centralized methods are slower than decentralized methods however they do often
find a solution in which each agent's path is as short as possible. They are
also able to find a solution to the problem if one exists. To do so they often
need sufficient computation time which is not what the experimental setup
allows for. Instead the experiments are focused on finding a solution in as
little time as possible. We consider being able to find a solution in
a reasonable time to be more important than being able to find a solution when
given hours or even days to calculate a solution. In a real-time applications
like video games or groups of autonomous robots it is more important to find a
solution in reasonable time than it is to find the most optimal solution.
Because WDPCA* is fully decentralized it is able to find a solution in a
reasonable amount of time while sacrificing the quality of the solution. This
means that WDPCA* is more suitable for real-time applications because it is
able to resolve a problem instance in a low amount of time. It is also able to
find a solution to problems with a large amount of agents because of this speed.

%window version being better
WDPCA* is able to solve more problem instances than DPCA* without a window. How
many more instances can be solved depends on the size of the window. As a
trade-off the amount of actions required for agents to reach their destination
increases. This effect get stronger when the window size gets smaller, when
$w=2$ most of the instances were solved, however the sum of the path lengths
was also the highest. That the speed of WDPCA*-2 is high is not surprising, it
reduces the algorithm to be almost reactive. There is barely any global
planning any more because agents don't look far ahead when trying to find their
way to the destination. Agents mainly solve conflicts that will
happen during the next time step, and some conflicts that happen one time step
later. Agents will also determine whether they have conflicts after each action.
This shows that it is a valid strategy to create many small plans with
low computational effort instead of a complete global plan, but cooperating in
too small a window has a negative effect on the quality of the solution.

Part of the lower quality of a solution when $w$ is small is because the
algorithm becomes more reactive. Agents will move towards their goal and notice
that they have a conflict with another agent. It may be that the best solution
for an agent is to backtrack and move to the position it just came from.
Sometimes agents also move out of the way of another agent and move to a grid
cell that it has visited earlier on its way to its goal. It may have been
possible to avoid these loops by increasing the size of the window. Agents can
then look further ahead and coordinate their plans earlier preventing the need
for backtracking. This shows that there is a clear trade-off between finding a
solution in a low amount of time and finding a good solution with few loops. In
some applications it may be undesirable to have loops because an observer might
see it as unintelligent behaviour.

% Discuss why loops occur:
%When
%$w=2$ then agents may make one or two moves to only discover that they now have
%a conflict. Sometimes the solution to a conflict is to backtrack and agents
%will revisit nodes that they have been to earlier

There is an interaction between which algorithm is used and whether previously
found paths were stored in a cache and reused. From \autoref{fig:cache} it
becomes clear that a smaller window also means that the effect is smaller. With
a small window there are fewer conflicts, so agents will not have to
participate in conflicts very often. This means that they will not consult the
cache as often and therefore there is less of a speed-boost. On top of that the
paths are also shorter and easier to calculate, so retrieving a path from the
cache is not faster than calculating the path outright.

DPCA* is based on the A* algorithm \cite{hart1968} but this can be changed to
any pathfinding algorithm. Dialogues result in a priority ordering for agents,
which in turn determines which agents should be considered moving obstacles by
other agents. This is independent from which path planning algorithm is used,
it only puts constraints on where an agent can move to. As long as an algorithm
is able to handle moving obstacles or can be modified to handle moving
obstacles then it can be used instead of A* in DPCA*. This means that our
algorithm can be adapted to work with other discrete space algorithms, or even
continuous space algorithms like RRT* \cite{lavalle1998,lavalle2001}.

The world defined in \autoref{sec:problem} is an abstracted version of reality
which only allows for very basic arguments and evaluations to be made during a
dialogue. A real world situation like rail traffic management is more complex
and should allow for richer dialogues. In such systems agents should be able to
make domain specific arguments. For example, in a public transport rail system
planner agents could make arguments based on how passenger friendly a proposal
is. A proposal that delays a train from reaching a station in time so that
passengers can transfer to other stations is less passenger friendly than a
proposal which doesn't have this effect.

Cooperative pathfinding is a specific instance of a coordination problem. It is
possible to generalise the findings here to other resource sharing problems.
Instead of agents making moves in a grid world the agents would claim the use
of a resource for some amount of time. Two agents have conflicting claims when
they try to claim the same resource at the same or overlapping times. They can
resolve this conflict in claims by starting a deliberation dialogue and make
arguments about why an agent should be allowed to access the resource before
the other. They could also make proposals about how to resolve the conflict in
claims and agents should be able to argue for or against its adaptation. A
voting system similar to the one used by DPCA* could also be used.

% weights dependent on agents
The proposals of priority orderings are evaluated by the agents to find the
best priority scheme. To do this they weigh different effects of the proposal.
Currently these weights are static. The weights that were found using simulated
annealing are a one-size-fits-all solution. The optimal weights that are found
by simulated annealing are dependent on the lower and upper bound of the number
of agents that can be in a problem instance. This suggests that the optimal
value of the weights depends on the number of agents in the problem. Future
work may look into using weights that depend on the number of agents in the
problem. WDPCA* could use a set of weights based on the number of agents in the
current window. Instead of basing the weights on the number of agents the
number of conflicts that an agent has can determine their value. This may make
the agents respond appropriately to the complexity of a problem instance.
Instead of using a one-size-fits-all solution the agents will adapt their
heuristics to the problem instance.

% Discuss why smaller windows have more dialogues:
%This is to be expected because agents do not start a dialogue for every
%conflict on the most optimal path, but only for those conflicts that occur
%within $w$ time steps. After they solve these conflicts they execute
%$\sfrac{w}{2}$ steps of the plan and move the centre of the window to their
%new
%position. Next they start resolving any conflicts that occur within the
%updated
%window. This means that when agents have a conflict that lies between
%$\sfrac{w}{2}$ and $w$ time steps then they are likely to have a dialogue
%about
%it several times. So when $w$ is small there are more dialogues because agents
%need to coordinate more often.

% Discuss why \autoref{fig:dialogues} trails off towards the end:
%This is likely to be because the algorithms are only able to find a solution
%to problems with few conflicts. More complex problems with many agents would
%require more dialogues but these can't be completed within the 2000ms time
%limit. We can see that the point where the growth in the number of  dialogues
%required to solve an instance starts to increase coincides with
%\autoref{fig:solved}. The start of the steep cliff for each respective
%algorithm in \autoref{fig:solved} is around the same number of agents as where
%the growth in the number of dialogues starts to decrease.
In several figures there is a clear initial trend which trails off when the
number of agents in the problem instance becomes larger. The number of
dialogues per number of agents in \autoref{fig:dialogues} is an example of
this. There is an initial exponential trend which flattens out in the last last
quarter of each plot. The plots for WDPCA* break the exponential trend starting
from between 25 agents and 33 agents per instance. The point where the growth
in the number of dialogues required to solve a problem decreases coincide with
the point in \autoref{fig:solved} where each algorithm has a cliff in the
fraction of instances that have been solved. This suggests that the effect on
the number of dialogues is caused by the algorithm not being able to find a
solution to complex problem instances within the 2000ms time limit. Complex
problems are those that require many dialogues to find a priority scheme that
allows all agents to find a path to their destination. This in turn suggests
that DPCA* successfully finding a solution to a problem instance depends on the
number of dialogues required to find that solution.
%This effect is likely caused by the algorithms only being able to find a
%solution to problem instances with a large number of agents that have a simple
%solution. More complex problems with many agents would require more dialogues
%but this requires more than 2000ms to complete all dialogues.

% Dialogues are pragmatic
% TODO: make this a coherent story
Deliberation dialogue models are complex. \cite{mcburney2007} has 8 stages,
most of which aren't used here. Instead \cite{dunin-keplicz2011} is used for
simplicity. It is verbose enough for this purpose but \cite{mcburney2007} may
be more appropriate for more complex applications (think planning trains),
\cite{walton2014} may be appropriate then too. \cite{mcburney2007,walton2014}
are both meant to model human deliberation dialogues, not necessarily meant for
calculation/communication time optimal. Agents sharing their paths may be
considered information sharing step of dialogue.

The propose and evaluate stages have been compounded. Some approaches only
allow one proposal to be made and evaluate that, while the next proposal to be
entered has to wait for the next proposal stage. We allow agents to make
proposals at the same time and all entered proposals will be evaluated in
sequence during the same round. Use more complex dialogues (arguments) and
judgement aggregation to achieve final result instead of voting.