\section{Introduction}\label{sec:intro}
Path planning for a single agent can be considered a solved problem.
When multiple agents have to find their way through a shared space then the
problem becomes more complicated. The agents have to find paths around
obstacles while they also need to ensure that they do not collide with each
other. Even when agents can prevent collisions then it is still possible that
congestions or even deadlocks may occur. Agents need to be able to get to their
destination as soon as possible so these situations are undesirable. To avoid
them there is a need for coordination between the agents. Cooperative
cooperative pathfinding finds its application in robotics, aviation, road
traffic management, crowd simulations, and video games \cite{standley2011}.

%Discuss cooperative pathfinding and abstract solutions.
The most straightforward approach to the cooperative pathfinding problem
is to search the Cartesian product of the state spaces of all agents. This
approach is computationally inefficient \cite{hopcroft1984,sharon2013} as the 
time to find a
solution grows exponential in the number of agents. Another common approach is 
to
impose a hierarchy on the agents by assigning them a unique priority. Agents
plan a path to their destination in descending order of priority. When it is an
agent's turn to plan their path they have to consider agents with a higher
priority as a moving obstacle. This means that they have to avoid planning any
movements that conflict with those of higher priority agents. Both of these
approaches result in abstract solutions; there is often no clear reason why a
particular solution was the one arrived at. The algorithm has found a set of
conflict free paths and that set was returned as the working solution without
any indication about the considerations of why it is a good plan.

These two common approaches to solving the cooperative pathfinding problem both
rely on a central processor \cite{chouhan2017}. The first is a category of
centralized methods that use a central processor to create a plan for all the
agents. The other category requires that a central processor determines a
priority ordering that the agents have to adhere to. After this has been done
then the calculation of the plans for the agents can be decoupled. This allows
the agents to make their individual plan on their own processor. During this
decoupled planning they need to communicate with each other about their paths
but they do not require a central point of communication. Next to the
centralized and decoupled methods there are also fully decentralized approaches
where agents only communicate in a small range. There is no central processor
that can be a single point of failure. As a trade-off these methods usually
have no global view of the problem. This means that agents can make decisions
early on that will lead to congestions or deadlocks at a later point in time 
without any agent
noticing at the time that the decision was made.

Methods of decentralised coordination have been developed by the field of 
computational argumentation. Formal models of argumentation have been used in 
Artificial Intelligence in expert systems, multi-agent systems and law
\cite{vaneemeren2014,rahwan2009}. An important concept in computational 
argumentation is that of
defeasible reasoning: the conclusion that can be drawn from a set of premises
does not need to hold when additional premises are added. This is in contrast
with classical logic where adding additional premises will never invalidate a
conclusion. Defeasible reasoning allow arguments to be made for or against a
conclusion. Arguments can also support or attack each other and thereby
strengthen or weaken a case for a conclusion.

Commonly computational argumentation in a multi-agent systems is modelled as a 
dialogue game.
In such a dialogue game the agents represent the players and the game rules
prescribes how the dialogue should occur \cite{walton1995}. There are rules
about what arguments agents can put forward, when they are allowed to do so,
and there can be rules about which agent gets to speak when. Most forms of
dialogue games also have rules about when the dialogue is finished and which
agent(s) have won if applicable. These dialogues can be used to give reasons
about why a group of agents decided to take a certain course of action. So they
can be used to remove some of the abstractness of cooperative pathfinding 
algorithms by showing why a solution is preferred over other solutions. 
Conventional algorithms deliver a solution without indicating why that solution 
is preferred over others.

%Discuss how partial global planning gives local to global view, and
%allows for cooperation.
Global cooperation between agents without a central processor is a difficult 
task. There are methods that do achieve a global plan without any single agent 
being vital to create it.
%can also be achieved using partial global planning. It is a distributed method 
%which does not rely on a single agent like centralized methods do.
%Without a central processor it is hard to create a plan that is globally well
%coordinated. There are methods that do achieve a global plan without any single
%agent being vital to create it.
Partial global planning has been used in distributed sensor networks to
distribute and coordinate tasks among the nodes that make up the network
\cite{durfee1991}. The nodes create their individual plans without regard for
each other. They will then exchange information on their plans and adapt them
to better coordinate their activities. Nodes can even take over each others
tasks to spread the computational load. Coordination is not rigid and nodes have
some freedom in how they execute their plan if circumstances change without
having to re-coordinate with the other nodes. None of the involved nodes ever
has a global view, but the end result is a plan that is globally coordinated
with each node holding a part of the global plan.

This method of constructing a global plan from local views can also be applied
to cooperative pathfinding. Agents only have to coordinate with those agents
that they have a conflicting path with. The freedom in planning allows agents
to find an alternative path without having to update all other agents. Other
agents that have previously been coordinated with don't need to update their
plan as a response. This is only necessary when new conflicts arise because of
the alternative path. This allows for a truly decentralised approach where
agents  only communicate with other agents when they have to. There is also no
need to wait for a central processor to tell agents what to do. At the same
time plans are well coordinated without lacking a global view like the
decentralized cooperative pathfinding algorithms.

% describe how to combine these three views
Dialogues can be used in cooperative pathfinding by applying techniques from
partial global planning. When agents have a conflict then they need to
cooperate to avoid conflicts. They can do so by starting a dialogue in which
they share and evaluate different hypothesis to solve the conflict. A
hypothesis consists of a priority ordering for the agents that are involved in
a conflict. The hypotheses offered will be discussed and evaluated in the
dialogue and the agents will give their preference for each proposal. The
proposal which is most preferred is used as the solution to the conflict. All
agents involved in the dialogue adapt the hypothesis. Next they update their
plans so that there are no conflicts between them any more. This means that
there are many small local changes to an agent's position in the hierarchy and
therefore also in their plans. The end result is a global solution to the
cooperative pathfinding problem without any agent having known it explicitly.
There is also no single agent which has been vital to its calculation like in a
centralized approach.

By combining partial global planning, deliberation dialogues we can develop
an algorithm that is able to overcome the weaknesses which other cooperative
pathfinding algorithms have. Using the decoupled method as a starting point we
use partial global planning to remove the reliance on a central processor to
determine the hierarchy. This also prevents a common pitfall of decentralized
methods which are not able to coordinate plans on a global level. So we
essentially achieve a decentralized algorithm that is able to create a global
plan. To enable the use of partial global planning we use deliberation
dialogues so that agents can determine a hierarchy in a decentralized fashion.
Another benefit of using deliberation dialogues is that it is possible to get
reasons why agents settled on a particular hierarchy. This makes it possible to
explain to an end-user why the solution is the best solution. This is not
possible with conventional cooperative pathfinding algorithms because they only
compute a solution according to an algorithm without having any explanatory
power.

% TODO: MAPF is a subcase of coordination

The rest of this thesis is structured as follows. First, a formal description
of the cooperative pathfinding problem is given in \autoref{sec:problem}.
Previous work in cooperative pathfinding, argumentation and partial global
planning is discussed in \autoref{sec:related}. A new method to find
conflict-free paths is proposed in \autoref{sec:method}. The method is
evaluated and compared to other algorithms in \autoref{sec:results}. Final
remarks on the proposed method and its implications are discussed in
\autoref{sec:discussion}.
