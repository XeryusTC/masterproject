\documentclass[a4paper]{article}

\usepackage[utf8]{inputenc}
\usepackage[T1]{fontenc}
\usepackage{natbib}

\title{Literature summary}
\author{Xeryus Stokkel}

\begin{document}

\maketitle

\section{Cooperative pathfinding}

\section{Argumentation}
Argumentation has long been studied by philosophers, but it has been used in 
the field of Artificial Intelligence as well. In AI it has mainly been used in 
legal argumentation (AI \& Law), defeasible reasoning and multi-agent systems. 
One of the main pillars is non-monotonic logic. A logic is non-monotonic when a 
conclusion that follows based on the premises does not necessarily hold any 
more when additional premises are added \cite{vaneemeren2014}. A classic 
example of this is that birds can fly, so when you see a bird you assume that 
it can fly. However, when you're told that the bird is a penguin and that 
penguins can't fly then you will no longer conclude that the bird can fly. An 
argument is defeasible when it can be defeated by other arguments, in the 
previous example the fact that the bird that you see can fly is defeasible.

Pollock distinguishes two different types of defeating arguments 
\cite{pollock1995}. \emph{Rebutting defeaters} attack an argument directly and 
give a reason for an opposite argument. \emph{Undercutting defeaters} do not 
attack an argument directly, instead they attack the relation between an 
argument and its support. The standard example given by Pollock is about an 
object that looks red: "The ball looks red to John" is a support for John to 
believe that the ball is red, but there may be a red light shining on the ball. 
This is a undercutting defeater because it does not attack the conclusion 
directly, instead it attacks the relation between the observation and the 
conclusion that the ball is red, after all, a white object with a red light 
shining on it will also look red.

Other researchers have formulated additional forms of defeaters, but they can 
be distilled into three main forms \cite{vaneemeren2014}:
\begin{description}
	\item[Undermining defeaters] attack the premises or assumptions of an 
	argument.
	\item[Undercutting defeaters] attack the connection between a set of 
	reasons and the conclusion in an argument.
	\item[Rebutting defeaters] attack an argument in favour of an opposite 
	conclusion.
\end{description}

A formal model of argumentation that introduces a structure to ease the 
computation of validity in arguments has been proposed in a highly influential 
paper by Dung \cite{dung1995}. This work focused mainly on the argument attacks 
as a formal relation, giving the model the name of abstract argumentation. The 
main concept is the \emph{argumentation framework}, a directed graph in which 
the nodes form arguments and the edges between them represent one argument 
attacking another. An important concept that Dung introduced was that of 
admissibility of sets of arguments. A set of arguments is admissible when it is 
conflict free and acceptable. A set being conflict free means that no argument 
in the set attacks another argument in the set. Acceptability means that when 
an argument is attacked by another argument outside of the set, then the set 
attacks that argument. In other words the admissible set defends itself from 
attacking arguments. On top of this Dung formulated other semantics. The 
preferred extension is the set theoretically maximal admissible set, that is, 
it is the largest possible admissible set such that adding one argument from 
the argumentation framework would make it not admissible. There is also the 
stable extension, this is an admissible set that attacks all arguments that are 
not in the set.

One important notion of an argumentation framework is that of the grounded 
extension. This extension is simple to compute by labelling the arguments in 
the argumentation framework as `justified' or `defeated':
\begin{enumerate}
	\item All unlabelled arguments $\alpha$ in the framework can be labelled as 
	`justified' if all arguments that attack $\alpha$ are labelled as 
	`defeated'. Note that when $\alpha$ is not attacked that it can then also 
	be labelled as `justified'.
	\item All unlabelled arguments $\alpha$ in the framework that are attacked 
	by an argument that has been labelled `justified' is labelled as `defeated'.
	\item Steps 1 and 2 are repeated until all arguments have been labelled.
\end{enumerate}
A finite argumentation framework is labelled in a finite amount of steps. All 
arguments that have been labelled as `justified' are included in the grounded 
extension. All arguments that have been labelled `defeated' are not included in 
the grounded extension.

\subsection{Dialogues}
Multiple agents can have an argument through a dialogue. Walton and 
Krabbe \cite{walton1995} proposed a typology of main dialogues that humans 
partake in. They distinguish six main types of dialogues, it should be noted 
that the list of dialogue types is not exhaustive. In \emph{information 
seeking} dialogues some of the participating agents aim to gather information 
from another agent that knows the anser. In \emph{inquiry} dialogues a group of 
agents collectively seeks an answer to a question to which non of the 
participating agents knows the answer on its own. \emph{Deliberation} dialogues 
are about what course of action to take in a given situation. A 
\emph{persuasion} dialogue occurs when an agent tries to convince on or 
multiple other agents of its position. It is successful when the other agent(s) 
adopt its position. Participants of \emph{negotiation} dialogues try to find a 
division of a scarce resource that all agents can be satisfied with. Finally 
\emph{eristic} dialogues are a verbal substitute for fighting.

\bibliographystyle{plain}
\bibliography{biblio}

\end{document}
