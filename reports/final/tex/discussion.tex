\section{Discussion}\label{sec:discussion}
% cache
% DPCA*+ and PCA are simular, later is simpler
With DPCA* agents resolve conflicts in pairs only while DPCA*+ can have larger
groups of agents resolve a conflict. This only happens when multiple agents try
to make conflicting moves at the same time. Allowing larger groups in a
dialogue means that it gets more complex, this is reflected in the fact that
PPCPF+ is slower overall and on individual problem instances. Larger groups of
agents solving conflicts is not necessarily faster than pairs of agents solving
conflicts. Several small pairwise dialogues can be very effective because the
pair $a_1$ and $a_2$ resolving their conflict means that $a_2$ gets rerouted,
this means that $a_2$'s conflict with $a_3$ is also resolved. This means that
$a_1$ and $a_3$ only need to come to a solution. In the case that all three
agents would find a solution in a single dialogue they will have to evaluate
multiple proposals. The added complexity of three agents trying to find a
solution to the conflict does not outweigh the speed of starting several
dialogues which come to a conclusion quickly.

%window version being better
WDPCA* is able to solve more problem instances than DPCA* without a window. How
many more instances can be solved depends on the size of the window. As a
trade-off the amount of actions required for agents to reach their destination
increases. This effect get stronger when the window size gets smaller, when
$w=2$ most of the instances were solved, however the sum of the path lengths
was also the highest. That the speed of WDPCA-2 is high is not surprising, it
reduces the algorithm to be reactive. Agents mainly solve conflicts that will
happen during the next time step, and some conflicts that happen a time step
later. Agents will also determine whether they have conflicts after every time
step. This shows that creating a complete global plan can be sacrificed to
obtain smaller plans in a shorter time.

There is an interaction between which algorithm is used and whether previously
found paths were stored in a cache and reused. From \autoref{fig:cache} it
becomes clear that a smaller window also means that the effect is smaller. With
a small window there are fewer conflicts, so agents will not have to
participate in conflicts very often. This means that they will not consult the
cache as often and therefore there is less of a speed-boost. On top of that the
paths are also shorter and easier to calculate, so retrieving a path from the
cache is not faster than calculating the path outright.