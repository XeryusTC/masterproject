\documentclass[a4paper]{article}

\title{Argumentation for cooperative pathfinding}
\author{}
\date{}

\begin{document}

\maketitle

\section*{Problem}
When multiple mobile robots share the same space they need to be able to plan
paths that not only avoids collisions with obstacles, but also with each other.
This field of research is called `cooperative pathfinding'. Simply taking the
Cartesian product of all the agent's state spaces results in an exponential
increase of the size of the state space as the number of agents grow. Smart
algorithms are necessary to constrain the size of the problem.

\section*{State of the art}
There are several paradigms in the field of cooperative pathfinding:

\subsection*{Centralized}
In this paradigm a central processor computes the paths for all agents. The
current best method is known as Operator Decomposition (OD), instead of taking
the Cartesian product of all agent's state spaces it plans for each agent in
turn per time step. This causes the problem to be linear in the number of agents
instead of being exponential. The algorithm will always find the most optimal
solution if there is one.

\subsection*{Decentralized}
Another common approach is to let each agent find its most optimal path on its
own. Commonly some priority order is decided beforehand, agents with a lower
priority are tasked with avoiding the agents with a higher priority. The ability
of this method to find a solution depends on the priority order that was picked.

\subsection*{Environment-based}
One method of using the environment for cooperative pathfinding is to register
the movements of agents at each position in the environment. When another agent
visits the same position it will tend to move in the direction that agents
previously moved in. This results in `roads' on which agents all move in the
same direction. Collision avoidance is implied by this, unfortunately it can
lead to long paths when agents need to take a detour because they cross some
heavy traffic area.

There is also a purely reactive environment based cooperative pathfinding
technique called Optimal Reciprocal Collision Avoidance (ORCA) in which agents
are assumed to all share the same rules for collision avoidance. Agents observe
each other's current velocity and will change their own velocity according to
the rules previously set when they are on a collision course with another agent
in $\tau$ seconds. Because all agents use the same rules it can be predicted
what other agents will do and thus future collisions can be avoided as well.
When this is combined with other pathfinding techniques it is computationally
efficient (hundreds or thousands of agents can be controlled in real time) but
it can lead to suboptimal solutions because of the reactive nature.

\section*{New idea}
An argumentation based approach may add some robustness to cooperative
pathfinding. Agents can share information about each other, and information
about agents that are not part of the system. That way agents can still infer
each other's (future) actions and respond to those.

\section*{Results}
The expected results are a system that is able to still find a (partial)
solution even if a portion of the agents in the system are not controlled by
it. This is in contrast to the state of the art algorithms which assume that
all agents are controlled by it and that will not find a solution if too large a
proportion is not controlled by it.

\section*{Relevance}
Most of the algorithms in the field of cooperative pathfinding do well in
finding solutions for a group of agents, they even perform well when 80\% of
nodes in a graph are occupied. Some outright require all agents to use the same
method while with most it is implied. In situations like traffic this is not
always possible, even if all cars where autonomous then individual manufacturers
might not agree on which algorithms to use. When adding human drivers then there
are agents in play that could not even communicate with other agents about their
planned paths. This means that existing algorithms might find solutions that do
not work in these kinds of situations.

\end{document}