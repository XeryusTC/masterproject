\section{Family of algorithms}\label{sec:method}
Decoupled algorithms are able to solve cooperative pathfinding problems while
requiring only minimal computational resources. The agents calculate their path
individually so this is inherently distributed. Calculating a hierarchy needs
to be done centrally so all dependencies between agents can be taken into
account \citep{latombe1991,bennewitz2002}. This means that there is a single
centralized bottleneck in an otherwise distributed system. In this article we
aim to overcome this bottleneck and make the calculation of the priority
ordering distributed as well. Below are the details of two different versions
of this algorithm. The second version version of the algorithm has some
improvements over the previous version. A summary of their properties is given
in \autoref{tbl:proposed}. Each algorithm is build upon Cooperative A*, a
variation on A* \citep{hart1968} which allows teams of agents to cooperate.
This variant does not allow agents to expand nodes if it would mean that the
action will conflict with the path of an agent with higher priority.

%\begin{table}
%    \centering
%    \caption{Comparison of proposed cooperative pathfinding algorithms. The
%        columns have the same meaning as those in
%        \autoref{tbl:planning-overview}. Both algorithms are decentralized and
%        not complete. The valuations column indicates whether agents vote on
%        their preferred solution.}
%    \label{tbl:proposed}
%    \begin{tabular}{l|l|l|l}
%        & Comm. & Online & Valuations \\ \hline
%        PCA*   & All & No & No \\
%        DPCA*  & All & No & Yes \\
%    \end{tabular}
%\end{table}

\subsection{Partial Cooperative A* (PCA*)}
The algorithm follows the approach of decoupled methods. During the first step
agents calculate their optimal path and share them with each other. This allows
each agent to determine where they have conflicts. During the second step the
agents will start dialogues for each conflict and update their plan after each
dialogue they finish. The result of the dialogue is a partial priority
ordering. Other methods calculate a permutation of the full priority scheme
$a_1 > a_2 > \ldots > a_k$. The result of a dialogue is a priority ordering for
the agents involved in the dialogue, ignoring those that did not partake.

Agents will always solve the conflict that occurs the earliest in their path.
When an agent is invited to a dialogue that occurs later it will notify the
agents in that conflict of this and the dialogue is put on hold. This is
because solving a conflict may have the side-effect of solving conflicts that
occur at a later time. During a dialogue the agents that participate will
evaluate all possible priority orders for them. They will use the priority
order that has the smallest increase in total path length for the group of
agents. There is no consideration for other effects on the solution. Only the
agents that participate in the conflict adapt the priority, other agents in the
problem are not notified of what the solution is. It may be the case that some
agents will not have any conflicts which also means that they do not occur in
any priority ordering and will not be present in the implied global hierarchy.
This has the effect of implied Independence Detection because these agents will
never have communicated beyond sharing their optimal path.

\subsection{Dialogue-based Partial Cooperative A* (DPCA*)}
\begin{table}
    \centering
    \caption{Stages of a conflict resolution dialogue.}
    \label{tbl:stages}
    \begin{tabularx}{\columnwidth}{l|X|l}
        Stage & Purpose & Next stage \\ \hline
        Opening & Exchange information & Proposal \\
        Proposal & Make (incomplete) priority proposals & Evaluation \\
        Evaluation & Vote on suitability of proposals & Proposal, Closing \\
        Closing & Permanently adapt best proposal & \\
    \end{tabularx}
\end{table}

Going through all possible permutations and evaluating them on a single
criterion is not the most clever method of finding a priority ordering
\citep{bennewitz2002}. To avoid evaluating the entire search space we employ a
deliberation dialogue during the conflict resolution step. This results in an
algorithm we call DPCA*. The deliberation dialogues consist of four stages as
summarised in \autoref{tbl:stages}. The stages are based on those used in
\textsc{TeamLog}. Each conflict has its own dedicated dialogue which starts in
the opening stage. Only agents that are involved in the conflict participate in
the dialogue. During the opening stage agents can notify each other of earlier
occurring conflicts. If this is the case then the dialogue is put on hold.
Otherwise the dialogue moves on to the proposal stage.

During the proposal stage agents can make priority order proposals. In our
implementation agents always propose that they get priority over the other
agents. After all agents have made a proposal or declined to make a proposal
the dialogue moves to the evaluation stage. During this stage the agents cast
votes on the proposals made during the previous stage. To be able to cast a
vote on a proposal the agents temporarily adopt the proposal and make a new
plan which adheres to its constraints. After updating their plan an agent can
evaluate and weigh two effects of the proposal. The first effect is the
difference in path length between their original (conflicting) path and the new
path. The second effect is the difference of the total of number of conflicts
in the two paths. These two effects are weighted and the result is cast as a
vote. Agents can also request an additional proposal stage to occur after the
evaluation stage. When an agent is unable to find a path to its destination
under a certain priority ordering then it can also notify the agents of this,
it is impossible to accept this priority ordering as the solution to the
conflict. After all agents have cast their vote on each proposal the evaluation
stage finishes. If no agent requested an additional proposal stage the dialogue
moves to the closing stage. Otherwise there is a new proposal stage where
agents can make new priority ordering proposals, followed by another evaluation
stage. During the closing stage all agents permanently adapt the priority
ordering which has the lowest sum of votes cast. If none of the proposals can
be accepted then the algorithm results in a failure.

%Conflicts that involve more than two agents can be solved in two different
%ways. The first is to let agents solve the conflict in pairs, this approach is
%known as Dialogue-based Partial Cooperative A* (DPCA*). In a conflict between
%the agents $a_1$, $a_2$ and $a_3$ at time $t$ then there would be three
%dialogues: one between $a_1$ and $a_2$, one between $a_1$ and $a_3$, and one
%between $a_2$ and$a_3$. Say that $a_1$ and $a_2$ are the first to hold a
%dialogue which finishes with the priority $a_1 > a_2$, meaning that $a_1$ has
%priority over $a_2$. Agent $a_2$ will have found a path that does not go
%through the location of the conflict at $t$. This has the effect of also
%solving the conflict between $a_2$ and $a_3$. Now only $a_1$ and $a_3$ still
%have a conflict at $t$ and they will have to hold a dialogue about which agent
%gets priority over the other. When this dialogue would end with the priority
%scheme $a_1 > a_3$ then the multi-agent conflict at $t$ is solved. It may be
%the case that $a_2$ and $a_3$ now have another conflict at a different position
%and/or time that they will have to resolve. For this conflict there will be at
%least two dialogues that need to lead to a conclusion, and at most three
%conflicts if $a_2$ and $a_3$ do have a conflict at a different location.

Conflicts that involve more than two agents can be solved in two different
ways. The most simple is to let agents solve conflicts in pairs, this is known
as Dialogue-based Partial Cooperative A* (DPCA*). In a conflict which involves
three agents then there will also be three dialogues, it is possible however
that the first two dialogues that find a solution which solves the entire
conflict making the third dialogue obsolete. If more than two agents have a
conflict at the same time about the same position then it is also possible to
hold one dialogue with all of them as participants, this is known as
Dialogue-based Partial Cooperative A* Plus (DPCA*+). A dialogue with multiple
agents requires that the dialogue supports proposals with a partial priority
ordering. These are proposals in which not every agent that participates in the
conflict has a position in the ordering, they are assumed to share the lowest
priority. The dialogue also includes the ability to request an additional round
of making proposals and evaluating them. During the proposal stage agents can
then expand on a partial proposal by assigning more agents with an explicit
position in the ordering. This makes DPCA*+ more complex than DPCA*, but it
also requires fewer dialogues to find a solution and therefore it may be faster
than DPCA*.

Each time that agents evaluate a proposal they have to compute paths that
satisfy the constraints imposed by the priority ordering. Often they have to
recompute the same paths. To reduce the amount of duplicate computation the
agents can cache the paths that they have calculated. Before calculating a path
the agents consult their path cache if there is a path which satisfies the
constraints set by the permanent and temporary priority ordering. In the case
that there is such a path and it does not cause any conflicts with higher
priority agents then it is used. If this is not the case then the agent
calculate a new path. An agent can use a cached path even if it causes
conflicts with lower priority agents or agents with which it has not
established a priority ordering with.

%\begin{figure*}
%    \centering
%    \begin{subfigure}[t]{.3\textwidth}
%        \centering
%        \def\svgscale{.6}
%        \input{images/dialogue-example-initial.pdf_tex}
%        \caption{Initial configuration.}
%        \label{fig:example-initial}
%    \end{subfigure}
%    ~
%    \begin{subfigure}[t]{.3\textwidth}
%        \centering
%        \def\svgscale{.6}
%        \input{images/dialogue-example-prop1.pdf_tex}
%        \caption{Configuration after proposal $a_2 > a_1$.}
%        \label{fig:example-prop1}
%    \end{subfigure}
%    ~
%    \begin{subfigure}[t]{.3\textwidth}
%        \centering
%        \def\svgscale{.6}
%        \input{images/dialogue-example-solution.pdf_tex}
%        \caption{Configuration after proposal $a_1 > a_2$.}
%        \label{fig:example-solution}
%    \end{subfigure}
%    \caption{Three stages of resolving a conflict. Agents are circles inscribed
%    by $a_i$, their respective goals are $g_i$. Paths that are followed are
%    indicated by the arrows. The red dot represent where two paths have
%    conflicting moves.}
%    \label{fig:example}
%\end{figure*}
%
%\paragraph{Example of conflict resolution} Consider the cooperative pathfinding
%problem in \autoref{fig:example-initial}. It shows a $4 \times 4$ grid with
%three agents in it in their starting positions. The initial optimal paths they
%found during the first step of the algorithm are shown as arrows.
%Agent $a_1$ has a path that consists of three \emph{south} moves, $p_{a_{1},1}
%= \{\text{\emph{south, south, south}}\}$, while both $a_2$ and $a_3$ have paths
%that consist of three consecutive \emph{east} moves, $p_{a_2,1} = p_{a_3,1} =
%\{\text{\emph{east, east, east}}\}$. None of the agents has a \emph{wait}
%action in their path. After finding the initial paths all three agents share
%their paths with each other, agents $a_1$ and $a_2$ discover that they collide
%after their first action. This means that they will have to resolve this
%conflict to prevent the collision.
%
%Before discussing the details of how the agents resolve the conflict in this
%situation some definitions are needed. A proposal where agent $a_i$ has
%priority over a agent $a_j$ is represented as $a_i > a_j$. Agents can cast
%votes on proposals, $\vote_{a_n}(a_i > a_j)$ is the vote of $a_n$ on the
%proposal $a_i > a_j$. To be able to cast votes agents need to evaluate a
%proposal. An evaluation  is based on two effects which can be weighted
%differently. The first effect is the difference in path length before and after
%a proposal has been adopted. The second effect is the change in the number of
%conflicts that an agent is involved in, this effect is weighted three times
%heavier than the former effect. Qualitatively this means that $\vote_{a_n}(a_i
%> a_j) = 1 \cdot \Delta l + 3 \cdot \Delta c$ where $\Delta l$ is the
%difference in path lengths and $\Delta c$ is the difference in conflicts.
%Agents only communicate their final vote and not how they arrived at it.
%
%Continuing with the example, after the agents have determined that they have a
%conflict they start a new dialogue. The dialogue starts in its initial stage,
%the opening. All messages are broadcast to all agents in the conflict dialogue,
%in this case $a_1$ and $a_2$:
%\\ \-\qquad $a_1$: no earlier conflicts
%\\ \-\qquad $a_2$: no earlier conflicts
%
%Because both agents don't have any conflicts that occur at an earlier time
%(this is the first time step) the dialogue can move on to the proposal stage.
%Both agents make a proposal in which they go first:
%\\ \-\qquad $a_2$: propose $a_2 > a_1$
%\\ \-\qquad $a_1$: propose $a_1 > a_2$
%
%Both agents have made a proposal to resolve the conflict. The dialogue can
%move to the evaluation stage. Agent $a_2$'s proposal will be evaluated first
%because $a_2$ submitted the proposal before $a_1$. In this proposal agent $a_2$
%has the highest priority so it doesn't need to update its plan and its path
%remains $p_{a_2,1}$. It updates $a_1$ of the fact that its path hasn't changed
%and still consists of three consecutive \emph{east} actions:
%\\ \-\qquad $a_2$: new path $p_{a_2,1} = \{\text{\emph{east, east, east}}\}$
%
%Agent $a_1$ does have to yield to $a_2$ so it will have to consider possible
%conflicts that arise with $p_{a_2,1}$ and plan around them. It finds a new path
%$p_{a_1,2} = \{\text{\emph{south east, south, south west}}\}$ which is shown in
%\autoref{fig:example-prop1}. After finding the new path $a_1$ will evaluate its
%quality so that it can send its vote to $a_2$. Before adopting the proposal the
%agent had one conflict, this has not changed after making a new plan so $\Delta
%c = 0$. There is no difference in the length of the paths before and after
%temporarily adopting the constraints of the proposal, so $\Delta l = $. This
%means that agent $a_1$ can send the vote to $a_2$. It will also send its new
%path along with the proposal
%\\ \-\qquad $a_1$: new path $p_{a_1,2} = \{\text{\emph{south east, south,
%south west}}\}$
%\\ \-\qquad $a_1$: $\vote_{a_1}(a_2 > a_1) = 0$
%
%Now that $a_2$ knows the new path of agent $a_1$ after adopting to $a_2 > a_1$
%it can also vote on the proposal. This happens in a similar vain as $a_1$'s
%evaluation, there is however one fewer conflict for $a_1$ as its path doesn't
%conflict with that of $a_2$ any more. Achieving this was the goal of the
%dialogue. The path remained the same so $\Delta l = 0$ and $\Delta c = -1$, the
%vote of $a_2$ is then $\vote_{a_2} = 1 \cdot \Delta l + 3 \cdot \Delta c = -3$.
%This vote is then cast:
%\\ \-\qquad $a_2$: $\vote_{a_2}(a_2 > a_1) = -3$
%
%Now that this proposal has been evaluated by both agents they can find the sum
%score of the proposal which is $\eval(a_2 > a_1) = \eval(a_1, a_2 > a_1) +
%\eval(a_2, a_2 > a_1) =
%-3$. Next the agents can evaluate $a_1 > a_2$. When agent $a_1$ goes to
%evaluate this conflict if finds that it can use the path $p_{a_1,1}$ which is
%stored in its cache. It can use this because it has a priority scheme similar
%to the one being evaluated; $a_1$ does not have to yield to any agent. It
%notifies $a_2$ of this:
%\\ \-\quad $a_1$: new path $p_{a_1,1} = \{\text{\emph{south, south, south}}\}$
%
%Next $a_2$ can evaluate the proposal. First it needs to plan a new path that
%does not conflict with the path that $a_1$ has just send. It finds the path
%$p_{a_2,2} = \{\text{\emph{north easth, east, south east}}\}$. The new
%situation is shown in \autoref{fig:example-solution}. It can
%immediately evaluate it, the lengths of $p_{a_2,1}$ and $p_{a_2,2}$ are equal
%and $a_1$ has one fewer conflicts so $\Delta c = -1$.
%\\ \-\quad $a_2$: new path $p_{a_2,2} = \{\text{\emph{north easth, east, south
%east}}\}$
%\\ \-\quad $a_2$: $\eval(a_2, a_1 > a_2) = 1 \cdot (|p_{a_2,2}| - |p_{a_2,1}|)
%+ 3 \cdot \Delta c = -3$
%
%Agent $a_1$ can also send its evaluation to $a_2$:
%\\ \-\quad $a_1$: $\eval(a_1, a_1 > a_2) = 1 \cdot (|p_{a_1,1}| - |p_{a_1,1}|)
%+ 3 \cdot \Delta c = -3$
%
%Now both agents have evaluated the proposal $a_1 > a_2$ they can find the sum
%of the evaluations which is $\eval(a_1 > a_2) = \eval(a_1, a_1 > a_2) +
%\eval(a_2, a_1 > a_2) =
%-6$. All proposals have been evaluated so the agents can notify each other if
%they want to make more proposals:
%\\ \- \quad $a_1$: no more proposals
%\\ \- \quad $a_2$: no more proposals
%
%Neither agent wants to make more priority ordering proposals, this is because
%there are no more possible priority orderings to make with these two agents.
%The dialogue can then move to the closing stage. In the closing stage agents
%will pick the best proposal, in this case that is the proposal with the lowest
%sum score which is the proposal $a_1 > a_2$. To adapt this proposal agents will
%need to use the respective paths that they used during the evaluation of the
%proposal. So $a_1$'s path will be $p_{a_1,1}$ while $a_2$'s path will be
%$p_{a_2,2}$. They also need to keep track of which agents have a higher
%priority than themselves. For agent $a_1$ nothing changes, while $a_2$ must now
%store that $a_1$ has a higher priority than it. All conflicts in
%\autoref{fig:example} have now been solved and agents are free to execute their
%paths.
%
%To evaluate a proposal an agent needs to weigh different factors of the
%quality. In this example a weight of 1 for the path length and a weight of 3
%for the number of solved/introduced conflicts was used. These weights are used
%for demonstration only and an implementation should have these weights set
%empirically. The weights can be any real number. When dialogues between groups
%of three or more agents are possible then a third penalty weight can be added.
%This penalty weight is included in the evaluation if there are two or more
%agents that still have a conflict with each other after adapting a priority
%proposal. It is a penalty for when a proposal only partially solves a conflict.
