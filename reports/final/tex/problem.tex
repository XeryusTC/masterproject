\section{Problem formulation}\label{sec:problem}
A shared space is divided into discrete cells such that it forms an 8-connected
grid. Some of the cells in this grid are static obstacles while the other cells
are open. A set of $k$ agents with respective initial positions $a_1, \ldots,
a_k$ and goal positions $g_1, \ldots, g_k$ need to find paths such that no two
agents have a conflict. A path consists of a series of moves, which can either
be to move to one of the neighbouring cells or \emph{wait} at the current
location. Each time step an agent can take exactly one of these actions.
%A shared space is divided into discrete cells such that it forms a regular
%grid. Some of these cells can be occupied by an obstacle while the other cells
%are open. A set of $k$ agents are located on the open cells of this grid, each
%one of them occupies a single cell and no two agents share the same cell. Each
%agent has a unique destination. Each time step an agent can move to one of the
%eight neighbouring cells or \emph{wait} in its current location.
An example initial configuration is shown in \autoref{fig:world}. The goal is
to find a series of moves, or paths, one for each agent, such that the agents
reach their destination without colliding with other agents or running into the
static obstacles.

The paths of two agents are not in conflict iff at no time step the agents
occupy the same cell, agents move along the same edge (swap positions), or
agents move along
crossing edges. Obstacle cells can be considered as stationary agents. Agents
are allowed to move along a diagonal even when the two cells on the opposing
diagonal are blocked, i.e. agent 5 in \autoref{fig:world} can move to its
destination in a single time step. An agent $a_i$ can move to a cell occupied by
agent $a_j$ given that $a_j$ will move to a different cell at the same time.
Agents 1--4 in \autoref{fig:world} can move their respective destinations
by ``rotating'' clockwise. They can do this in a single time step without
requiring any additional empty cells. Agents 7 and 8 cannot move to their
destinations in a single time step because that would mean that they move along
crossing edges at the same time. They can also not swap places because then
they would be travelling along the same edge.

\begin{figure}[h]
    \centering
    \def\svgscale{.7}
    \input{images/world.pdf_tex}
    \caption{A small space shared by some agents. Obstacles are black, agents
        are circles inscribed with the agent's number ($a_i$), the destination
        for
        agent $a_i$ is given by $g_i$.}
    \label{fig:world}
\end{figure}
