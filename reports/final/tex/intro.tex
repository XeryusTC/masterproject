\section{Introduction}\label{sec:intro}
Path planning for a single agent has long been considered a solved problem.
When multiple agents have to find their way through a shared space then the
problem becomes more complicated. The agents have to find paths around
obstacles, and they also need to ensure that they do not collide with each
other. Even when agents can prevent collisions then it is still possible that
congestions or even deadlocks may occur. To avoid this there is a need for
coordination between the agents. Cooperative pathfinding finds its application
in robotics, aviation, road traffic management, crowd simulations, and video
games \cite{standley2011}.

%TODO: Discuss cooperative pathfinding and abstract solutions.
The most straightforward approach to the cooperative pathfinding problem is to
search the Cartesian product of the state spaces of all agents. Unfortunately
this approach is very inefficient \cite{sharon2013}, the time to find a
solution is exponential in the number of agents. Another common approach is to
assign all agents a priority. Agents plan a path to their destination in
descending order of priority, they consider agents with a higher priority to be
a moving obstacle and avoid planning movements that conflict with those of
higher priority agents. Both of these approaches result in abstract solutions;
there is often no clear reason why a certain solution
was the one arrived at. The algorithm has found a set of conflict free paths
and that set was returned as the working solution without any indication about
the considerations of why it is a good plan.

These two approaches to solving the cooperative pathfinding problem both rely
on a central processor. The former method uses a central processor to plan for
all the agents. The latter method requires that a central processor determines
a priority ordering that the agents have to adhere to. After this has been done
then agents can make their individual plans. While doing so they need to
communicate with each other about the rules but this does not require a central
point of communication. Next to these two centralised methods there are also
decentralized approaches where agents only communicate in a small range. There
is no central processor that can be a single point of failure. As a trade-off
these methods usually have no global view of the problem, so agents can make
decisions early on that will lead to congestions or deadlocks later without any
agent noticing at the time that the decision was made.

%TODO: Discuss how partial global planning gives local to global view, and
%allows for cooperation.
Without a central processor it is hard to create a plan that is globally well
coordinated. There are methods that do achieve a global plan without any single
agent being vital to create it.
Partial global planning has been used in distributed sensor networks to
distribute and coordinate tasks among the nodes that make up the network
\cite{durfee1991}. The nodes create their individual plans without regard for
each other. They will then exchange information on their plans and adapt them
to better coordinate their activities. Nodes can even take over each others
tasks to spread the computational load. Coordination is not rigid, nodes have
some freedom in how they execute their plan if circumstances change without
having to re-coordinate with the other nodes. None of the involved nodes ever
has a global view, but the end result is a plan that is globally coordinated
with each node holding a part of the global plan.

This method of constructing a global plan from local views can also be applied
to cooperative pathfinding. Agents only have to coordinate with those agents
that they have a conflict with. The freedom in planning allows agents to find
alternative paths without having to update all other agents that were ever
coordinated with. This allows for a truly decentralised approach,
agents only communicate with other agents when they have to, and there is no
need to wait for a central processor to tell agents what to do.

Other methods of decentralised coordination have been developed by the field of
argumentation. Formal models of argumentation have been used in Artificial
Intelligence to reason about topics like expert systems, multi-agent systems
and law \cite{vaneemeren2014}. An important concept in argumentation is that of
defeasible reasoning: the conclusion that can be drawn from a set of premises
does not need to hold when additional premises are added. This is in contrast
with classical logic where adding additional premises will never invalidate a
conclusion. Defeasible reasoning allows arguments to be made for or against a
conclusion. Arguments can also support or attack each other and thereby
strengthen or weaken a case for a conclusion.

Commonly reasoning in a multi-agent system is modelled as a dialogue game. In
such a dialogue game the agents represent the players and the game rules
prescribes how the dialogue should occur \cite{walton1995}. There are rules
about what arguments agents can put forward, when they are allowed to do so,
and there can even be rules about which agent gets to speak when. Most forms of
dialogue games also have rules about when the dialogue is finished and which
agent(s) have won if applicable. These dialogues can be used to give reasons
about why a group of agents decided to take a certain course of action. They
can thus be used to remove some of the abstractness of cooperative pathfinding
solutions by giving reasons why this course of action was arrived at.
%
%%TODO: describe how to combine these three views
Dialogues can be used in cooperative pathfinding by applying techniques from
partial global planning. When agents have a conflict then they need to
cooperate to avoid conflicts, they can do so by starting a dialogue. In this
dialogue they can offer different hypotheses to solve the conflict. The
hypotheses offered will be discussed and evaluated in the dialogue and the best
one (according to various criteria) is used as the solution to a conflict. All
agents involved in the dialogue adapt their plans so that they are better
coordinated. This means that there are many small local changes to agent's
plans, with the result of a global solution to the cooperative pathfinding
problem.

The rest of this thesis is structured as follows. First, a formal description
of the cooperative pathfinding problem is given in \autoref{sec:problem}.
Previous work in cooperative pathfinding, argumentation and partial global
planning is discussed in \autoref{sec:related}. A new method to find
conflict-free paths is proposed in \autoref{sec:method}. This method is
evaluated and compared to other algorithms in \autoref{sec:results}. Final
remarks on the proposed method and its implications are discussed in
\autoref{sec:discussion}.
