\begin{abstract}
Cooperative pathfinding research studies coordination algorithms
addressing congestions, deadlocks, and collisions in multi-agent
systems. In typical algorithms individual agents have no say in
resolving conflicts. We propose algorithms in which agents engage in an
argumentative dialogue in case of local conflicts, leading to the
transparent and fast construction of global solutions.  We combine ideas
from computational argumentation, multi-agent coordination and continual
planning.  From computational argumentation we use argumentative
deliberation dialogues in which agents discuss and resolve conflicting
local plans.  From the study of multi-agent coordination we use partial
global planning, a distributed method to incrementally create a global
plan.  Using ideas from continual planning we obtain an online algorithm
in which planning and execution are interleaved.  We show that our
algorithms generally solve cooperative pathfinding problems faster than
a state of the art complete and optimal algorithm, at the cost of
slightly longer path lengths and gaining the explanatory power of
argumentation dialogues. An online version of our algorithm is the
fastest with the trade-off that it has the lowest quality paths. 
\end{abstract}