\section{Methodology}\label{sec:method}
Decoupled algorithms are able to solve cooperative pathfinding problems while
requiring only minimal resources. Unfortunately they have a single bottleneck:
the calculation of the priority scheme. In a distributed multi-agent system all
agents will have to halt calculating a solution while the priority scheme is
determined in a centralized fashion. To overcome this bottleneck the
calculation of the
priority ordering can also become distributed. The base decoupled method can be
altered to allow for this, the first step where each agent plans its optimal
path without regard for the other agents remains the same. Next agents share
their paths with each other and they will be able to determine where the paths
have conflicting moves that would lead to a collision. The agents will then be
able to solve the conflicts without having to wait for slower agents to
calculate and communicate their optimal paths. To solve conflicts agents will
start a dialogue where possible solutions are proposed, evaluated and adapted.
The proposals made consist of a priority order for the agents involved in the
conflict. Agents will only need to solve the first conflict that occurs in
their path because solving it may have the side-effect of solving later
conflicts. After a conflict has been successfully solved then the agents
involved can work on solving the next conflict.
Below are the details of three different versions of this algorithm, each
version of the algorithm has some improvements over the previous version.

\subsection{Partial Cooperative A*}
The simplest approach to resolving the conflicts is by creating all possible
ordering permutations for the agents involved in the conflict. The permutation
that has the lowest sum path length is used as the solution for the conflict.
This simple method of evaluating the various solutions is unlikely to find the
most appropriate one because it is not able to predict the consequences of
selecting one solution over the others.
Most conflicts involve just two agents, so there are only two possible ordering
permutations which means that the amount of priority permutations that need to
be evaluated is low.

\subsection{Dialogue-based Partial Cooperative A*}
\begin{table}
    \centering
    \caption{Stages of a conflict resolution dialogue.}
    \label{tbl:stages}
    \begin{tabular}{l|l|l}
        Stage & Goal & Next stage \\ \hline
        Opening & Exchange information & Proposal \\
        Proposal & Make (incomplete) priority proposals & Evaluation \\
        Evaluation & Vote on suitability of proposals & Proposal, Closing \\
        Closing & Permanently adapt best proposal & \\
    \end{tabular}
\end{table}

Going through all possible permutations and evaluating them on a single
criterion is not the most
clever method of finding a priority ordering \cite{bennewitz2002}. To reduce
the amount of computation needed to find a solution some improvements can be
made. Using a dialogue to solve conflicts is more intelligent than naively
computing all possible combinations. Agents start a dialogue of which the goal
is to find a solution to the conflict that works for all involved agents. This
dialogue consists of several stages which are summarised in
\autoref{tbl:stages}. The first stage is the opening where agents enter the
dialogue and tell each other whether they have any conflicts that are occurring
at an earlier time. If this is the case then resolution is postponed until
there are  no prior occurring conflicts any more. If all agents involved in the
conflict do not have any
earlier conflicts then the dialogue moves to the proposal stage. Each agent
gets to enter a new proposal for an ordering. This ordering doesn't need to be
complete, it can assign one agent the highest priority while other agents are
assigned the same lower priority.

The third stage is the evaluation stage which is reached when all agents have
made a proposal or declined to make one. Each of the new proposals will be
evaluated in turn. During the evaluation of a proposal each agent has to update
their plan, the restrictions imposed by the ordering of the proposal should be
respected, as well as the restrictions imposed by previously solved conflicts.
Once an
agent is done with updating its plan then it will cast a vote based on how
suitable the proposal is. All votes are collected and the best proposal is
used as the solution. It is also possible that an agent was blocked from
reaching its destination with the proposed priorities, in this case an agent
can also notify the other agents of this and the proposal is rejected.

When multiple agents are involved then it is possible that a proposal can not
be solved by giving one agent priority over all the others, in this case a more
complete priority ordering is required. If this happens then the dialogue will
have another proposal stage. Agents will now be allowed to expand on earlier
proposals as well as making completely new proposals. Afterwards there is
another evaluation stage.

After each evaluation stage the agents will tell each other whether they want
to make additional proposals. If none of the agents does so then the dialogue
moves to the closing stage. During the closing stage the proposal with the most
votes is selected as the solution for the conflict. All agents that took part
in the dialogue permanently adapt the priority ordering. After the dialogue is
finished the agents will start working on resolving any further conflicts that
they may have. As a result of resolving conflict agent's plans may have been
changed, so they will have to communicate their updated plans to the other
agents, they will also have to recalculate where and when conflicts in their
plans occur.

There are two approaches to solving conflicts that involve multiple agents. The
first is to have a single dialogue for all agents, this is referred to as
Dialogue-based Partial Cooperative A* Plus (DPCA*+). Another approach is to let
agents solve conflicts in pairs only. If there is a conflict between agents
$a_1$, $a_2$, $a_3$ at time $t$ then there would be three dialogues: one
between $a_1$ and $a_2$, one between $a_1$ and $a_3$, and one between $a_2$ and
$a_3$. Say $a_1$ and $a_2$ are first to hold a dialogue which finishes with the
priority $a_1 > a_2$, meaning that $a_1$ has priority over $a_2$. $a_2$ is then
routed away from the conflict location. Next $a_1$ and $a_3$ will have to
resolve their remaining conflict. When this is also resolved as $a_1 > a_3$
then the conflict at $t$ is solved. It may be the case that $a_2$ and $a_3$ now
have a conflict at another position and they will have to solve this as well.
This algorithm is called Dialogue-based Partial Cooperative A* (DPCA*). This
approach of having multiple smaller dialogues can be faster than having a
single large complex dialogue.

To reduce the amount of computation required agents can store the paths that
they have calculated. Agents can then consult their path cache when evaluating
proposals. This allows them to pick a path that has been found earlier and use
that as a solution. The only restriction is that the cached path does not have
a conflict with agents of a higher priority. Using a cached path may increase
the number of conflicts, these will be solved by later dialogues and do not
influence the use of a cached path.

\subsection{Windowed Dialogue-based Partial Cooperative A*}
One of the issues with this algorithm is that all dialogues and computation
occur before execution of the plan. Because both planning and execution take
time without requiring the same resources it is also possible to do them at the
same time. This means that the plan can be executed while it is still being
constructed. One way of doing this is my applying a window to DPCA*. A
window $w$ determines that agents will use the above algorithm to solve all
conflicts that occur within $w$ time steps from their current position. Agents
will not coordinate past the boundary of the window, solving conflicts that
happen beyond that border is deferred to a later point in time. Periodically
the agents will move their window and solve any new conflicts in the window.
Because agents only coordinate in the window it is not necessary for them to
plan a path past the window boundary as well. Instead agents can plan a path
for the next $w$ time steps so they get closer to their goal. To achieve this
the graph can be changed so that the nodes at the window boundary connect
directly to the goal node. This can be achieved by changing the cost function
between adjacent nodes $P$ and $G$ \cite{silver2005}
\[
\text{\textsc{cost}(P,Q)} =
\begin{cases}
    0 & \text{if } P = Q = G, t < w \\
    \textsc{HeuristicDistance(P,G)} & \text{if } t = w \\
    1 & \text{otherwise}
\end{cases}
\]

Using the window spreads out the computation over the course of execution, but
it has other benefits as well. In large multi-agent systems the communication
between all agents may take a long time, by limiting the search using a window
there are only a limited number of agents that need to coordinate. This reduces
the initial planning time, as well as for each subsequent window. Agents that
are never in each other's window will never have to communicate with each other,
saving a lot of unnecessary communication and conflict detection overhead.
Windowing the search also has benefits in systems where agents can change their
destination during execution. Instead of recalculating the entire plan when
this happens, only agents within the window of the agent changing destination
have to update their plan. Agents that are not affected by the change in
destination do not have to update their plan. When there would be no window all
agents would have to recalculate and solve all conflicts again, even if they
would not need to update their plan, leading to wasted computational resources.

An overview of all proposed algorithms is given in \autoref{tbl:proposed}, the
categories shown are similar to those in \autoref{tbl:planning-overview}. All
methods are decoupled, not complete and do use priorities to ensure that agents
do not have conflicts. The meaning of communication has slightly changed, it
now restricts with which agents dialogues can be started.

\begin{table}
    \centering
    \caption{Comparison of proposed cooperative pathfinding algorithms. The
    communication and online columns are similar to that in
    \autoref{tbl:planning-overview}.}
    \label{tbl:proposed}
    \begin{tabular}{l|l|l|l}
        Algorithm & Na\"ive & Communication & Online \\ \hline
        PCA*   & Yes & All & No \\
        DPCA*  & No  & All & No \\
        WDPCA* & No  & Window & Yes \\
    \end{tabular}
\end{table}