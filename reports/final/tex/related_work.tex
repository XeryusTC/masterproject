\section{Related Work}\label{sec:related}
The following sections discuss previous work into cooperative pathfinding,
argumentation and coordination. The cooperative pathfinding problem requires
that agents are able to coordinate their movement, several different approaches
that achieve this will be discussed below. Computational argumentation has been
used to construct a plan for an agent, this field of argumentation is known as
\emph{practical reasoning}. It can also be used to make plans in a multi-agent
system, allowing the agents in such a system to coordinate. Argumentation has
not yet been used to find solutions for cooperative pathfinding, but research
in argumentation has been generic enough that it can be applied to a specific
field such as cooperative pathfinding. Work in coordination is also discussed
to help bridge the gap between argumentation and cooperative pathfinding.

\begin{table}[b]
    \centering
    \caption{Comparison of several cooperative pathfinding algorithms.}
    \label{tbl:planning-overview}
    \begin{tabular}{l|l|l|l|l|l}
        Method & Category & Complete & Priority & Comm. & Online \\
        \hline
        OD+ID \cite{standley2010,standley2011} & Centralized & Yes & No & All &
        No \\
        ICTS \cite{sharon2013} & Centralized & Yes & No & All & No \\
        ADPP \cite{cap2012} & Decoupled & No & Yes & All & No \\
        WHCA* \cite{silver2005} & Decoupled & No & Yes & Window
        & Yes \\
		DMRCP \cite{wei2016} & Decentralized & No & No & 2 nodes & Yes \\
        DiMPP \cite{chouhan2017} & Decentralized & Yes & Yes & Ring & No \\
        ORCA \cite{vandenberg2011} & Decentralized & No & No & None & Yes \\
        %Proposed & Decentralized & No & Partial & All & Before
    \end{tabular}
\end{table}

\subsection{Cooperative pathfinding}
In the grid world of \autoref{fig:world} each agent can take one of $b+1$
actions, where $b$ is the
current number of neighbouring cells, there is also a \textit{wait} action
where an agent does not move. All cells that are adjecent to the agents current
cell are considered to be neighbouring, this include cells that can be reached
by moving diagonally. The naive approach to finding conflict free paths
takes the Cartesian product of all $k$ state spaces and searches the new
combined state space with an algorithm like A*, this is also known as the
Standard Algorithm \cite{standley2010}. This results in a branching factor of
$(b+1)^k$, the branching factor grows exponentially in the number of agents and
the problem quickly becomes intractable even with efficient search algorithms
like A* \cite{sharon2013}.

There are a few common strategies that are used to tackle this problem.
Centralised methods use one single processor to calculate the paths for all
agents. They are often complete, they will find a solution to the problem if
one exists. This also means that they are slow. An alternative strategy is to
decouple the agents from each other. Each agent plans its own path and a
hierarchy is enforced on the agents such that agents with a lower priority need
to give way to agents with a higher priority. Decoupled methods sacrifice
completeness for speed. They often calculate the priorities at a central
processor, but can exploit the inherent parallelism in multi-agent systems to
calculate the paths. There are also decentralized reactive methods that will
only solve conflicts when they occur during plan execution.

\autoref{tbl:planning-overview} shows an overview of several algorithms and
summarises these properties, each algorithm is discussed in more detail below.
Some other aspects of the algorithms are also discussed, among these is the
communication range which may limit which agents are allowed to coordinate with
each other. Some of the algorithms create a plan before executing it, while
other algorithms interleave planning and execution. The latter category of
algorithms allow agents to move even though there is not a full solution net,
they are known as online algorithms.

One centralized method called Operator Decomposition (OD) deals with the
intractability of the problem by considering the possible moves of each agent
separately \cite{standley2010,standley2011}. Instead of taking the Cartesian
product of
the agents' state spaces it assigns actions to agents individually. This leads
to two different kind of states: in standard states no agent has been assigned
an action; in intermediate states some of the agents have been assigned an
action, when all agents are assigned an action it results in a new standard
state. Because intermediate states are considered individually the algorithm is
less likely to expand those intermediate states that are sub-optimal and thus
fewer states are generated. The result of this is that the branching factor
becomes $(b+1)$, but that the depth of the solution in the search tree grows
with a linear factor $k$. This makes finding a solution with an algorithm like
A* more tractable. OD is a complete and optimal algorithm, meaning that it will
always find a solution if one exists, and it will find the best solution.

On its own OD is not always very efficient, so an additional algorithm called
Independence Detection (ID) was introduced \cite{standley2010}. Before planning
$k$ groups are created, one for each agent, each agent is then placed in its
respective group. Each group then makes a
plan, when the paths for two groups conflict then one group is tasked with
finding a new conflict free path. If this also fails then the groups are merged
and a new plan is formed for the group using OD. This is repeated until a
set of conflict free paths for all agents has been found. Several variants on
ID+OD have been proposed, leading to the Optimal Anytime algorithm
\cite{standley2011} which will quickly find a solution and can then spend more
time on improving the solution. Because ID is an extension that can be applied
to any cooperative pathfinding algorithm OD+ID is still complete. Although
there is implicit priority in which order the agents are assigned actions, it
has no influence on the ability to find a solution or the quality of the
solution. Planning is completed before agents start executing it, and there is
no need to update the plan during execution.

Another centralized method is called the Increasing Cost Tree Search (ICTS)
which is a
two-fold search method \cite{sharon2013}. It consists of a high-level search on
an Increasing Cost Tree (ICT) which has a root node which contains the optimal
path costs for each individual agent. Each child node increases the path cost
for a different agent, so each level in the tree increases the sum of the path
costs by one. This tree is searched using breadth-first search, each node in
the tree will be searched using a low-level search. This low level search
generates all paths for each agents that is equal to the cost in the current
ICT node, it will then try to find a conflict free combination of these paths.
If such a set of paths exists then the algorithm is done, otherwise the
high-level search will continue to the next node in the ICT. Pruning can be
used to decrease the amount of duplicate nodes in the ICT, but it is possible
to use ID as well. The ICTS is a complete algorithm like OD+ID, but it is
faster in situations when the number of agents relative to the number of nodes
is high.

%The algorithms described above are both centralized approaches to cooperative
%pathfinding. They have a couple of characteristics in common, the most notable
%being completeness. Given enough time the algorithms will always find a
%solution if one exists, the solution that is found is often also the best
%solution, so the algorithms are also optimal. Centralized methods take the
%state spaces of all agents into account simultaneously to

The above algorithms both fall into the centralized category of solutions, these
can become very slow because of the state-space explosion. Decoupled methods
reduce the required calculation time by considering each agent separately. They
generally use the same three step approach;
\begin{enumerate}
    \item Find optimal paths for each agent independently of each other.
    \item Impose a hierarchy on the agents, often this is done by assigning
    them a unique priority.
    \item Make new plans for all the agents, this time an agent has to consider
    all agents with a higher priority as a moving obstacle. Agents with a lower
    priority can safely be ignored.
\end{enumerate}
This often leads to a set of conflict free plans. Finding the optimal priority
order is a combinatorial problem \cite{bennewitz2002}. A common algorithm of
assigning priorities first calculates a dependence hierarchy based on the paths
found in the first step, then priorities can be assigned such that agents have
a priority that is higher than that of agents that may block them. Circular
dependencies may mean that multiple priority schemes have to be tried. The
quality of the final solution depends highly on the priority ordering employed,
some of the possible priority schemes may not even lead to a solution. This
means that these kind of algorithms are not complete.

Most proposals for decoupled methods don't mention whether a central processor
must make the plans for all agents, or whether the agents can do it themselves.
Determining the prioritization scheme is often centralized, since a
single processor needs to determine all dependencies \cite{bennewitz2002}. One
method called Asynchronous Decentralized prioritized Planning \cite{cap2012}
exploits the inherent parallelism of a multi-robot team during the planning
stages. The method allows
agents to make their individual plans, after an agent has found a path it will
notify all agents with a lower priority of its (new) path. These lower priority
agents will then update their plans if conflicts arise, notifying lower
priority agents. These agents will then update their plans etc. The benefit of
this method is that agents can make a new plan as soon as any one higher
priority agent has send a conflicting plan, there is no need for agents to
wait for each other to finish their plans. This means that agents can plan
simultaneously and that some agents may finish planning before higher priority
agents if their paths are conflict free.

Windowed Hierarchical Cooperative A* (WHCA*) is a decoupled algorithm that has
been very successful in the video-game industry \cite{silver2005}. It uses a
reservation table to denote where agents plan to be and thus prevent other
agents from entering the same space at the same time. It requires that agents
have been assigned a priority ordering in which they plan so that they can take
each other's reservations into account. The amount of computation required
depends on the quality of the heuristic used during A* planning. Hierarchical
Cooperative A* (HCA*) uses an abstraction of the search space to obtain perfect
distance estimates. The reservation table and time dimension are ignored for
this abstract space so that the heuristic distance is the same as an agent's
optimal path. Agents still use the reservation table to find the conflict free
paths.
The search by
the above algorithm is windowed so that the reservation table is only used in
the window and the rest of the path is planned using the same abstract space as
HCA*,
effectively ignoring the other agent's actions outside of the window. The
window is moved at regular intervals and the agent's plan is updated. Because
of this the agents have no fixed priority, it varies based on the current
window. Computation is spread out over the time it takes for agents to get to
their destination, so there is no need to calculate all paths before execution,
they are instead updated during execution. Another issue that is solved is that
agents would reach their goal position, stay there and stop cooperating. This
can potentially block the paths of other agents.

The window of WHCA* limits the size of the reservation table the agents have to
take into account, limiting the communication range between agents to the size
of the window. This is in contrast to the other algorithms described above,
they all have an unlimited communication range. This means that those
algorithms allow (and often require) all agents to communicate with each other
to find a solution to the problem. Centralised algorithms use a single
processor to find the solution, this means that all agents indirectly
communicate with each other through the central processor.

One model of truly decentralized cooperative pathfinding called DMRCP has been
proposed by \cite{wei2016}. Agents move towards their destination and only
communicate with other agents that are at most two graph nodes away. They can
give each other commands like
move out of the way, follow me, wait etc. Agents are altruistic which means
that they are willing to make concessions during conflicts even if that means
that they will be at a disadvantage. Agents use various strategies to deal with
different conflict situations. Because of the limited communication range and
the various strategies employed the agents often need to recalculate the
optimal path to their destination during the execution of their old plan. This
approach works well, it requires slightly less computation time than OD+ID and
on average the agents only need two thirds of the number of movement steps to
reach their goal positions. Although completeness is not discussed, the
algorithm is based in decoupled methods which are generally not complete. Some
of the conflict resolving strategies used by the agents are able to solve
situations in which decoupled methods would not find a solution.
Because agents only communicate in a limited range there is no indication
whether agents will have conflicts at a later point in time. This lack of a
global overview means that agents must include strategies to resolve deadlocks
when they occur, there is no way to prevent deadlocks from happening.

Another method that doesn't use a central processor is Distributed Multi-agent
Path Planning (DiMPP) \cite{chouhan2017}. This is a distributed algorithm that
is complete, it is guaranteed that it will find a solution. To find the
solution all agents are only allowed to communicate in a unidirectional ring:
agent $a_i$ receives messages from $a_{i-1}$ and will send messages to
$a_{i+1}$, counting is modulo $n$ so agent $a_n$ will send its messages to
$a_1$. Sending and receiving messages is done by all agents at the same time.
The algorithm finds a solution by evaluating different priority orders. Naively
doing so would require the algorithm to evaluate $n!$ priority schemes for $n$
agents. Instead of this naive search the algorithm will only evaluate the
orderings
\[ \langle a_1, a_2, \ldots, a_{k-1}, a_k \rangle \]
\[ \langle a_2, a_3, \ldots, a_k, a_1 \rangle \]
\[ \vdots \]
\[ \langle a_k, a_1, \ldots a_{k-2}, a_{k-1} \rangle \]
Doing so means that the algorithm only has to evaluate $n$ orderings instead of
all possible $n!$ permutations. The algorithm finds the priority ordering
by letting $a_1$ find its optimal path. It will then send its path to $a_2$
which will find an optimal path that does not conflict with the path of $a_1$.
After this $a_2$ will send the global plan (the paths from $a_1$ and $a_2$) to
$a_3$. This process of calculating the optimal path for an agent considering
the constraints imposed by the paths of the algorithm continues around the
ring. If an agent $a_i$ is not able to find a path that has no conflicts with
the paths in the global plan then it will reset the global plan to contain no
paths. It will now start this procedure again by calculating an optimal path to
its destination and putting this as the only path in the global plan and
passing the global plan on to $a_{i+1}$. When an agent $a_j$ receives a global
plan in which it already has a path then it knows that all agents have found a
conflict free path and the algorithm has found a solution to the problem. In
the case that all agents have reset the global plan but no agent ever receives
a global plan that includes a path for itself then the algorithm has failed to
find a solution. DiMPP has been proven to be a complete algorithm, it will
evaluate all $n$ priority orderings which is sufficient to find a solution if
one exists. Proof for the completeness of the algorithm are given in
\cite[subsection 5.1]{chouhan2017}. The main idea is that an ordering that
starts with $a_1$ will never lead to a solution if any agent is not able to
find a conflict free path, it doesn't matter which agents will always have
conflicting paths. So when the algorithm evaluates
\[ \langle a_1, a_2, \ldots, a_{k-1}, a_k \rangle \]
and fails to find a solution it will not have to
consider the $(n-1)!$ other orderings where $a_1$ has the highest priority. The
algorithm requires no central processor but it does not fully exploit the
distribution in the multi-agent system. Because agent $a_{i+1}$ has to wait for
$a_i$ to finish planning there is a dependency between agents that means that
they will have to wait until other agents finish their calculations. This
algorithm is also not online like most decentralized algorithms because the
global plan will be constructed before it is executed.

Optimal Reciprocal Collision Avoidance (ORCA) \cite{vandenberg2011} is a
decentralised cooperative pathfinding algorithm that requires no communication
between agents. The only requirement is that all agents use the same method of
collision avoidance. Agents observe each other's position and velocity and use
that to construct a velocity obstacle (VO) to predict where the agent goes in
the next $\tau$ seconds. VOs can also be used to describe the static objects in
the environment. An agent will calculate the collision-avoiding velocities that
prevent the agents colliding within $\tau$ seconds. Multiple VOs can be
combined to limit the possible collision-avoiding velocities to prevent
colliding with multiple agents. ORCA assumes that all agents use the same
method of avoiding collisions. Because agents only observe the positions and
velocities of nearby agents the algorithm is purely reactive, congestions are
possible and become common when there are many agents moving in different
directions. It can be used together with a global planning algorithm that will
determine what the preferred direction for the agent is, and ORCA will try to
match this as closely as possible. Calculating VOs is so simple that the
algorithm can handle hundreds or even thousands of agents in real-time. ORCA
also fits quite well with human behaviour and is used in crowd simulations.

\subsection{Argumentation}
Cooperative pathfinding can be seen as an instance of a resource sharing
problem. From this perspective a conflict occurs when two agents try to access
the same resource at the same time. One way of dealing with this resource
sharing conflict is by constructing an argument with the goal of determining
which agent gets to access the resource at what time. Argumentation has long
been studied by philosophers, but in recent decades it has also been
extensively researched in
the field of Artificial Intelligence as well. In AI it has mainly been studied
in the fields of legal argumentation (AI \& Law), defeasible reasoning and
multi-agent systems. One of the main pillars is non-monotonic logic. A logic is
non-monotonic when a conclusion that follows based on the premises does not
necessarily hold any more when additional premises are added
\cite{vaneemeren2014}. A classic example of this is that birds can fly, so when
you see a bird you assume that it can fly. However, when you're told that the
bird is a penguin and that penguins can't fly then you will no longer conclude
that the bird can fly. A argument is defeasible when it can be defeated by
other arguments, in the previous example the fact that the bird that you see
can fly is defeasible.

Pollock distinguishes two different types of defeating arguments
\cite{pollock1995}. \emph{Rebutting defeaters} attack an argument directly and
give a reason for an opposite argument. \emph{Undercutting defeaters} do not
attack an argument directly, instead they attack the relation between an
argument and its support. The standard example given by Pollock is about an
object that looks red: "The ball looks red to John" is a support for John to
believe that the ball is red, but there may be a red light shining on the ball.
This is a undercutting defeater because it does not attack the conclusion
directly, instead it attacks the relation between the observation and the
conclusion that the ball is red, after all, a white object with a red light
shining on it will also look red.

Other researchers have formulated additional forms of defeaters, but they can
be distilled into three main forms \cite{vaneemeren2014}:
\begin{description}
	\item[Undermining defeaters] attack the premises or assumptions of an
	argument.
	\item[Undercutting defeaters] attack the connection between a set of
	reasons and the conclusion in an argument.
	\item[Rebutting defeaters] raise an argument in favour of an opposite
	conclusion, thereby attacking an argument.
\end{description}

A model of argumentation that adds mathematical structure to arguments has been
proposed in a highly influential paper by Dung \cite{dung1995}.
%A formal model of argumentation that introduces a structure to ease the
%computation of validity in arguments has been proposed in a highly influential
%paper by Dung \cite{dung1995}.
This work focused mainly on the argument attacks as a formal relation, giving
the model the name of abstract argumentation. The main concept is the
\emph{argumentation framework}, a directed graph in which the nodes form
arguments and the edges between them represent one argument attacking another.
An important concept that Dung introduced was that of admissibility of sets of
arguments. A set of arguments is admissible when it is conflict free and
acceptable. A set being conflict free means that no argument in the set attacks
another argument in the set. Acceptability means that when an argument is
attacked by another argument outside of the set, then the set attacks that
argument. In other words the admissible set defends itself from attacking
arguments. On top of this Dung formulated other semantics. The preferred
extension is the set theoretically maximal admissible set, that is, it is the
largest possible admissible set such that adding one argument from the
argumentation framework would make it not admissible. There is also the stable
extension, this is an admissible set that attacks all arguments that are not in
the set.

One important notion of an argumentation framework is that of the grounded
extension. This extension is simple to compute by labelling the arguments in
the argumentation framework as `justified' or `defeated':
\begin{enumerate}
	\item All unlabelled arguments $\alpha$ in the framework can be labelled as
	`justified' if all arguments that attack $\alpha$ are labelled as
	`defeated'. Note that when $\alpha$ is not attacked that it can then also
	be labelled as `justified'.
	\item All unlabelled arguments $\alpha$ in the framework that are attacked
	by an argument that has been labelled `justified' is labelled as `defeated'.
	\item Steps 1 and 2 are repeated until all arguments have been labelled.
\end{enumerate}
A finite argumentation framework is labelled in a finite amount of steps. All
arguments that have been labelled as `justified' are included in the grounded
extension. All arguments that have been labelled `defeated' are not included in
the grounded extension.

\subsubsection{Dialogues}
Multiple agents can have an argument through a dialogue. Walton and
Krabbe \cite{walton1995} proposed a typology of main dialogues that humans
partake in. They distinguish six main types of dialogues, it should be noted
that the list of dialogue types is not exhaustive. In \emph{information
seeking} dialogues some of the participating agents aim to gather information
from another agent that knows the anser. In \emph{inquiry} dialogues a group of
agents collectively seeks an answer to a question to which non of the
participating agents knows the answer on its own. \emph{Deliberation} dialogues
are about what course of action to take in a given situation. A
\emph{persuasion} dialogue occurs when an agent tries to convince on or
multiple other agents of its position. It is successful when the other agent(s)
adopt its position. Participants of \emph{negotiation} dialogues try to find a
division of a scarce resource that all agents can be satisfied with. Finally
\emph{eristic} dialogues are a verbal substitute for fighting. Note that most
actual dialogues combine these dialogue types.

Dialogues are often analysed in a game-theoretic sense, where the utterances
that agents can make are analogous to the moves in a game. Which utterances are
appropriate at each moment is then defined by the rules of the game. Most of
the research into dialogues follows this approach
\cite{prakken2006,prakken2009}. Most dialogue systems have a two language
set-up. The first is the topic language which is about what agents are
discussing and is typically a formal logic, it defines the context of the
dialogue. The second language is the communication language which specifies
which utterances can be made, what effects they have and the rules of outcome.
This latter language is at the core of dialogue games. Most dialogue systems
have the following syntax in common \cite{prakken2006,prakken2009,mcburney2009}.
\begin{description}
    \item[Commencement rules] Rules that concern when and how a dialogue can
    start and what its context is.
    \item[Locution rules] Which utterances are permitted are known as the
    locution rules. They may also define when an utterance is obligatory.
    Common locutions include asserting propositions, questioning or contesting
    assertions and justifying previous assertions after they have been
    questioned.
    \item[Commitments] Some locutions incur commitments on an agent which are
    subsequently put into the agent's commitment store. A dialogue system may
    limit which utterances an agent can make based on what is in its commitment
    store.
    \item[Speaker order] Most dialogue systems specify an order in which agents
    can speak, this can range from agents alternating turns to each agent being
    allowed to make an utterance at any time.
    \item[Outcome rules] These determine what the outcome of the dialogue is.
    Some systems define an outcome while the dialogue may continue and lead to
    a different outcome at a later point.
\end{description}

One model of deliberation dialogues is presented in \cite{mcburney2007}. It
consists of eight stages, starting with an \textbf{Open} stage and ending with
a \textbf{Close} stage. The other stages can occur multiple times during a
dialogue, as long as they occur following the rules of the dialogue game.
During the dialogue agents will collect the preferences, goals and other
constraints that need to be considered. Agents will then propose common plans
of action, when multiple plans have been proposed agents can specify which they
prefer. At some points an agent can recommend a plan after which all agents
will vote for that plan. The dialogue requires unanimity before the recommended
plan is adopted, but it allows for a voting mechanism to pick the most
preferred plan among many. By gathering the requirements of all agents during
the dialogue their local views combine into a single global view that can be
used to create a plan. There are variants of this model that require fewer
stages while still allowing for the same expressiveness
\cite{dunin-keplicz2011}.

\subsection{Coordination}
Argumentation can be used to allow coordination between agents by letting them
deliberate in a dialogue. Cooperative pathfinding is a particular instance of a
coordination problem. Before combining the two we can consider the
argumentative method of building plans for a single agent that was introduced
by Pollock \cite{pollock1995}. An agent starts out by  making a global plan
consisting only of coarse steps. This saves computation time and it defers
planning specific actions to a later time when more information about the
problem is available. When the agent reaches a step in a plan that is not
concrete enough yet it will start constructing a sub-plan for that step, it may
also do this when another planning process depends on it. This is done in
multiple levels leading to a hierarchical plan, the lowest level consists of
basic actions that are inherent in the agent (like lifting an arm). At the same
time the agent also keeps track of whether it is still possible to execute the
future steps in the plan. The agent will have to adapt its plans once it
notices that the remainder of the plan is not executable any more for any
reason. This allows an agent to adapt to a changing environment and changing
desires. Although this design focusses on planning actions for a single agent
it can easily be extended to planning for groups of agents.

Coordination in a multi-agent system can be done through partial global
planning \cite[pp. 202--204]{woodridge2009}, \cite{durfee1991}. The goal is to
let agents cooperate without any one of them formulating a global plan, instead
agents will coordinate with other agents only when they need to. This leads to
the construction of many small local plans, which can be communicated to other
agents as well. This means that eventually there will be a global plan that
covers all agents, but the agents themselves will only know a part of the
global plan. The global plan is implied by these partial global plans. Key to
partial global planning is that no agent needs to know the global plan, it only
needs to know which parts of the plan it is affected by. This approach is
similar to that of decoupled cooperative pathfinding because they use a similar
planning structure. Partial global planning starts out with letting each agent
make their individual goals, next agents communicate information on where plans
interact, finally they will alter their plans such that their actions are
better coordinated and there are no negative influences.
