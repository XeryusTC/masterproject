\section{Discussion}\label{sec:discussion}
\begin{table*}[h]
    \centering
    \caption{Comparison of several cooperative pathfinding algorithms.}
    \label{tbl:planning-overview}
    \begin{tabular}{ll|l|l|l|l|l}
        & & Category & Complete & Priority & Comm. & Eval. \\
        \hline
        OD+ID & \cite{standley2010,standley2011} & Centralized & Yes & No &
        All & No \\
        ICTS & \cite{sharon2013} & Centralized & Yes & No & All & No\\
        IADPP & \cite{cap2012} & Decoupled & No & Yes & All & No \\
        WHCA* & \cite{silver2005} & Decoupled & No & Yes & Window & No \\
        DMRCP & \cite{wei2016} & Decentralized & No & No & 2 nodes & No
        \\
        DiMPP & \cite{chouhan2017} & Decentralized & Yes & Yes & Ring & No
        \\ \hline
        PCA* & & Decentralized & No & Yes & All & No \\
        \multicolumn{2}{l|}{DPCA* / DPCA*+} & Decentralized & No & Yes & All & 
        Yes \\
    \end{tabular}
\end{table*}

A qualitative comparison of various cooperative pathfinding algorithms from 
the literature (as discussed in Section~\ref{sec:intro}) and those presented here is given in 
\autoref{tbl:planning-overview}. Algorithms come in three \emph{categories} of how  they 
calculate a set of paths: centralized, decoupled or decentralized. 
Algorithms can be \emph{complete} or not, depending on whether they find a solution if one exists. Some algorithms impose a \emph{priority} ordering on 
the agents. The 
\emph{communication} column indicates restrictions on which agents are allowed 
to communicate with each other. Finally the \emph{evaluation} column indicates 
whether 
agents can evaluate a plan by voting. The table shows that our three algorithms 
PCA*, DPCA* and DPCA*+ are decentralized, not complete, use priority ordering 
and involve communication between all agents, while DPCA* and DPCA*+ allow the 
agents to evaluate plans.


% TODO: scalability
% TODO: more compartmentalized than decoupled methods
% TODO: discuss WHCA* vs Continual Planning vs WDPCA*
% TODO: talk about agents changing destinations and how WDCPA* helps.
% TODO: extracting reasons for why solution was settled on
% TODO: going back to previous conflicts

The experimental outcomes in Figures~\ref{fig:perfgraph} and~\ref{fig:solved} 
show that PCA* took longest and solved the fewest problem instances of the 
algorithms compared, including OD+ID for most instances. An explanation is that 
the algorithm 
evaluates all permutations of partial priority orderings and selects one that has
minimal cost, without considering side-effects. This finding led to the 
development of DPCA* and DPCA*+, the fastest among those that were evaluated, 
showing the effectiveness of deliberation 
dialogues in this setting.

The slight difference between DPCA* and DPCA*+ in \autoref{fig:perfgraph} shows a tradeoff between computational resources and solution quality: DPCA* solves somewhat more problem 
instances while DPCA*+ finds solutions with slightly shorter path lengths.
The added computational cost of the more complex dialogues involving more than two agents that are available only in DPCA*+ leads to solutions with somewhat shorter paths. In~\autoref{fig:conflict-sizes} we see that these dialogues are much more rare than dialogues between 2 agents. 

Conventional cooperative pathfinding algorithms find an abstract solution for a 
pathfinding problem based on minimal cost. DPCA* and DPCA*+ add transparency to 
solution process by the consensus-forming deliberation dialogues. Agents
evaluate and vote on each proposal based on several 
criteria. This gives agents some influence over which solution 
is picked for a problem instance. The dialogues can therefore be regarded as 
explicit versions of a voting mechanism. For an outside observer, the agents' 
deliberation dialogues provide an explanation why a group of agents have picked 
a particular 
solution. 

The four stages of DPCA* and DPCA*+ dialogues are based on the 
four stages of \textsc{TeamLog}. The dialogue model of 
McBurney \emph{et al.}~\cite{mcburney2007} is less abstract and remains closer 
to natural dialogues, but is unnecessary to find 
priority orders for the agents. 
%By basing our algorithms on the decoupled method we separated 
%finding paths from 
%the argumentative process. 
%The ordering proposals influence which possible 
%paths are valid but they do not directly alter a path. 

\textsf{DeLP-POP} and \textsf{DeLP-MAPOP}~\cite{pardo2011,ferrando2012} can be regarded as an argumentative adaptation of OD+ID for general multi-agent planning, whereas our algorithms are confined to the abstract setting of cooperative pathfinding discussed in Section~\ref{sec:problem}.
%It is possible to use 
%\textsf{DeLP-MAPOP} to integrate planning and the deliberation process more 
%closely. In this case the agents can discuss individual actions in a plan and 
%propose alternate courses of action. This would result in an algorithm that 
%would be similar to a distributed version of OD+ID. It would also mean that the 
%paths are more tightly coupled than they are in our proposal. As a result 
%agents would be less flexible to change their plan to resolve other conflicts.
%
%Partial global planning ensures that all of our proposed algorithms 
%incrementally build towards a global well coordinated plan from the optimal 
%plans of individual agents. Because we incrementally construct a global plan we 
%can start deliberation dialogues for each local conflict. Here partial global 
%planning bridges the gap between conventional decoupled approaches and 
%computational argumentation. The norm in cooperative pathfinding is to 
%calculate a global priority ordering. Partial global planning allows us to 
%create partial priority ordering that imply a global priority ordering.
In contrast to this abstract setting, potential applications like traffic management are more complex and 
can require domain-specific arguments. For example, in an air traffic control 
system the agents could make a sensible argument for being given a high priority if their fuel 
levels are low.
In such applications, such domain-dependent reasons can provide helpful additional explanations why a certain solution was 
arrived at. In the present abstract setting, such explanations are restricted to the inspection of the evaluations 
made during dialogues. Adapting our abstract algorithms to include domain-dependent arguments may prove a useful extension of the present research.
%In applications where specific arguments made during a dialogue play a larger 
%role in finding the eventual solution then these arguments can be used to 
%extract reasons for why the solution is the most appropriate.

\section{Conclusion}
We have taken inspiration from ideas in the fields of cooperative pathfinding, 
computational 
argumentation and multi-agent planning to propose three algorithms that find 
conflict-free paths for groups of mobile agents. Agents engage in 
consensus-forming dialogues to resolve conflicts in their individually optimal 
plans. By 
deliberation dialogues between agents, the local views of 
agents are incrementally combined to form a global well-coordinated plan.
In contrast to other cooperative pathfinding algorithms it is possible for 
agents to indicate why they arrived at a particular solution.
Our algorithms are generally faster and can solve instances with more agents than a 
centralized state of the art algorithm and a recent decentralized algorithm.
