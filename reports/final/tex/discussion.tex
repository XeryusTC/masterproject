\section{Discussion}\label{sec:discussion}
% TODO: scalability
% TODO: more compartmentalized than decoupled methods

% DPCA*+ and PCA are simular, later is simpler
%With DPCA* agents resolve conflicts in pairs only while DPCA*+ can have larger
%groups of agents resolve a conflict. This only happens when multiple agents try
%to make conflicting moves at the same time. Allowing larger groups in a
%dialogue means that it gets more complex, this is reflected in the fact that
%PPCPF+ is slower overall and on individual problem instances. Larger groups of
%agents solving conflicts is not necessarily faster than pairs of agents solving
%conflicts. Several small pairwise dialogues can be very effective because the
%pair $a_1$ and $a_2$ resolving their conflict means that $a_2$ gets rerouted,
%this means that $a_2$'s conflict with $a_3$ is also resolved. This means that
%$a_1$ and $a_3$ only need to come to a solution. In the case that all three
%agents would find a solution in a single dialogue they will have to evaluate
%multiple proposals. The added complexity of three agents trying to find a
%solution to the conflict does not outweigh the speed of starting several
%dialogues which come to a conclusion quickly.

%window version being better
WDPCA* is able to solve more problem instances than DPCA* without a window. How
many more instances can be solved depends on the size of the window. As a
trade-off the amount of actions required for agents to reach their destination
increases. This effect get stronger when the window size gets smaller, when
$w=2$ most of the instances were solved, however the sum of the path lengths
was also the highest. That the speed of WDPCA*-2 is high is not surprising, it
reduces the algorithm to be almost reactive. There is barely any global
planning any more because agents don't look far ahead when trying to find their
way to the destination. Agents mainly solve conflicts that will
happen during the next time step, and some conflicts that happen one time step
later. Agents will also determine whether they have conflicts after action.
This shows that it is a valid strategy to create many small plans with
low computational effort instead of a complete global plan, but creating plans
that are too small has a negative effect on the quality of the solution.

There is an interaction between which algorithm is used and whether previously
found paths were stored in a cache and reused. From \autoref{fig:cache} it
becomes clear that a smaller window also means that the effect is smaller. With
a small window there are fewer conflicts, so agents will not have to
participate in conflicts very often. This means that they will not consult the
cache as often and therefore there is less of a speed-boost. On top of that the
paths are also shorter and easier to calculate, so retrieving a path from the
cache is not faster than calculating the path outright.

DPCA* is based on the A* algorithm \cite{hart1968} but this can be changed to
any pathfinding algorithm. Dialogues result in a priority ordering for agents,
which in turn determines which agents should be considered moving obstacles by
other agents. This is independent from which path planning algorithm is used,
it only puts constraints on where an agent can move to. As long as an algorithm
is able to handle moving obstacles or can be modified to handle moving
obstacles then it can be used instead of A* in DPCA*. This means that our
algorithm can be adapted to work with other discrete space algorithms, or even
continuous space algorithms like RRT* \cite{lavalle1998,lavalle2001}.

The world defined in \autoref{sec:problem} is an abstracted version of reality
which only allows for very basic arguments and evaluations to be made during a
dialogue. A real world situation like rail traffic management is more complex
and should allow for richer dialogues. In such systems agents should be able to
make domain specific arguments. For example, in a public transport rail system
planner agents could make arguments based on how passenger friendly a proposal
is. A proposal that delays a train from reaching a station in time so that
passengers can transfer to other stations is less passenger friendly than a
proposal which doesn't have this effect.

Cooperative pathfinding is a specific instance of a coordination problem. It is
possible to generalise the findings here to other resource sharing problems.
Instead of agents making moves in a grid world the agents would claim the use
of a resource for some amount of time. Two agents have conflicting claims when
they try to claim the same resource at the same or overlapping times. They can
resolve this conflict in claims by starting a deliberation dialogue and make
arguments about why an agent should be allowed to access the resource before
the other. They could also make proposals about how to resolve the conflict in
claims and agents should be able to argue for or against its adaptation. A
voting system similar to the one used by DPCA* could also be used.
