\section{Problem formulation}\label{sec:problem}

\begin{figure}[t]
    \centering
    \def\svgscale{.7}
    \input{images/world.pdf_tex}
    \caption{A small space shared by some agents. Obstacles are black, agents
        are circles inscribed with the agent's number ($a_i$), the destination
        for
        agent $a_i$ is given by $g_i$.}
    \label{fig:world}
\end{figure}

A shared space is divided into discrete cells such that it forms an 8-connected
grid. Some of the cells in this grid are static obstacles while the other cells
are open. A set of $k$ agents $\{a_1, a_k\}$ occupy cells within the grid, the
agents have goal positions $g_1, \ldots, g_k$. A set of paths need to be found,
one for each agent, such that each agent gets to its goal position without
crashing into any of the other agents.
A path consists of a series of moves, which can either
be to move to one of the neighbouring cells or \emph{wait} at the current
location. Each time step an agent can take exactly one of these actions.
An example initial configuration is shown in \autoref{fig:world}. The goal is
to find a series of moves, or paths, one for each agent, such that the agents
reach their destination without colliding with other agents or running into the
static obstacles.

The paths of two agents are not in conflict iff at no time step the agents
occupy the same cell, agents move along the same edge (swap positions), or
agents move along crossing edges. Obstacle cells can be considered as
stationary agents. Examples of each of these conflicts are given
in \autoref{fig:conflicts}, it shows that conflicts involving agents moving
along the same edge or moving along crossing edges can only occur when agents
are in neighbouring cells. A conflict in which agents move to the same cell can
happen whenever the agents are at most two actions away from each other. An
single action can result in an agent having multiple conflicts at the same
time. If \autoref{fig:conflict-position} had an agent $a_3$ in the top right
cell moving to the bottom middle cell then $a_2$ would have a conflict with
both $a_1$ and $a_3$ at the same time.
Agents are allowed to move along a diagonal even when the two cells on the
opposing diagonal are blocked, i.e. $a_5$ in \autoref{fig:world} can move to its
destination in a single time step. An agent $a_i$ can move to a cell occupied by
agent $a_j$ given that $a_j$ will move to a different cell at the same time.
Agents $a_1, \ldots a_4$ in \autoref{fig:world} can move their respective
destinations
by ``rotating'' clockwise. They can do this in a single time step without
requiring any additional empty cells. $a_7$ and $a_8$ cannot move to their
destinations in a single time step because that would mean that they move along
crossing edges at the same time. They can also not swap places because then
they would be travelling along the same edge.

\begin{figure}[b]
	\centering
	\begin{subfigure}[b]{.3\textwidth}
		\centering
	    \def\svgscale{.7}
		\input{images/conflict1.pdf_tex}
		\caption{Moving to the same position.}
		\label{fig:conflict-position}
	\end{subfigure}
	~
	\begin{subfigure}[b]{.3\textwidth}
		\centering
	    \def\svgscale{.7}
		\input{images/conflict2.pdf_tex}
		\caption{Moving along the same edge.}
		\label{fig:conflict-same}
	\end{subfigure}
	~
	\begin{subfigure}[b]{.3\textwidth}
		\centering
  \def\svgscale{.7}
		\input{images/conflict3.pdf_tex}
		\caption{Moving on crossing edges.}
		\label{fig:conflict-crossing}
	\end{subfigure}
	\caption{Examples of conflicting actions. Agents are circles inscribed with
	$a_j$, their movements are indicated by the arrows starting in the cell
	they occupy, the action ends in the cell that the arrow points to.}
	\label{fig:conflicts}
\end{figure}